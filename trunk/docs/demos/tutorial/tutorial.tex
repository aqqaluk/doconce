\documentclass{article}
\usepackage{hyperref,relsize,,epsfig,makeidx}
\usepackage[latin1]{inputenc}
% required by ptex2tex:
\usepackage{graphicx,hyperref,relsize,fancyvrb,epsfig}
\usepackage{a4,amsmath,amssymb,framed,subfigure}
\usepackage[usenames]{color}
%\usepackage{ptex2tex}  % replaces the above

\makeindex

\begin{document}




\begin{center}
{\LARGE\bf Doconce: Document Once, Include Anywhere}
\end{center}



\begin{center}
{\bf Hans Petter Langtangen${}^{1, 2}$} \\ [0mm]
\end{center}

\begin{center}
{\small ${}^1$Simula Research Laboratory} \\ [-1.0mm]
\end{center}

\begin{center}
{\small ${}^2$University of Oslo} \\ [-1.0mm]
\end{center}

%\vspace{4mm}




\begin{center}
August 25, 2010
\end{center}


% lines beginning with # are comment lines

\begin{itemize}
 \item When writing a note, report, manual, etc., do you find it difficult
   to choose the typesetting format? That is, to choose between plain
   (email-like) text, Wiki, Word/OpenOffice, {\LaTeX}, HTML,
   reStructuredText, Sphinx, XML, etc.  Would it be convenient to
   start with some very simple text-like format that easily converts
   to the formats listed above, and at some later stage eventually go
   with a particular format?

 \item Do you find it problematic that you have the same information
   scattered around in different documents in different typesetting
   formats? Would it be a good idea to write things once, in one
   place, and include it anywhere?
\end{itemize}

\noindent
If any of these questions are of interest, you should keep on reading.


\section{The Doconce Concept}

Doconce is two things:

\begin{enumerate}
 \item Doconce is a working strategy for documenting software in a single
    place and avoiding duplication of information. The slogan is:
    "Write once, include anywhere". This requires that what you write
    can be transformed to many different formats for a variety of
    documents (manuals, tutorials, books, doc strings, source code
    documentation, etc.).

 \item Doconce is a simple and minimally tagged markup language that can
    be used for the above purpose. That is, the Doconce format look
    like ordinary ASCII text (much like what you would use in an
    email), but the text can be transformed to numerous other formats,
    including HTML, Wiki, {\LaTeX}, PDF, reStructuredText (reST), Sphinx,
    Epytext, and also plain text (where non-obvious formatting/tags are
    removed for clear reading in, e.g., emails). From reStructuredText
    you can go to XML, HTML, {\LaTeX}, PDF, OpenOffice, and from the
    latter to RTF and MS Word.
\end{enumerate}

\noindent
Doconce was particularly written for the following sample applications:

\begin{itemize}
  \item Large books written in {\LaTeX}, but where many pieces (computer demos,
    projects, examples) can be written in Doconce to appear in other
    contexts in other formats, including plain HTML, Sphinx, or MS Word.

  \item Software documentation, primarily Python doc strings, which one wants
    to appear as plain untagged text for viewing in Pydoc, as reStructuredText
    for use with Sphinx, as wiki text when publishing the software at
    googlecode.com, and as {\LaTeX} integrated in, e.g., a master's thesis.

  \item Quick memos, which start as plain text in email, then some small
    amount of Doconce tagging is added, before the memos can appear as
    MS Word documents or in wikis.
\end{itemize}

\noindent


\section{What Does Doconce Look Like?}

Doconce text looks like ordinary text, but there are some almost invisible
text constructions that allow you to control the formating. For example,

\begin{itemize}
  \item bullet lists arise from lines starting with an asterisk,

  \item \emph{emphasized words} are surrounded by asterisks, 

  \item \textbf{words in boldface} are surrounded by underscores, 

  \item words from computer code are enclosed in back quotes and 
    then typeset verbatim,

  \item blocks of computer code can easily be included, also from source files,

  \item blocks of {\LaTeX} mathematics can easily be included,

  \item there is support for both {\LaTeX} and text-like inline mathematics,

  \item figures with captions, URLs with links, labels and references
    are supported,

  \item comments can be inserted throughout the text,

  \item a preprocessor (much like the C preprocessor) is integrated so
    other documents (files) can be included and large portions of text
    can be defined in or out of the text.
\end{itemize}

\noindent
Here is an example of some simple text written in the Doconce format:
\begin{Verbatim}[fontsize=\fontsize{9pt}{9pt},tabsize=8,baselinestretch=0.85,
fontfamily=tt,xleftmargin=7mm]
===== A Subsection with Sample Text =====
label{my:first:sec}

Ordinary text looks like ordinary text, and the tags used for
_boldface_ words, *emphasized* words, and `computer` words look
natural in plain text.  Lists are typeset as you would do in an email,

  * item 1
  * item 2
  * item 3

Lists can also have automatically numbered items instead of bullets,

  o item 1
  o item 2
  o item 3

URLs with a link word are possible, as in http://folk.uio.no/hpl<hpl>.
Just a file link goes like URL:"tutorial.do.txt". References
to sections may use logical names as labels (e.g., a "label" command right
after the section title), as in the reference to 
Chapter ref{my:first:sec}.

Tables are also supperted, e.g.,

  |--------------------------------|
  |time  | velocity | acceleration |
  |--------------------------------|
  | 0.0  | 1.4186   | -5.01        |
  | 2.0  | 1.376512 | 11.919       |
  | 4.0  | 1.1E+1   | 14.717624    |
  |--------------------------------|
\end{Verbatim}
\noindent
The Doconce text above results in the following little document:

\subsection{A Subsection with Sample Text}

\label{my:first:sec}

Ordinary text looks like ordinary text, and the tags used for
\textbf{boldface} words, \emph{emphasized} words, and {\fontsize{10pt}{10pt}\verb!computer!} words look
natural in plain text.  Lists are typeset as you would do in an email,

\begin{itemize}
  \item item 1

  \item item 2

  \item item 3
\end{itemize}

\noindent
Lists can also have numbered items instead of bullets, just use an {\fontsize{10pt}{10pt}\verb!o!}
(for ordered) instead of the asterisk:

\begin{enumerate}
 \item item 1

 \item item 2

 \item item 3
\end{enumerate}

\noindent
URLs with a link word are possible, as in \href{http://folk.uio.no/hpl}{hpl}.
Just a file link goes like \href{tutorial.do.txt}{tutorial.do.txt}. References
to sections may use logical names as labels (e.g., a "label" command right
after the section title), as in the reference to 
Chapter~\ref{my:first:sec}.

Tables are also supperted, e.g.,


\begin{quote}\begin{tabular}{ccc}
\hline
\multicolumn{1}{c}{time} & \multicolumn{1}{c}{velocity} & \multicolumn{1}{c}{acceleration} \\
\hline
0.0          & 1.4186       & -5.01        \\
2.0          & 1.376512     & 11.919       \\
4.0          & 1.1E+1       & 14.717624    \\
\hline
\end{tabular}\end{quote}

\noindent

\subsection{Mathematics and Computer Code}

Inline mathematics, such as $\nu = \sin(x)$,
allows the formula to be specified both as {\LaTeX} and as plain text.
This results in a professional {\LaTeX} typesetting, but in other formats
the text version normally looks better than raw {\LaTeX} mathematics with
backslashes. An inline formula like $\nu = \sin(x)$ is
typeset as
\begin{Verbatim}[fontsize=\fontsize{9pt}{9pt},tabsize=8,baselinestretch=0.85,
fontfamily=tt,xleftmargin=7mm]
$\nu = \sin(x)$|$v = sin(x)$
\end{Verbatim}
\noindent
The pipe symbol acts as a delimiter between {\LaTeX} code and the plain text
version of the formula.

Blocks of mathematics are better typeset with raw {\LaTeX}, inside
{\fontsize{10pt}{10pt}\verb!!bt!} and {\fontsize{10pt}{10pt}\verb!!et!} (begin tex / end tex) instructions. 
The result looks like this:

\begin{eqnarray}
{\partial u\over\partial t} &=& \nabla^2 u + f,\label{myeq1}\\
{\partial v\over\partial t} &=& \nabla\cdot(q(u)\nabla v) + g
\end{eqnarray}
Of course, such blocks only looks nice in {\LaTeX}. The raw
{\LaTeX} syntax appears in all other formats (but can still be useful
for those who can read {\LaTeX} syntax).

You can have blocks of computer code, starting and ending with
{\fontsize{10pt}{10pt}\verb!!bc!} and {\fontsize{10pt}{10pt}\verb!!ec!} instructions, respectively. Such blocks look like
\begin{Verbatim}[fontsize=\fontsize{9pt}{9pt},tabsize=8,baselinestretch=0.85,
fontfamily=tt,xleftmargin=7mm]
from math import sin, pi
def myfunc(x):
    return sin(pi*x)

import integrate
I = integrate.trapezoidal(myfunc, 0, pi, 100)
\end{Verbatim}
\noindent

One can also copy computer code directly from files, either the
complete file or specified parts.  Computer code is then never
duplicated in the documentation (important for the principle of
avoiding copying information!).

Another document can be included by writing {\fontsize{10pt}{10pt}\verb!#include "mynote.do.txt"!}
on a line starting with (another) hash sign.  Doconce documents have
extension {\fontsize{10pt}{10pt}\verb!do.txt!}. The {\fontsize{10pt}{10pt}\verb!do!} part stands for doconce, while the
trailing {\fontsize{10pt}{10pt}\verb!.txt!} denotes a text document so that editors gives you the
right writing enviroment for plain text.

\subsection{Macros (Newcommands), Cross-References, Index, and Bibliography}

Doconce supports a type of macros via a {\LaTeX}-style \emph{newcommand}
construction.  The newcommands defined in a file with name
{\fontsize{10pt}{10pt}\verb!newcommand_replace.tex!} are expanded when Doconce is filtered to
other formats, except for {\LaTeX} (since {\LaTeX} performs the expansion
itself).  Newcommands in files with names {\fontsize{10pt}{10pt}\verb!newcommands.tex!} and
{\fontsize{10pt}{10pt}\verb!newcommands_keep.tex!} are kept unaltered when Doconce text is
filtered to other formats, except for the Sphinx format. Since Sphinx
understands {\LaTeX} math, but not newcommands if the Sphinx output is
HTML, it makes most sense to expand all newcommands.  Normally, a user
will put all newcommands that appear in math blocks surrounded by
{\fontsize{10pt}{10pt}\verb!!bt!} and {\fontsize{10pt}{10pt}\verb!!et!} in {\fontsize{10pt}{10pt}\verb!newcommands_keep.tex!} to keep them unchanged, at
least if they contribute to make the raw {\LaTeX} math text easier to
read in the formats that cannot render {\LaTeX}.  Newcommands used
elsewhere throughout the text will usually be placed in
{\fontsize{10pt}{10pt}\verb!newcommands_replace.tex!} and expanded by Doconce.  The definitions of
newcommands in the {\fontsize{10pt}{10pt}\verb!newcommands*.tex!} files \emph{must} appear on a single
line (multi-line newcommands are too hard to parse with regular
expressions).

Recent versions of Doconce also offer cross referencing, typically one
can define labels below (sub)sections, in figure captions, or in
equations, and then refer to these later. Entries in an index can be
defined and result in an index at the end for the {\LaTeX} and Sphinx
formats. Citations to literature, with an accompanying bibliography in
a file, are also supported. The syntax of labels, references,
citations, and the bibliography closely resembles that of {\LaTeX},
making it easy for Doconce documents to be integrated in {\LaTeX}
projects (manuals, books). For further details on functionality and
syntax we refer to the {\fontsize{10pt}{10pt}\verb!docs/manual/manual.do.txt!} file (see the
\href{https://doconce.googlecode.com/hg/trunk/docs/demos/manual/index.html}{demo
page} for various formats of this document).


% Example on including another Doconce file:


\section{From Doconce to Other Formats}

Transformation of a Doconce document to various other
formats applies the script {\fontsize{10pt}{10pt}\verb!doconce2format!}:
\begin{Verbatim}[fontsize=\fontsize{9pt}{9pt},tabsize=8,baselinestretch=0.85,
fontfamily=tt,xleftmargin=7mm]
Unix/DOS> doconce2format format mydoc.do.txt
\end{Verbatim}
\noindent
The {\fontsize{10pt}{10pt}\verb!preprocess!} program is always used to preprocess the file first,
and options to {\fontsize{10pt}{10pt}\verb!preprocess!} can be added after the filename. For example,
\begin{Verbatim}[fontsize=\fontsize{9pt}{9pt},tabsize=8,baselinestretch=0.85,
fontfamily=tt,xleftmargin=7mm]
Unix/DOS> doconce2format LaTeX mydoc.do.txt -Dextra_sections
\end{Verbatim}
\noindent
The variable {\fontsize{10pt}{10pt}\verb!FORMAT!} is always defined as the current format when
running {\fontsize{10pt}{10pt}\verb!preprocess!}. That is, in the last example, {\fontsize{10pt}{10pt}\verb!FORMAT!} is
defined as {\fontsize{10pt}{10pt}\verb!LaTeX!}. Inside the Doconce document one can then perform
format specific actions through tests like {\fontsize{10pt}{10pt}\verb!#if FORMAT == "LaTeX"!}.

\subsection{HTML}

Making an HTML version of a Doconce file {\fontsize{10pt}{10pt}\verb!mydoc.do.txt!}
is performed by
\begin{Verbatim}[fontsize=\fontsize{9pt}{9pt},tabsize=8,baselinestretch=0.85,
fontfamily=tt,xleftmargin=7mm]
Unix/DOS> doconce2format HTML mydoc.do.txt
\end{Verbatim}
\noindent
The resulting file {\fontsize{10pt}{10pt}\verb!mydoc.html!} can be loaded into any web browser for viewing.

\subsection{{\LaTeX}}

Making a {\LaTeX} file {\fontsize{10pt}{10pt}\verb!mydoc.tex!} from {\fontsize{10pt}{10pt}\verb!mydoc.do.txt!} is done in two steps:
% Note: putting code blocks inside a list is not successful in many
% formats - the text may be messed up. A better choice is a paragraph
% environment, as used here.

\paragraph{Step 1.}
Filter the doconce text to a pre-{\LaTeX} form {\fontsize{10pt}{10pt}\verb!mydoc.p.tex!} for
     {\fontsize{10pt}{10pt}\verb!ptex2tex!}:
\begin{Verbatim}[fontsize=\fontsize{9pt}{9pt},tabsize=8,baselinestretch=0.85,
fontfamily=tt,xleftmargin=7mm]
Unix/DOS> doconce2format LaTeX mydoc.do.txt
\end{Verbatim}
\noindent
{\LaTeX}-specific commands ("newcommands") in math formulas and similar
can be placed in a file {\fontsize{10pt}{10pt}\verb!newcommands.tex!}. If this file is present,
it is included in the {\LaTeX} document so that your commands are
defined.

\paragraph{Step 2.}
Run {\fontsize{10pt}{10pt}\verb!ptex2tex!} (if you have it) to make a standard {\LaTeX} file,
\begin{Verbatim}[fontsize=\fontsize{9pt}{9pt},tabsize=8,baselinestretch=0.85,
fontfamily=tt,xleftmargin=7mm]
Unix/DOS> ptex2tex mydoc
\end{Verbatim}
\noindent
or just perform a plain copy,
\begin{Verbatim}[fontsize=\fontsize{9pt}{9pt},tabsize=8,baselinestretch=0.85,
fontfamily=tt,xleftmargin=7mm]
Unix/DOS> cp mydoc.p.tex mydoc.tex
\end{Verbatim}
\noindent
The {\fontsize{10pt}{10pt}\verb!ptex2tex!} tool makes it possible to easily switch between many
different fancy formattings of computer or verbatim code in {\LaTeX}
documents.
Finally, compile {\fontsize{10pt}{10pt}\verb!mydoc.tex!} the usual way and create the PDF file:
\begin{Verbatim}[fontsize=\fontsize{9pt}{9pt},tabsize=8,baselinestretch=0.85,
fontfamily=tt,xleftmargin=7mm]
Unix/DOS> latex mydoc
Unix/DOS> latex mydoc
Unix/DOS> makeindex mydoc   # if index
Unix/DOS> bibitem mydoc     # if bibliography
Unix/DOS> latex mydoc
Unix/DOS> dvipdf mydoc
\end{Verbatim}
\noindent

\subsection{Plain ASCII Text}

We can go from Doconce "back to" plain untagged text suitable for viewing
in terminal windows, inclusion in email text, or for insertion in
computer source code:
\begin{Verbatim}[fontsize=\fontsize{9pt}{9pt},tabsize=8,baselinestretch=0.85,
fontfamily=tt,xleftmargin=7mm]
Unix/DOS> doconce2format plain mydoc.do.txt  # results in mydoc.txt
\end{Verbatim}
\noindent

\subsection{reStructuredText}

Going from Doconce to reStructuredText gives a lot of possibilities to
go to other formats. First we filter the Doconce text to a
reStructuredText file {\fontsize{10pt}{10pt}\verb!mydoc.rst!}:
\begin{Verbatim}[fontsize=\fontsize{9pt}{9pt},tabsize=8,baselinestretch=0.85,
fontfamily=tt,xleftmargin=7mm]
Unix/DOS> doconce2format rst mydoc.do.txt
\end{Verbatim}
\noindent
We may now produce various other formats:
\begin{Verbatim}[fontsize=\fontsize{9pt}{9pt},tabsize=8,baselinestretch=0.85,
fontfamily=tt,xleftmargin=7mm]
Unix/DOS> rst2html.py  mydoc.rst > mydoc.html # HTML
Unix/DOS> rst2latex.py mydoc.rst > mydoc.tex  # LaTeX
Unix/DOS> rst2xml.py   mydoc.rst > mydoc.xml  # XML
Unix/DOS> rst2odt.py   mydoc.rst > mydoc.odt  # OpenOffice
\end{Verbatim}
\noindent
The OpenOffice file {\fontsize{10pt}{10pt}\verb!mydoc.odt!} can be loaded into OpenOffice and
saved in, among other things, the RTF format or the Microsoft Word format.
That is, one can easily go from Doconce to Microsoft Word.

\subsection{Sphinx}

Sphinx documents can be created from a Doconce source in a few steps.

\paragraph{Step 1.}
Translate Doconce into the Sphinx dialect of
the reStructuredText format:
\begin{Verbatim}[fontsize=\fontsize{9pt}{9pt},tabsize=8,baselinestretch=0.85,
fontfamily=tt,xleftmargin=7mm]
Unix/DOS> doconce2format sphinx mydoc.do.txt
\end{Verbatim}
\noindent

\paragraph{Step 2.}
Create a Sphinx root directory with a {\fontsize{10pt}{10pt}\verb!conf.py!} file, 
either manually or by using the interactive {\fontsize{10pt}{10pt}\verb!sphinx-quickstart!}
program. Here is a scripted version of the steps with the latter:
\begin{Verbatim}[fontsize=\fontsize{9pt}{9pt},tabsize=8,baselinestretch=0.85,
fontfamily=tt,xleftmargin=7mm]
mkdir sphinx-rootdir
sphinx-quickstart <<EOF
sphinx-rootdir
n
_
Name of My Sphinx Document
Author
version
version
.rst
index
n
y
n
n
n
n
y
n
n
y
y
y
EOF
\end{Verbatim}
\noindent

\paragraph{Step 3.}
Move the {\fontsize{10pt}{10pt}\verb!tutorial.rst!} file to the Sphinx root directory:
\begin{Verbatim}[fontsize=\fontsize{9pt}{9pt},tabsize=8,baselinestretch=0.85,
fontfamily=tt,xleftmargin=7mm]
Unix/DOS> mv mydoc.rst sphinx-rootdir
\end{Verbatim}
\noindent
If you have figures in your document, the relative paths to those will
be invalid when you work with {\fontsize{10pt}{10pt}\verb!mydoc.rst!} in the {\fontsize{10pt}{10pt}\verb!sphinx-rootdir!}
directory. Either edit {\fontsize{10pt}{10pt}\verb!mydoc.rst!} so that figure file paths are correct,
or simply copy your figure directory to {\fontsize{10pt}{10pt}\verb!sphinx-rootdir!} (if all figures
are located in a subdirectory).

\paragraph{Step 4.}
Edit the generated {\fontsize{10pt}{10pt}\verb!index.rst!} file so that {\fontsize{10pt}{10pt}\verb!mydoc.rst!}
is included, i.e., add {\fontsize{10pt}{10pt}\verb!mydoc!} to the {\fontsize{10pt}{10pt}\verb!toctree!} section so that it becomes
\begin{Verbatim}[fontsize=\fontsize{9pt}{9pt},tabsize=8,baselinestretch=0.85,
fontfamily=tt,xleftmargin=7mm]
.. toctree::
   :maxdepth: 2

   mydoc
\end{Verbatim}
\noindent
(The spaces before {\fontsize{10pt}{10pt}\verb!mydoc!} are important!)

\paragraph{Step 5.}
Generate, for instance, an HTML version of the Sphinx source:
\begin{Verbatim}[fontsize=\fontsize{9pt}{9pt},tabsize=8,baselinestretch=0.85,
fontfamily=tt,xleftmargin=7mm]
make clean   # remove old versions
make html
\end{Verbatim}
\noindent
Many other formats are also possible.

\paragraph{Step 6.}
View the result:
\begin{Verbatim}[fontsize=\fontsize{9pt}{9pt},tabsize=8,baselinestretch=0.85,
fontfamily=tt,xleftmargin=7mm]
Unix/DOS> firefox _build/html/index.html
\end{Verbatim}
\noindent

\subsection{Google Code Wiki}

There are several different wiki dialects, but Doconce only support the
one used by \href{http://code.google.com/p/support/wiki/WikiSyntax}{Google Code}.
The transformation to this format, called {\fontsize{10pt}{10pt}\verb!gwiki!} to explicitly mark
it as the Google Code dialect, is done by
\begin{Verbatim}[fontsize=\fontsize{9pt}{9pt},tabsize=8,baselinestretch=0.85,
fontfamily=tt,xleftmargin=7mm]
Unix/DOS> doconce2format gwiki mydoc.do.txt
\end{Verbatim}
\noindent
You can then open a new wiki page for your Google Code project, copy
the {\fontsize{10pt}{10pt}\verb!mydoc.gwiki!} output file from {\fontsize{10pt}{10pt}\verb!doconce2format!} and paste the
file contents into the wiki page. Press \textbf{Preview} or \textbf{Save Page} to
see the formatted result.

\subsection{Demos}

The current text is generated from a Doconce format stored in the file
\begin{Verbatim}[fontsize=\fontsize{9pt}{9pt},tabsize=8,baselinestretch=0.85,
fontfamily=tt,xleftmargin=7mm]
docs/tutorial/tutorial.do.txt
\end{Verbatim}
\noindent
The file {\fontsize{10pt}{10pt}\verb!make.sh!} in the {\fontsize{10pt}{10pt}\verb!tutorial!} directory of the
Doconce source code contains a demo of how to produce a variety of
formats.  The source of this tutorial, {\fontsize{10pt}{10pt}\verb!tutorial.do.txt!} is the
starting point.  Running {\fontsize{10pt}{10pt}\verb!make.sh!} and studying the various generated
files and comparing them with the original {\fontsize{10pt}{10pt}\verb!tutorial.do.txt!} file,
gives a quick introduction to how Doconce is used in a real case.
\href{https://doconce.googlecode.com/hg/trunk/docs/demos/tutorial/index.html}{Here} 
is a sample of how this tutorial looks in different formats.

There is another demo in the {\fontsize{10pt}{10pt}\verb!docs/manual!} directory which
translates the more comprehensive documentation, {\fontsize{10pt}{10pt}\verb!manual.do.txt!}, to
various formats. The {\fontsize{10pt}{10pt}\verb!make.sh!} script runs a set of translations.

\subsection{Dependencies}

Doconce depends on the Python package
\href{http://code.google.com/p/preprocess/}{preprocess}.  To make {\LaTeX}
documents (without going through the reStructuredText format) you also
need \href{http://code.google.com/p/ptex2tex}{ptex2tex} and some style files
that ptex2tex potentially makes use of.  Going from reStructuredText
to formats such as XML, OpenOffice, HTML, and {\LaTeX} requires
\href{http://docutils.sourceforge.net/}{docutils}.  Making Sphinx documents
requires of course \href{http://sphinx.pocoo.org}{sphinx}.

\subsection{The Doconce Documentation Strategy for User Manuals}

Doconce was particularly made for writing tutorials or user manuals
associated with computer codes. The text is written in Doconce format
in separate files. {\LaTeX}, HTML, XML, and other versions of the text
is easily produced by the {\fontsize{10pt}{10pt}\verb!doconce2format!} script and standard tools.
A plain text version is often wanted for the computer source code,
this is easy to make, and then one can use
{\fontsize{10pt}{10pt}\verb!#include!} statements in the computer source code to automatically
get the manual or tutorial text in comments or doc strings.
Below is a worked example.

Consider an example involving a Python module in a {\fontsize{10pt}{10pt}\verb!basename.p.py!} file.
The {\fontsize{10pt}{10pt}\verb!.p.py!} extension identifies this as a file that has to be
preprocessed) by the {\fontsize{10pt}{10pt}\verb!preprocess!} program. 
In a doc string in {\fontsize{10pt}{10pt}\verb!basename.p.py!} we do a preprocessor include
in a comment line, say
\begin{Verbatim}[fontsize=\fontsize{9pt}{9pt},tabsize=8,baselinestretch=0.85,
fontfamily=tt,xleftmargin=7mm]
#    #include "docstrings/doc1.dst.txt
\end{Verbatim}
\noindent
% 
% Note: we insert an error right above as the right quote is missing.
% Then preprocess skips the statement, otherwise it gives an error
% message about a missing file docstrings/doc1.dst.txt (which we don't
% have, it's just a sample file name). Also note that comment lines
% must not come before a code block for the rst/st/epytext formats to work.
% 
The file {\fontsize{10pt}{10pt}\verb!docstrings/doc1.dst.txt!} is a file filtered to a specific format
(typically plain text, reStructedText, or Epytext) from an original
"singleton" documentation file named {\fontsize{10pt}{10pt}\verb!docstrings/doc1.do.txt!}. The {\fontsize{10pt}{10pt}\verb!.dst.txt!}
is the extension of a file filtered ready for being included in a doc
string ({\fontsize{10pt}{10pt}\verb!d!} for doc, {\fontsize{10pt}{10pt}\verb!st!} for string).

For making an Epydoc manual, the {\fontsize{10pt}{10pt}\verb!docstrings/doc1.do.txt!} file is
filtered to {\fontsize{10pt}{10pt}\verb!docstrings/doc1.epytext!} and renamed to
{\fontsize{10pt}{10pt}\verb!docstrings/doc1.dst.txt!}.  Then we run the preprocessor on the
{\fontsize{10pt}{10pt}\verb!basename.p.py!} file and create a real Python file
{\fontsize{10pt}{10pt}\verb!basename.py!}. Finally, we run Epydoc on this file. Alternatively, and
nowadays preferably, we use Sphinx for API documentation and then the
Doconce {\fontsize{10pt}{10pt}\verb!docstrings/doc1.do.txt!} file is filtered to
{\fontsize{10pt}{10pt}\verb!docstrings/doc1.rst!} and renamed to {\fontsize{10pt}{10pt}\verb!docstrings/doc1.dst.txt!}. A
Sphinx directory must have been made with the right {\fontsize{10pt}{10pt}\verb!index.rst!} and
{\fontsize{10pt}{10pt}\verb!conf.py!} files. Going to this directory and typing {\fontsize{10pt}{10pt}\verb!make html!} makes
the HTML version of the Sphinx API documentation.

The next step is to produce the final pure Python source code. For
this purpose we filter {\fontsize{10pt}{10pt}\verb!docstrings/doc1.do.txt!} to plain text format
({\fontsize{10pt}{10pt}\verb!docstrings/doc1.txt!}) and rename to {\fontsize{10pt}{10pt}\verb!docstrings/doc1.dst.txt!}. The
preprocessor transforms the {\fontsize{10pt}{10pt}\verb!basename.p.py!} file to a standard Python
file {\fontsize{10pt}{10pt}\verb!basename.py!}. The doc strings are now in plain text and well
suited for Pydoc or reading by humans. All these steps are automated
by the {\fontsize{10pt}{10pt}\verb!insertdocstr.py!} script.  Here are the corresponding Unix
commands:
\begin{Verbatim}[fontsize=\fontsize{9pt}{9pt},tabsize=8,baselinestretch=0.85,
fontfamily=tt,xleftmargin=7mm]
# make Epydoc API manual of basename module:
cd docstrings
doconce2format epytext doc1.do.txt
mv doc1.epytext doc1.dst.txt
cd ..
preprocess basename.p.py > basename.py
epydoc basename

# make Sphinx API manual of basename module:
cd doc
doconce2format sphinx doc1.do.txt
mv doc1.rst doc1.dst.txt
cd ..
preprocess basename.p.py > basename.py
cd docstrings/sphinx-rootdir  # sphinx directory for API source
make clean
make html
cd ../..

# make ordinary Python module files with doc strings:
cd docstrings
doconce2format plain doc1.do.txt
mv doc1.txt doc1.dst.txt
cd ..
preprocess basename.p.py > basename.py

# can automate inserting doc strings in all .p.py files:
insertdocstr.py plain .
# (runs through all .do.txt files and filters them to plain format and
# renames to .dst.txt extension, then the script runs through all 
# .p.py files and runs the preprocessor, which includes the .dst.txt
# files)
\end{Verbatim}
\noindent


\section{Warning/Disclaimer}

Doconce can be viewed is a unified interface to a variety of
typesetting formats.  This interface is minimal in the sense that a
lot of typesetting features are not supported, for example, footnotes
and bibliography. For many documents the simple Doconce format is
sufficient, while in other cases you need more sophisticated
formats. Then you can just filter the Doconce text to a more
approprite format and continue working in this format only.  For
example, reStructuredText is a good alternative: it is more tagged
than Doconce and cannot be filtered to plain, untagged text, or wiki,
and the {\LaTeX} output is not at all as clean, but it also has a lot
more typesetting and tagging features than Doconce.
\printindex

\end{document}