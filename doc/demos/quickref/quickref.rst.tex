\documentclass[a4paper]{article}
% generated by Docutils <http://docutils.sourceforge.net/>
\usepackage{fixltx2e} % LaTeX patches, \textsubscript
\usepackage{cmap} % fix search and cut-and-paste in Acrobat
\usepackage{ifthen}
\usepackage[T1]{fontenc}
\usepackage[utf8]{inputenc}
\usepackage{longtable,ltcaption,array}
\setlength{\extrarowheight}{2pt}
\newlength{\DUtablewidth} % internal use in tables
\usepackage{tabularx}

%%% Custom LaTeX preamble
% PDF Standard Fonts
\usepackage{mathptmx} % Times
\usepackage[scaled=.90]{helvet}
\usepackage{courier}

%%% User specified packages and stylesheets

%%% Fallback definitions for Docutils-specific commands

% providelength (provide a length variable and set default, if it is new)
\providecommand*{\DUprovidelength}[2]{
  \ifthenelse{\isundefined{#1}}{\newlength{#1}\setlength{#1}{#2}}{}
}

% docinfo (width of docinfo table)
\DUprovidelength{\DUdocinfowidth}{0.9\textwidth}

% hyperlinks:
\ifthenelse{\isundefined{\hypersetup}}{
  \usepackage[colorlinks=true,linkcolor=blue,urlcolor=blue]{hyperref}
  \urlstyle{same} % normal text font (alternatives: tt, rm, sf)
}{}
\hypersetup{
  pdftitle={Doconce Quick Reference},
  pdfauthor={Hans Petter Langtangen}
}

%%% Title Data
\title{\phantomsection%
  Doconce Quick Reference%
  \label{doconce-quick-reference}}
\author{}
\date{}

%%% Body
\begin{document}
\maketitle

% Docinfo
\begin{center}
\begin{tabularx}{\DUdocinfowidth}{lX}
\textbf{Author}: &
	Hans Petter Langtangen \\
\textbf{Date}: &
	Sep 9, 2012 \\
\end{tabularx}
\end{center}

% Automatically generated reST file from Doconce source
% (http://code.google.com/p/doconce/)

\phantomsection\label{table-of-contents}
\pdfbookmark[1]{Table of Contents}{table-of-contents}
\setcounter{tocdepth}{2}
\renewcommand{\contentsname}{Table of Contents}
\tableofcontents


% Very preliminary

\textbf{WARNING: This quick reference is very incomplete!}


%___________________________________________________________________________

\section*{\phantomsection%
  Supported Formats%
  \addcontentsline{toc}{section}{Supported Formats}%
  \label{supported-formats}%
}

Doconce currently translates files to the following formats:
%
\begin{quote}
%
\begin{itemize}

\item LaTeX (format \texttt{latex} and \texttt{pdflatex})

\item HTML (format \texttt{html})

\item reStructuredText (format \texttt{rst})

\item plain (untagged) ASCII (format \texttt{plain})

\item Sphinx (format \texttt{sphinx})

\item (Pandoc extended) Markdown (format \texttt{pandoc})

\item Googlecode wiki (format \texttt{gwiki})

\item MediaWiki for Wikipedia and Wikibooks (format \texttt{mwiki})

\item Creoloe wiki (format \texttt{cwiki})

\item Epydoc (format \texttt{epydoc})

\item StructuredText (format \texttt{st})

\end{itemize}

\end{quote}

The best supported formats are \texttt{latex}, \texttt{sphinx}, \texttt{html}, and \texttt{plain}.


%___________________________________________________________________________

\section*{\phantomsection%
  Title, Authors, and Date%
  \addcontentsline{toc}{section}{Title, Authors, and Date}%
  \label{title-authors-and-date}%
}

A typical example of giving a title, a set of authors, a date,
and an optional table of contents
reads:
%
\begin{quote}{\ttfamily \raggedright \noindent
TITLE:~On~an~Ultimate~Markup~Language\\
AUTHOR:~H.~P.~Langtangen~at~Center~for~Biomedical~Computing,~Simula~Research~Laboratory~and~Dept.~of~Informatics,~Univ.~of~Oslo\\
AUTHOR:~Kaare~Dump~Email:~dump@cyb.space.com~at~Segfault,~Cyberspace~Inc.\\
AUTHOR:~A.~Dummy~Author\\
DATE:~today\\
TOC:~on
}
\end{quote}

The entire title must appear on a single line.
The author syntax is:
%
\begin{quote}{\ttfamily \raggedright \noindent
name~Email:~somename@adr.net~at~institution1~and~institution2
}
\end{quote}

where the email is optional, the ``at'' keyword is required if one or
more institutions are to be specified, and the ``and'' keyword
separates the institutions. Each author specification must appear
on a single line.
When more than one author belong to the
same institution, make sure that the institution is specified in an identical
way for each author.

The date can be set as any text different from \texttt{today} if not the
current date is wanted, e.g., \texttt{Feb 22, 2016}.

The table of contents is removed by writing \texttt{TOC: off}.


%___________________________________________________________________________

\section*{\phantomsection%
  Section Types%
  \addcontentsline{toc}{section}{Section Types}%
  \label{section-types}%
  \label{quick-sections}%
}

\setlength{\DUtablewidth}{\linewidth}
\begin{longtable*}[c]{|p{0.435\DUtablewidth}|p{0.482\DUtablewidth}|}
\hline
\textbf{%
Section type
} & \textbf{%
Syntax
} \\
\hline
\endfirsthead
\hline
\textbf{%
Section type
} & \textbf{%
Syntax
} \\
\hline
\endhead
\multicolumn{2}{c}{\hfill ... continued on next page} \\
\endfoot
\endlastfoot

chapter
 & 
\texttt{========= Heading ========} (9 \texttt{=})
 \\
\hline

section
 & 
\texttt{======= Heading =======}    (7 \texttt{=})
 \\
\hline

subsection
 & 
\texttt{===== Heading =====}        (5 \texttt{=})
 \\
\hline

subsubsection
 & 
\texttt{=== Heading ===}            (3 \texttt{=})
 \\
\hline

paragraph
 & 
\texttt{\_\_Heading.\_\_}               (2 \texttt{\_})
 \\
\hline

abstract
 & 
\texttt{\_\_Abstract.\_\_} Running text...
 \\
\hline
\end{longtable*}

Note that abstracts are recognized by starting with \texttt{\_\_Abstract.\_\_} at
the beginning of a line and ending with three or more \texttt{=} signs of the
next heading.


%___________________________________________________________________________

\section*{\phantomsection%
  Inline Formatting%
  \addcontentsline{toc}{section}{Inline Formatting}%
  \label{inline-formatting}%
}

Words surrounded by \texttt{*} are emphasized: \texttt{*emphasized words*} becomes
\emph{emphasized words}. Similarly, an underscore surrounds words that
appear in boldface: \texttt{\_boldface\_} become \textbf{boldface}.


%___________________________________________________________________________

\section*{\phantomsection%
  Lists%
  \addcontentsline{toc}{section}{Lists}%
  \label{lists}%
}

There are three types of lists: \emph{bullet lists}, where each item starts
with \texttt{*}, \emph{enumeration lists}, where each item starts with \texttt{o} and gets
consqutive numbers,
and \emph{description} lists, where each item starts with \texttt{-} followed
by a keyword and a colon:
%
\begin{quote}{\ttfamily \raggedright \noindent
Here~is~a~bullet~list:\\
~\\
~*~item1\\
~*~item2\\
~~*~subitem1~of~item2\\
~~*~subitem2~of~item2\\
~*~item3\\
~\\
Note~that~sublists~are~consistently~indented~by~one~or~more~blanks..\\
Here~is~an~enumeration~list:\\
~\\
~o~item1\\
~o~item2\\
~~~may~appear~on\\
~~~multiple~lines\\
~~o~subitem1~of~item2\\
~~o~subitem2~of~item2\\
~o~item3\\
~\\
And~finally~a~description~list:\\
~\\
~-~keyword1:~followed~by\\
~~~some~text\\
~~~over~multiple\\
~~~lines\\
~-~keyword2:\\
~~~followed~by~text~on~the~next~line\\
~-~keyword3:~and~its~description~may~fit~on~one~line
}
\end{quote}

The code above follows.

Here is a bullet list:
%
\begin{quote}
%
\begin{itemize}

\item item1

\item item2

\end{itemize}
%
\begin{quote}
%
\begin{itemize}

\item subitem1 of item2

\item subitem2 of item2

\end{itemize}

\end{quote}
%
\begin{itemize}

\item item3

\end{itemize}

\end{quote}

Note that sublists are indented.
Here is an enumeration list:
\newcounter{listcnt0}
\begin{list}{\arabic{listcnt0}.}
{
\usecounter{listcnt0}
\setlength{\rightmargin}{\leftmargin}
}

\item item1

\item item2
may appear on
multiple lines
\end{list}
%
\begin{quote}
\setcounter{listcnt0}{0}
\begin{list}{\arabic{listcnt0}.}
{
\usecounter{listcnt0}
\setlength{\rightmargin}{\leftmargin}
}

\item subitem1 of item2

\item subitem2 of item2
\end{list}

\end{quote}
\setcounter{listcnt0}{0}
\begin{list}{\arabic{listcnt0}.}
{
\usecounter{listcnt0}
\addtocounter{listcnt0}{2}
\setlength{\rightmargin}{\leftmargin}
}

\item item3
\end{list}

And finally a description list:
%
\begin{quote}
%
\begin{description}
\item[{keyword1:}] \leavevmode 
followed by
some text
over multiple
lines

\item[{keyword2:}] \leavevmode 
followed by text on the next line

\item[{keyword3:}] \leavevmode 
and its description may fit on one line

\end{description}

\end{quote}


%___________________________________________________________________________

\section*{\phantomsection%
  Comments%
  \addcontentsline{toc}{section}{Comments}%
  \label{comments}%
}

Lines starting with \texttt{\#} are treated as comments in the document and
translated to the proper syntax for comments in the output
document. Such comment lines should not appear before LaTeX math
blocks, verbatim code blocks, or lists if the formats \texttt{rst} and
\texttt{sphinx} are desired.

When using the Mako preprocessor one can also place comments in
the Doconce source file that will be removed by Mako before
Doconce starts processing the file. Mako comments are recognized
by lines starting with two hashes \texttt{\#\#} or by blocks of text
inside the comment directives \texttt{\%<doc>} (beginning) and \texttt{<\%doc/>} (end).

Inline comments, in the text, that are meant as messages or notes to readers
(authors in particular)
are often useful and enabled by the syntax:
%
\begin{quote}{\ttfamily \raggedright \noindent
{[}name:~running~text{]}
}
\end{quote}

where \texttt{name} is the name or ID of an author or reader making the comment,
and \texttt{running text} is the comment. There must be a space after the colon.
Running:
%
\begin{quote}{\ttfamily \raggedright \noindent
doconce~format~html~mydoc.do.txt~-{}-skip\_inline\_comments
}
\end{quote}

removes all such inline comments from the output. This feature makes it easy
to turn on and off notes to readers and is frequently used while writing
a document.

All inline comments to readers can also be physically
removed from the Doconce source if desired:
%
\begin{quote}{\ttfamily \raggedright \noindent
doconce~remove\_inline\_comments~mydoc.do.txt
}
\end{quote}

This action is appropriate when all issues with such comments are resolved.


%___________________________________________________________________________

\section*{\phantomsection%
  Verbatim/Computer Code%
  \addcontentsline{toc}{section}{Verbatim/Computer Code}%
  \label{verbatim-computer-code}%
}

Inline verbatim code is typeset within back-ticks, as in:
%
\begin{quote}{\ttfamily \raggedright \noindent
Some~sentence~with~`words~in~verbatim~style`.
}
\end{quote}

resulting in Some sentence with \texttt{words in verbatim style}.

Multi-line blocks of verbatim text, typically computer code, is typeset
in between \texttt{!bc xxx} and \texttt{!ec} directives (which must appear on the
beginning of the line). A specification \texttt{xxx} indicates what verbatim
formatting style that is to be used. Typical values for \texttt{xxx} are
nothing, \texttt{cod} for a code snippet, \texttt{pro} for a complete program,
\texttt{sys} for a terminal session, \texttt{dat} for a data file (or output from a
program),
\texttt{Xpro} or \texttt{Xcod} for a program or code snipped, respectively,
in programming \texttt{X}, where \texttt{X} may be \texttt{py} for Python,
\texttt{cy} for Cython, \texttt{sh} for Bash or other Unix shells,
\texttt{f} for Fortran, \texttt{c} for C, \texttt{cpp} for C++, \texttt{m} for MATLAB,
\texttt{pl} for Perl. For output in \texttt{latex} one can let \texttt{xxx} reflect any
defined verbatim environment in the \texttt{ptex2tex} configuration file
(\texttt{.ptex2tex.cfg}). For \texttt{sphinx} output one can insert a comment:
%
\begin{quote}{\ttfamily \raggedright \noindent
\#~sphinx~code-blocks:~pycod=python~cod=fortran~cppcod=c++~sys=console
}
\end{quote}

that maps environments (\texttt{xxx}) onto valid language types for
Pygments (which is what \texttt{sphinx} applies to typeset computer code).

The \texttt{xxx} specifier has only effect for \texttt{latex} and
\texttt{sphinx} output. All other formats use a fixed monospace font for all
kinds of verbatim output.

% When showing copy from file in !bc envir, intent a character - otherwise

% ptex2tex is confused and starts copying...

Computer code can also be copied from a file:
%
\begin{quote}{\ttfamily \raggedright \noindent
@@@CODE~doconce\_program.sh\\
@@@CODE~doconce\_program.sh~~fromto:~doconce~clean@\textasciicircum{}doconce~split\_rst\\
@@@CODE~doconce\_program.sh~~from-to:~doconce~clean@\textasciicircum{}doconce~split\_rst
}
\end{quote}

The \texttt{@@@CODE} identifier must appear at the very beginning of the line.
The first specification copies the complete file \texttt{doconce\_program.sh}.
The second specification copies from the first line matching the \emph{regular
expression} \texttt{doconce clean} up to, but not including the line
matching the \emph{regular expression} \texttt{\textasciicircum{}doconce split\_rst}.
The third specification behaves as the second, but the line matching
the first regular expression is not copied (aimed at copying
text between begin-end comment pair in the file).

The copied line from file are in this example put inside \texttt{!bc shpro}
and \texttt{!ec} directives, if a complete file is copied, while the
directives become \texttt{!bc shcod} and \texttt{!ec} when a code snippet is copied
from file. In general, for a filename extension \texttt{.X}, the environment
becomes \texttt{!bc Xpro} or \texttt{!bc Xcod} for a complete program or snippet,
respectively. The enivorments (\texttt{Xcod} and \texttt{Xpro}) are only active
for \texttt{latex} and \texttt{sphinx} outout.

Important warnings:
%
\begin{quote}
%
\begin{itemize}

\item A code block must come after some plain sentence (at least for successful
output in reStructredText), not directly after a section/paragraph heading,
table, comment, figure, or movie.

\item Verbatim code blocks inside lists can be ugly typeset in some
output formats. A more robust approach is to replace the list by
paragraphs with headings.

\end{itemize}

\end{quote}


%___________________________________________________________________________

\section*{\phantomsection%
  LaTeX Mathematics%
  \addcontentsline{toc}{section}{LaTeX Mathematics}%
  \label{latex-mathematics}%
}

Doconce supports inline mathematics and blocks of mathematics, using
standard LaTeX syntax. The output formats \texttt{sphinx}, \texttt{latex}, and \texttt{pdflatex}
work with this syntax while all other formats will just display the
raw LaTeX code.

Inline expressions are written in the standard
LaTeX way with the mathematics surrounded by dollar signs, as in
Ax=b. To help increase readability in other formats than \texttt{sphinx},
\texttt{latex}, and \texttt{pdflatex}, inline mathematics may have a more human
readable companion expression. The syntax is like:
%
\begin{quote}{\ttfamily \raggedright \noindent
\$\textbackslash{}sin(\textbackslash{}norm\{\textbackslash{}bf~u\})\$|\$sin(||u||)\$
}
\end{quote}

That is, the LaTeX expression appears to the left of a vertical bar (pipe
symbol) and the more readable expression appears to the right. Both
expressions are surrounded by dollar signs. Plain text formats and HTML
will applied the expression to the right.

Blocks of LaTeX mathematics are written within
\texttt{!bt}
and
\texttt{!et} (``begin/end TeX'') directives.
For example:
%
\begin{quote}{\ttfamily \raggedright \noindent
!bt\\
\textbackslash{}begin\{align*\}\\
\textbackslash{}nabla\textbackslash{}cdot~u~\&=~0,\textbackslash{}\textbackslash{}\\
\textbackslash{}nabla\textbackslash{}times~u~\&~0.\\
\textbackslash{}end\{align*\}
}
\end{quote}

will appear as:
%
\begin{quote}{\ttfamily \raggedright \noindent
\textbackslash{}begin\{align*\}\\
\textbackslash{}nabla\textbackslash{}cdot~u~\&=~0,\textbackslash{}\textbackslash{}\\
\textbackslash{}nabla\textbackslash{}times~u~\&~0.\\
\textbackslash{}end\{align*\}
}
\end{quote}

One can use \texttt{\#if FORMAT in ("latex", "pdflatex", "sphinx", "mwiki")} to let
the preprocessor choose a block of mathematics in LaTeX format
or (\texttt{\#else}) a modified form more suited for plain text and wiki
formats without support for mathematics.

Any LaTeX syntax is accepted, but if output in the \texttt{sphinx}, \texttt{pandoc},
or \texttt{html} formats
is important, one must know that these formats does not support many
LaTeX constructs. For output both in \texttt{latex} and the mentioned formats
the following rules are recommended:
%
\begin{quote}
%
\begin{itemize}

\item Use only the equation environments \texttt{\textbackslash{}{[}}, \texttt{\textbackslash{}{]}},
\texttt{equation}, \texttt{equation*}, \texttt{align}, and \texttt{align*}.

\item Labels in multiple equation environments such as \texttt{align} are
not (yet) handled by \texttt{sphinx} and \texttt{pandoc}, so avoid inserting
labels and referring  to equation labels in \texttt{align} environments.
Actually, \texttt{align*} is the preferred environment for multiple equations.

\item LaTeX supports lots of fancy formatting, for example, multiple
plots in the same figure, footnotes, margin notes, etc.
Allowing other output formats, such as \texttt{sphinx}, makes it necessary
to only utilze very standard LaTeX and avoid, for instance, more than
one plot per figure. However, one can use preprocessor if-tests on
the format (typically \texttt{\# \#if FORMAT in ("latex", "pdflatex")}) to
include special code for \texttt{latex} and \texttt{pdflatex} output and more
straightforward typesetting for other formats. In this way, one can
also allow advanced LaTeX features and fine tuning of resulting
PDF document.

\end{itemize}

\end{quote}

\emph{LaTeX Newcommands.} Text missing...


%___________________________________________________________________________

\section*{\phantomsection%
  Figures and Movies%
  \addcontentsline{toc}{section}{Figures and Movies}%
  \label{figures-and-movies}%
}

Figures and movies have almost equal syntax:
%
\begin{quote}{\ttfamily \raggedright \noindent
FIGURE:~{[}relative/path/to/figurefile,~width=500{]}~Here~goes~the~caption~which~must~be~on~a~single~line.~label\{some:fig:label\}\\
~\\
MOVIE:~{[}relative/path/to/moviefile,~width=500{]}~Here~goes~the~caption~which~must~be~on~a~single~line.~label\{some:fig:label\}
}
\end{quote}

Note the mandatory comma after the figure/movie file.

The figure file can be listed without extension. Doconce will then find
the version of the file with the most appropriate extension for the chosen
output format. If not suitable version is found, Doconce will convert
another format to the desired one.

Movie files can either be a video or a wildcard expression for a
series of frames. In the latter case, a simple device in an HTML page
will display the individual frame files as a movie.

Combining several image files into one can be done by the
\texttt{convert} and \texttt{montage} programs from the ImageMagick suite:
%
\begin{quote}{\ttfamily \raggedright \noindent
montage~file1.png~file2.png~...~file4.png~-geometry~+2+2~~result.png\\
montage~file1.png~file2.png~-tile~x1~result.png\\
montage~file1.png~file2.png~-tile~1x~result.png\\
~\\
convert~-background~white~file1.png~file2.png~+append~tmp.png
}
\end{quote}

Use \texttt{+append} for stacking left to right, \texttt{-append} for top to bottom.
The positioning of the figures can be controlled by \texttt{-gravity}.


%___________________________________________________________________________

\section*{\phantomsection%
  Tables%
  \addcontentsline{toc}{section}{Tables}%
  \label{tables}%
}

The table in the section \hyperref[section-types]{Section Types} was written with this
syntax:
%
\begin{quote}{\ttfamily \raggedright \noindent
|-{}-{}-{}-{}-{}-{}-{}-{}-{}-{}-{}-{}-{}-{}-{}-c-{}-{}-{}-{}-{}-{}-{}-|-{}-{}-{}-{}-{}-{}-{}-{}-{}-{}-{}-{}-{}-{}-{}-{}-{}-c-{}-{}-{}-{}-{}-{}-{}-{}-{}-{}-{}-{}-{}-{}-{}-{}-{}-{}-{}-|\\
|~~~~~~Section~type~~~~~~~|~~~~~~~~Syntax~~~~~~~~~~~~~~~~~~~~~~~~~|\\
|-{}-{}-{}-{}-{}-{}-{}-{}-{}-{}-{}-{}-{}-{}-{}-l-{}-{}-{}-{}-{}-{}-{}-|-{}-{}-{}-{}-{}-{}-{}-{}-{}-{}-{}-{}-{}-{}-{}-{}-{}-l-{}-{}-{}-{}-{}-{}-{}-{}-{}-{}-{}-{}-{}-{}-{}-{}-{}-{}-{}-|\\
|~chapter~~~~~~~~~~~~~~~~~|~`=========~Heading~========`~(9~`=`)~~|\\
|~section~~~~~~~~~~~~~~~~~|~`=======~Heading~=======`~~~~(7~`=`)~~|\\
|~subsection~~~~~~~~~~~~~~|~`=====~Heading~=====`~~~~~~~~(5~`=`)~~|\\
|~subsubsection~~~~~~~~~~~|~`===~Heading~===`~~~~~~~~~~~~(3~`=`)~~|\\
|~paragraph~~~~~~~~~~~~~~~|~`\_\_Heading.\_\_`~~~~~~~~~~~~~~~(2~`\_`)~~|\\
|-{}-{}-{}-{}-{}-{}-{}-{}-{}-{}-{}-{}-{}-{}-{}-{}-{}-{}-{}-{}-{}-{}-{}-{}-{}-{}-{}-{}-{}-{}-{}-{}-{}-{}-{}-{}-{}-{}-{}-{}-{}-{}-{}-{}-{}-{}-{}-{}-{}-{}-{}-{}-{}-{}-{}-{}-{}-{}-{}-{}-{}-{}-{}-{}-|
}
\end{quote}

Note that
%
\begin{quote}
%
\begin{itemize}

\item Each line begins and ends with a vertical bar (pipe symbol).

\item Column data are separated by a vertical bar (pipe symbol).

\item There may be horizontal rules, i.e., lines with dashes for
indicating the heading and the end of the table, and these may
contain characters 'c', 'l', or 'r' for how to align headings or
columns. The first horizontal rule may indicate how to align
headings (center, left, right), and the horizontal rule after the
heading line may indicate how to align the data in the columns
(center, left, right).

\item If the horizontal rules are without alignment information there should
be no vertical bar (pipe symbol) between the columns. Otherwise, such
a bar indicates a vertical bar between columns in LaTeX.

\item Many output formats are so primitive that heading and column alignment
have no effect.

\end{itemize}

\end{quote}


%___________________________________________________________________________

\section*{\phantomsection%
  Labels, References, Citations, and Index%
  \addcontentsline{toc}{section}{Labels, References, Citations, and Index}%
  \label{labels-references-citations-and-index}%
}

The notion of labels, references, citations, and an index is adopted
from LaTeX with a very similar syntax. As in LaTeX, a label can be
inserted anywhere, using the syntax:
%
\begin{quote}{\ttfamily \raggedright \noindent
label\{name\}
}
\end{quote}

with no backslash
preceding the label keyword! It is common practice to choose \texttt{name}
as some hierarchical name, say \texttt{a:b:c}, where \texttt{a} and \texttt{b} indicate
some abbreviations for a section and/or subsection for the topic and
\texttt{c} is some name for the particular unit that has a label.

A reference to the label \texttt{name} is written as:
%
\begin{quote}{\ttfamily \raggedright \noindent
ref\{name\}
}
\end{quote}

again with no backslash before \texttt{ref}.

Single citations are written as:
%
\begin{quote}{\ttfamily \raggedright \noindent
cite\{name\}
}
\end{quote}

where \texttt{name} is a logical name
of the reference (again, LaTeX writers must not insert a backslash).
Bibliography citations often have \texttt{name} on the form
\texttt{Author1\_Author2\_YYYY}, \texttt{Author\_YYYY}, or \texttt{Author1\_etal\_YYYY}, where
\texttt{YYYY} is the year of the publication.
Multiple citations at once is possible by separating the logical names
by comma:
%
\begin{quote}{\ttfamily \raggedright \noindent
cite\{name1,name2,name3\}
}
\end{quote}

The bibliography is specified by a line \texttt{BIBFILE: name\_bib.bib,
name\_bib.rst, name\_bib.py}, where \texttt{name} is the logical name of the
document (the doconce file will then normally have the name
\texttt{name.do.txt}), and the various files reflect different formattings of
the bibliography: '.bib' indicates a BibTeX file, '.rst' a reST-style
bibliography, and '.py' a Python list of dictionaries for specifying
the entries in the bibliography. The bibliography (as read from file)
is inserted where the \texttt{BIBFILE} keyword appears.

There is a \emph{generalized referencing} feature in Doconce that allows
a reference with \texttt{ref} to have one formulation if the label is
in the same document and another formulation if the reference is
to an item in an external document. The syntax of a generalized
reference is:
%
\begin{quote}{\ttfamily \raggedright \noindent
ref{[}internal{]}{[}cite{]}{[}external{]}\\
~\\
\#~Example:\\
As~explained~in\\
ref{[}Section~ref\{subsec:ex\}{]}{[}in~cite\{testdoc:12\}{]}{[}a~"section":\\
"testdoc.html\#\_\_\_sec2"~in~the~document\\
"A~Document~for~Testing~Doconce":~"testdoc.html"~cite\{testdoc:12\}{]},\\
Doconce~documents~may~include~movies.
}
\end{quote}

The output from a generalized reference is the text \texttt{internal} if all
\texttt{label`\_` references in `{}`internal} are references to labels in the
present document. Otherwise, if cite is non-empty and the format is
\texttt{latex} or \texttt{pdflatex} one assumes that the references in \texttt{internal}
are to external documents declared by a comment line \texttt{\#
Externaldocuments: testdoc, mydoc} (usually after the title, authors,
and date). In this case the output text is \texttt{internal cite} and the
LaTeX package \texttt{xr} is used to handle the labels in the external
documents.  If none of the two situations above applies, the
\texttt{external} text will be the output.

Doconce supports creating an index of keywords. A certain keyword
is registered for the index by a syntax like (no
backslash!):
%
\begin{quote}{\ttfamily \raggedright \noindent
index\{name\}
}
\end{quote}

It is recommended to place any index of this type outside
running text, i.e., after (sub)section titles and in the space between
paragraphs. Index specifications placed right before paragraphs also
gives the doconce source code an indication of the content in the
forthcoming text. The index is only produced for the \texttt{latex}, \texttt{rst}, and
\texttt{sphinx} formats.


%___________________________________________________________________________

\section*{\phantomsection%
  Capabilities of the ``doconce'' Program%
  \addcontentsline{toc}{section}{Capabilities of the ``doconce'' Program}%
  \label{capabilities-of-the-doconce-program}%
}

The \texttt{doconce} program can be used for a number of purposes besides
transforming a \texttt{.do.txt} file to some format. Here is the
list of capabilities:
%
\begin{quote}{\ttfamily \raggedright \noindent
Usage:~doconce~command~{[}optional~arguments{]}\\
commands:~format~insertdocstr~old2new\_format~gwiki\_figsubst~remove\_inline\_comments~latin2html~sphinx\_dir~subst~replace~replace\_from\_file~clean~help~latex\_header~latex\_footer~guess\_encoding~change\_encoding~bbl2rst~split\_rst~list\_labels~teamod~sphinxfix\_localURLs~make\_figure\_code\_links~grab~remove~remove\_exercise\_answers~spellcheck~ptex2tex~expand\_commands~latex\_exercise\_toc\\
~\\
doconce~format~html|latex|pdflatex|rst|sphinx|plain|gwiki|mwiki|cwiki|pandoc|st|epytext~file.do.txt\\
~\\
doconce~subst~{[}-s~-m~-x~-{}-restore{]}~regex-pattern~regex-replacement~file1~file2~...\\
(-s~is~the~re.DOTALL~modifier,~-m~is~the~re.MULTILINE~modifier,\\
~-x~is~the~re.VERBOSE~modifier,~-{}-restore~copies~backup~files~back~again)\\
~\\
doconce~replace~from-text~to-text~file1~file2~...\\
(exact~text~substutition)\\
~\\
doconce~replace\_from\_file~file-with-from-to~file1~file2~...\\
(exact~text~substitution,~but~a~set~of~from-to~relations)\\
~\\
doconce~gwiki\_figsubst~file.gwiki~URL-of-fig-dir\\
~\\
doconce~remove\_inline\_comments~file.do.txt\\
~\\
doconce~sphinx\_dir~author='Me~and~you'~title='Quick~title'~\textbackslash{}\\
~~~~version=0.1~dirname=sphinx-rootdir~theme=default~\textbackslash{}\\
~~~~file1~file2~file3\\
(requires~sphinx~version~>=~1.1)\\
~\\
doconce~latin2html~file.html\\
~\\
doconce~insertdocstr~rootdir\\
~\\
doconce~clean\\
(remove~all~files~that~the~doconce~format~can~regenerate)\\
~\\
doconce~latex\_header\\
doconce~latex\_footer\\
~\\
doconce~change\_encoding~utf-8~latin1~filename\\
doconce~guess\_encoding~filename\\
~\\
doconce~bbl2rst~file.bbl\\
doconce~split\_rst~complete\_file.rst\\
doconce~sphinxfix\_local\_URLs~file.rst\\
~\\
doconce~grab~~~-{}-from{[}-{]}~from-text~{[}-{}-to{[}-{]}~to-text{]}~somefile\\
doconce~remove~-{}-from{[}-{]}~from-text~{[}-{}-to{[}-{]}~to-text{]}~somefile\\
doconce~remove\_exercise\_answers~file\_in\_some\_format\\
doconce~spellcheck~{[}-d~.mydict.txt{]}~*.do.txt\\
doconce~ptex2tex~mydoc~-DMINTED~pycod=minted~sys=Verbatim~\textbackslash{}\\
~~~~~~~~dat=\textbackslash{}begin\{quote\}\textbackslash{}begin\{verbatim\};\textbackslash{}end\{verbatim\}\textbackslash{}end\{quote\}\\
~\\
doconce~expand\_commands~file1~file2~...\\
doconce~latex\_exercise\_toc~doconcefile.do.txt\\
doconce~list\_labels~doconcefile.do.txt~|~latexfile.tex\\
doconce~teamod~name\\
doconce~assemble~name~master.do.txt
}
\end{quote}


%___________________________________________________________________________

\section*{\phantomsection%
  Exercises%
  \addcontentsline{toc}{section}{Exercises}%
  \label{exercises}%
}

Doconce supports \emph{Exercise}, \emph{Problem}, and \emph{Project}. These are typeset
as ordinary sections and referred to by their section labels.
An exercise, problem, or project sections contains certain \emph{elements}:
%
\begin{quote}
%
\begin{itemize}

\item a headline at the level of a subsection or subsubsection,
containing one of the words ``Exercise:'', ``Problem:'', or
``Project:'', followed by a title (required)

\item a label (optional)

\item a solution file (optional)

\item name of file with a student solution (optional)

\item main exercise text (required)

\item a short answer (optional)

\item a full solution (optional)

\item one or more hints (optional)

\item one or more subexercises (subproblems, subprojects), which can also
contain a text, a short answer, a full solution, name student file
to be handed in, and one or more hints (optional)

\end{itemize}

\end{quote}

A typical sketch of a a problem without subexercises goes as follows:
%
\begin{quote}{\ttfamily \raggedright \noindent
=====~Problem:~Derive~the~Formula~for~the~Area~of~an~Ellipse~=====\\
label\{problem:ellipsearea1\}\\
file=ellipse\_area.pdf\\
solution=ellipse\_area1\_sol.pdf\\
~\\
Derive~an~expression~for~the~area~of~an~ellipse~by~integrating\\
the~area~under~a~curve~that~defines~half~of~the~allipse.\\
Show~each~step~in~the~mathematical~derivation.\\
~\\
!bhint\\
Wikipedia~has~the~formula~for~the~curve.\\
!ehint\\
~\\
!bhint\\
"Wolframalpha":~"http://wolframalpha.com"~can~perhaps\\
compute~the~integral.\\
!ehint
}
\end{quote}

An exercise with subproblems, answers and full solutions has this
setup-up:
%
\begin{quote}{\ttfamily \raggedright \noindent
=====~Exercise:~Determine~the~Distance~to~the~Moon~=====\\
label\{exer:moondist\}\\
~\\
Intro~to~this~exercise.~Questions~are~in~subexercises~below.\\
~\\
!bsubex\\
Subexercises~are~numbered~a),~b),~etc.\\
~\\
file=subexer\_a.pdf\\
~\\
!bans\\
Short~answer~to~subexercise~a).\\
!eans\\
~\\
!bhint\\
First~hint~to~subexercise~a).\\
!ehint\\
~\\
!bhint\\
Second~hint~to~subexercise~a).\\
!ehint\\
!esubex\\
~\\
!bsubex\\
Here~goes~the~text~for~subexercise~b).\\
~\\
file=subexer\_b.pdf\\
~\\
!bhint\\
A~hint~for~this~subexercise.\\
!ehint\\
!esubex
}
\end{quote}

By default, answers, solutions, and hints are typeset as paragraphs.


%___________________________________________________________________________

\section*{\phantomsection%
  Labels, Index, and Citations%
  \addcontentsline{toc}{section}{Labels, Index, and Citations}%
  \label{labels-index-and-citations}%
}


%___________________________________________________________________________

\section*{\phantomsection%
  Preprocessing%
  \addcontentsline{toc}{section}{Preprocessing}%
  \label{preprocessing}%
}

Doconce documents may utilize a preprocessor, either \texttt{preprocess} and/or
\texttt{mako}. The former is a C-style preprocessor that allows if-tests
and including other files (but not macros with arguments).
The \texttt{mako} preprocessor is much more advanced - it is actually a full
programming language, very similar to Python.

The command \texttt{doconce format} first runs \texttt{preprocess} and then \texttt{mako}.
Here is a typical example on utilizing \texttt{preprocess} to include another
document, ``comment out'' a large portion of text, and to write format-specific
constructions:
%
\begin{quote}{\ttfamily \raggedright \noindent
\#~\#include~"myotherdoc.do.txt"\\
~\\
\#~\#if~FORMAT~in~("latex",~"pdflatex")\\
\textbackslash{}begin\{table\}\\
\textbackslash{}caption\{Some~words...~label\{mytab\}\}\\
\textbackslash{}begin\{tabular\}\{lrr\}\\
\textbackslash{}hline\textbackslash{}noalign\{\textbackslash{}smallskip\}\\
\textbackslash{}multicolumn\{1\}\{c\}\{time\}~\&~\textbackslash{}multicolumn\{1\}\{c\}\{velocity\}~\&~\textbackslash{}multicolumn\{1\}\{c\}\{acceleration\}~\textbackslash{}\textbackslash{}\\
\textbackslash{}hline\\
0.0~~~~~~~~~~\&~1.4186~~~~~~~\&~-5.01~~~~~~~~\textbackslash{}\textbackslash{}\\
2.0~~~~~~~~~~\&~1.376512~~~~~\&~11.919~~~~~~~\textbackslash{}\textbackslash{}\\
4.0~~~~~~~~~~\&~1.1E+1~~~~~~~\&~14.717624~~~~\textbackslash{}\textbackslash{}\\
\textbackslash{}hline\\
\textbackslash{}end\{tabular\}\\
\textbackslash{}end\{table\}\\
\#~\#else\\
~~|-{}-{}-{}-{}-{}-{}-{}-{}-{}-{}-{}-{}-{}-{}-{}-{}-{}-{}-{}-{}-{}-{}-{}-{}-{}-{}-{}-{}-{}-{}-{}-|\\
~~|time~~|~velocity~|~acceleration~|\\
~~|-{}-l-{}-{}-{}-{}-{}-{}-{}-r-{}-{}-{}-{}-{}-{}-{}-{}-{}-{}-r-{}-{}-{}-{}-{}-{}-{}-|\\
~~|~0.0~~|~1.4186~~~|~-5.01~~~~~~~~|\\
~~|~2.0~~|~1.376512~|~11.919~~~~~~~|\\
~~|~4.0~~|~1.1E+1~~~|~14.717624~~~~|\\
~~|-{}-{}-{}-{}-{}-{}-{}-{}-{}-{}-{}-{}-{}-{}-{}-{}-{}-{}-{}-{}-{}-{}-{}-{}-{}-{}-{}-{}-{}-{}-{}-|\\
\#~\#endif\\
~\\
\#~\#ifdef~EXTRA\_MATERIAL\\
....large~portions~of~text...\\
\#~\#endif
}
\end{quote}

With the \texttt{mako} preprocessor the if-else tests have slightly different syntax.
An \href{http://hplgit.github.com/bioinf-py/}{example document} contains
some illustrations on how to utilize \texttt{mako} (clone the GitHub project and
examine the Doconce source and the \texttt{doc/src/make.sh} script).


%___________________________________________________________________________

\section*{\phantomsection%
  Resources%
  \addcontentsline{toc}{section}{Resources}%
  \label{resources}%
}
%
\begin{quote}
%
\begin{itemize}

\item Excellent ``Sphinx Tutorial'' by C. Reller: ``\url{http://people.ee.ethz.ch/~creller/web/tricks/reST.html}''

\end{itemize}

\end{quote}

\end{document}
