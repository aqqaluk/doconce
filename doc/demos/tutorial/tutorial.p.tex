%%
%% Automatically generated ptex2tex (extended LaTeX) file
%% from Doconce source
%% http://code.google.com/p/doconce/
%%

% #ifdef BOOK
\documentclass{book}
% #else
\documentclass{article}
% #endif

\usepackage{relsize,epsfig,makeidx,amsmath,amsfonts}
\usepackage[latin1]{inputenc}
\usepackage{ptex2tex}
% #ifdef MOVIE15
\usepackage{movie15}
% #endif
% #ifdef MINTED
\usepackage{minted}  % requires latex -shell-escape (for Minted_* ptex2tex envirs)
% #endif

% #ifdef HELVETICA
% Set helvetica as the default font family:
\RequirePackage{helvet}
\renewcommand\familydefault{phv}
% #endif

\usepackage[%
colorlinks=true,
linkcolor=blue,
citecolor=black,
filecolor=blue,
urlcolor=blue]{hyperref}
%\hyperbaseurl{}   % hyperlinks are relative to this root

\newcommand{\inlinecomment}[2]{  ({\bf #1}: \emph{#2})  }
%\newcommand{\inlinecomment}[2]{}  % turn off inline comments

\makeindex

\begin{document}



% #ifndef LATEX_HEADING
% #define LATEX_HEADING
% #endif

% ----------------- Title -------------------------
% #if LATEX_HEADING == "traditional"

\title{Doconce: Document Once, Include Anywhere}

% #elif LATEX_HEADING == "titlepage"

\thispagestyle{empty}
\hbox{\ \ }
\vfill
\begin{center}
{\huge{\bfseries{Doconce: Document Once, Include Anywhere}}}

% #else

\begin{center}
{\LARGE\bf Doconce: Document Once, Include \\ [1.5mm] Anywhere}
\end{center}

% #endif



% ----------------- Author(s) -------------------------
% #if LATEX_HEADING == "traditional"
\author{Hans Petter Langtangen\footnote{Simula Research Laboratory and University of Oslo.}}

% #elif LATEX_HEADING == "titlepage"
\vspace{1.3cm}

{\Large\textsf{Hans Petter Langtangen${}^{1, 2}$}}\\ [3mm]

\ \\ [2mm]

{\large\textsf{${}^1$Simula Research Laboratory} \\ [1.5mm]}
{\large\textsf{${}^2$University of Oslo} \\ [1.5mm]}

% #else

\begin{center}
{\bf Hans Petter Langtangen${}^{1, 2}$} \\ [0mm]
\end{center}

\begin{center}
% List of all institutions:
\centerline{{\small ${}^1$Simula Research Laboratory}}
\centerline{{\small ${}^2$University of Oslo}}
\end{center}
% #endif
% ----------------- End of author(s) -------------------------



% ----------------- Date -------------------------

% #if LATEX_HEADING == "traditional"

\date{Jun 24, 2012}
\maketitle

% #elif LATEX_HEADING == "titlepage"

\ \\ [10mm]
{\large\textsf{Jun 24, 2012}}

\end{center}
\vfill
\clearpage

% #else

\begin{center}
Jun 24, 2012
\end{center}

% #endif

\begin{itemize}
 \item When writing a note, report, manual, etc., do you find it difficult
   to choose the typesetting format? That is, to choose between plain
   (email-like) text, wiki, Word/OpenOffice, {\LaTeX}, HTML,
   reStructuredText, Sphinx, XML, etc.  Would it be convenient to
   start with some very simple text-like format that easily converts
   to the formats listed above, and then at some later stage
   eventually go with a particular format?

 \item Do you need to write documents in varying formats but find it
   difficult to remember all the typesetting details of various
   formats like \href{{http://refcards.com/docs/silvermanj/amslatex/LaTeXRefCard.v2.0.pdf}}{LaTeX}, \href{{http://www.htmlcodetutorial.com/}}{HTML}, \href{{http://docutils.sourceforge.net/docs/ref/rst/restructuredtext.html}}{reStructuredText}, \href{{http://sphinx.pocoo.org/contents.html}}{Sphinx}, and \href{{http://code.google.com/p/support/wiki/WikiSyntax}}{wiki}? Would it be convenient
   to generate the typesetting details of a particular format from a
   very simple text-like format with minimal tagging?

 \item Do you have the same information scattered around in different
   documents in different typesetting formats? Would it be a good idea
   to write things once, in one format, stored in one place, and
   include it anywhere?
\end{itemize}

\noindent
If any of these questions are of interest, you should keep on reading.


\section{The Doconce Concept}

Doconce is two things:

\begin{enumerate}
 \item Doconce is a very simple and minimally tagged markup language that
    looks like ordinary ASCII text (much like what you would use in an
    email), but the text can be transformed to numerous other formats,
    including HTML, Pandoc, Google wiki, {\LaTeX}, PDF, reStructuredText
    (reST), Sphinx, Epytext, and also plain text (where non-obvious
    formatting/tags are removed for clear reading in, e.g.,
    emails). From reST you can (via \code{rst2*} programs) go to XML, HTML,
    {\LaTeX}, PDF, OpenOffice, and from the latter (via \code{unoconv}) to
    RTF, numerous MS Word formats (including MS Office Open XML),
    DocBook, PDF, MediaWiki, XHTML. From Pandoc one can generate
    Markdown, reST, {\LaTeX}, HTML, PDF, DocBook XML, OpenOffice, GNU
    Texinfo, MediaWiki, RTF, Groff, and other formats.

 \item Doconce is a working strategy for never duplicating information.
    Text is written in a single place and then transformed to
    a number of different destinations of diverse type (software
    source code, manuals, tutorials, books, wikis, memos, emails, etc.).
    The Doconce markup language support this working strategy.
    The slogan is: "Write once, include anywhere".
\end{enumerate}

\noindent
Here are some Doconce features:

\begin{itemize}
  \item Doconce markup does include tags, so the format is more tagged than
    Markdown and Pandoc, but less than reST, and very much less than
    {\LaTeX} and HTML.

  \item Doconce can be converted to plain \emph{untagged} text,
    often desirable for computer programs and email.

  \item Doconce has good support for copying in parts of computer code
    directly from the source code files via regular expressions
    for the start and end lines.

  \item Doconce has full support for {\LaTeX} math and integrates well
    with big {\LaTeX} projects (books).

  \item Doconce is almost self-explanatory and is a handy starting point
    for generating documents in more complicated markup languages, such
    as Google wiki, {\LaTeX}, and Sphinx. A primary application of Doconce
    is just to make the initial versions of a Sphinx or wiki document.

  \item Contrary to the similar (and superior) Pandoc translator, Doconce
    supports Sphinx, Google wiki, Creole wiki (for bitbucket.org),
    lots of computer code environments in {\LaTeX}, and a special exercise
    syntax. Doconce also also runs preprocessors (including Mako)
    such that the author can mix ordinary text with programming
    construction for generating parts of the text.
\end{itemize}

\noindent
Doconce was particularly written for the following sample applications:

\begin{itemize}
  \item Large books written in {\LaTeX}, but where many pieces (computer demos,
    projects, examples) can be written in Doconce to appear in other
    contexts in other formats, including plain HTML, Sphinx, wiki, or MS Word.

  \item Software documentation, primarily Python doc strings, which one wants
    to appear as plain untagged text for viewing in Pydoc, as reStructuredText
    for use with Sphinx, as wiki text when publishing the software at
    web sites, and as {\LaTeX} integrated in, e.g., a thesis.

  \item Quick memos, which start as plain text in email, then some small
    amount of Doconce tagging is added, before the memos can appear as
    Sphinx web pages, MS Word documents, or in wikis.
\end{itemize}

\noindent
History: Doconce was developed in 2006 at a time when most popular
markup languages used quite some tagging.  Later, almost untagged
markup languages like Markdown and Pandoc became popular. Doconce is
not a replacement of Pandoc, which is a considerably more
sophisticated project. Moreover, Doconce was developed mainly to
fulfill the needs for a flexible source code base for books with much
mathematics and computer code.

Disclaimer: Doconce is a simple tool, largely based on interpreting
and handling text through regular expressions. The possibility for
tweaking the layout is obviously limited since the text can go to
all sorts of sophisticated markup languages. Moreover, because of
limitations of regular expressions, some formatting of Doconce syntax
may face problems when transformed to HTML, {\LaTeX}, Sphinx, and similar
formats.


\section{What Does Doconce Look Like?}

Doconce text looks like ordinary text, but there are some almost invisible
text constructions that allow you to control the formating. Here are
som examples.

\begin{itemize}
  \item Bullet lists arise from lines starting with an asterisk.

  \item \emph{Emphasized words} are surrounded by asterisks.

  \item \textbf{Words in boldface} are surrounded by underscores.

  \item Words from computer code are enclosed in back quotes and
    then typeset \code{verbatim (in a monospace font)}.

  \item Section headings are recognied by equality (\code{=}) signs before
    and after the title, and the number of \code{=} signs indicates the
    level of the section: 7 for main section, 5 for subsection, and
    3 for subsubsection.

  \item Paragraph headings are recognized by a double underscore
    before and after the heading.

  \item The abstract of a document starts with \emph{Abstract} as paragraph
    heading, and all text up to the next heading makes up the abstract,

  \item Blocks of computer code can easily be included by placing
    \code{!bc} (begin code) and \code{!ec} (end code) commands at separate lines
    before and after the code block.

  \item Blocks of computer code can also be imported from source files.

  \item Blocks of {\LaTeX} mathematics can easily be included by placing
    \code{!bt} (begin TeX) and \code{!et} (end TeX) commands at separate lines
    before and after the math block.

  \item There is support for both {\LaTeX} and text-like inline mathematics.

  \item Figures and movies with captions, simple tables,
    URLs with links, index list, labels and references are supported.

  \item Invisible comments in the output format can be inserted throughout
    the text.

  \item Visible comments can be inserted so that authors and readers can
    comment upon the text (and at any time turn on/off output of such
    comments).

  \item There is special support for advanced exercises features.

  \item With a simple preprocessor, Preprocess or Mako, one can include
    other documents (files) and large portions of text can be defined
    in or out of the text.

  \item With the Mako preprocessor one can even embed Python
    code and use this to steer generation of Doconce text.
\end{itemize}

\noindent
Here is an example of some simple text written in the Doconce format:
\bccq
===== A Subsection with Sample Text =====
label{my:first:sec}

Ordinary text looks like ordinary text, and the tags used for
_boldface_ words, *emphasized* words, and `computer` words look
natural in plain text.  Lists are typeset as you would do in an email,

  * item 1
  * item 2
  * item 3

Lists can also have automatically numbered items instead of bullets,

  o item 1
  o item 2
  o item 3

URLs with a link word are possible, as in "hpl":"http://folk.uio.no/hpl".
If the word is URL, the URL itself becomes the link name,
as in "URL":"tutorial.do.txt".

References to sections may use logical names as labels (e.g., a
"label" command right after the section title), as in the reference to
Section ref{my:first:sec}.

Doconce also allows inline comments such as [hpl: here I will make
some remarks to the text] for allowing authors to make notes. Inline
comments can be removed from the output by a command-line argument
(see Section ref{doconce2formats} for an example).

Tables are also supperted, e.g.,

  |--------------------------------|
  |time  | velocity | acceleration |
  |---r-------r-----------r--------|
  | 0.0  | 1.4186   | -5.01        |
  | 2.0  | 1.376512 | 11.919       |
  | 4.0  | 1.1E+1   | 14.717624    |
  |--------------------------------|

# lines beginning with # are comment lines
\eccq
The Doconce text above results in the following little document:

\subsection{A Subsection with Sample Text}

\label{my:first:sec}

Ordinary text looks like ordinary text, and the tags used for
\textbf{boldface} words, \emph{emphasized} words, and \code{computer} words look
natural in plain text.  Lists are typeset as you would do in an email,

\begin{itemize}
  \item item 1

  \item item 2

  \item item 3
\end{itemize}

\noindent
Lists can also have numbered items instead of bullets, just use an \code{o}
(for ordered) instead of the asterisk:

\begin{enumerate}
 \item item 1

 \item item 2

 \item item 3
\end{enumerate}

\noindent
URLs with a link word are possible, as in \href{{http://folk.uio.no/hpl}}{hpl}.
If the word is URL, the URL itself becomes the link name,
as in \href{{tutorial.do.txt}}{\nolinkurl{tutorial.do.txt}}.

References to sections may use logical names as labels (e.g., a
"label" command right after the section title), as in the reference to
Section~\ref{my:first:sec}.

Doconce also allows inline comments such as \inlinecomment{hpl}{here I will make
some remarks to the text} for allowing authors to make notes. Inline
comments can be removed from the output by a command-line argument
(see Section~\ref{doconce2formats} for an example).

Tables are also supperted, e.g.,


\begin{quote}\begin{tabular}{rrr}
\hline
\multicolumn{1}{c}{time} & \multicolumn{1}{c}{velocity} & \multicolumn{1}{c}{acceleration} \\
\hline
0.0          & 1.4186       & -5.01        \\
2.0          & 1.376512     & 11.919       \\
4.0          & 1.1E+1       & 14.717624    \\
\hline
\end{tabular}\end{quote}

\noindent

\subsection{Mathematics and Computer Code}

Inline mathematics, such as $\nu = \sin(x)$,
allows the formula to be specified both as {\LaTeX} and as plain text.
This results in a professional {\LaTeX} typesetting, but in other formats
the text version normally looks better than raw {\LaTeX} mathematics with
backslashes. An inline formula like $\nu = \sin(x)$ is
typeset as
\bccq
$\nu = \sin(x)$|$v = sin(x)$
\eccq
The pipe symbol acts as a delimiter between {\LaTeX} code and the plain text
version of the formula.

Blocks of mathematics are better typeset with raw {\LaTeX}, inside
\code{!bt} and \code{!et} (begin tex / end tex) instructions.
The result looks like this:
\begin{eqnarray}
{\partial u\over\partial t} &=& \nabla^2 u + f, \label{myeq1}\\
{\partial v\over\partial t} &=& \nabla\cdot(q(u)\nabla v) + g
\end{eqnarray}
Of course, such blocks only looks nice in {\LaTeX}. The raw
{\LaTeX} syntax appears in all other formats (but can still be useful
for those who can read {\LaTeX} syntax).

You can have blocks of computer code, starting and ending with
\code{!bc} and \code{!ec} instructions, respectively. Such blocks look like
\bcod
from math import sin, pi
def myfunc(x):
    return sin(pi*x)

import integrate
I = integrate.trapezoidal(myfunc, 0, pi, 100)
\ecod
A code block must come after some plain sentence (at least for successful
output to \code{sphinx}, \code{rst}, and ASCII-close formats),
not directly after a section/paragraph heading or a table.

It is possible to add a specification of an
environment for typesetting the verbatim code block, e.g., \code{!bc xxx}
where \code{xxx} is an identifier like \code{pycod} for code snippet in Python,
\code{sys} for terminal session, etc. When Doconce is filtered to {\LaTeX},
these identifiers are used as in \code{ptex2tex} and defined in a
configuration file \code{.ptext2tex.cfg}, while when filtering
to Sphinx, one can have a comment line in the Doconce file for
mapping the identifiers to legal language names for Sphinx (which equals
the legal language names for Pygments):
\bccq
# sphinx code-blocks: pycod=python cod=fortran cppcod=c++ sys=console
\eccq
By default, \code{pro} and \code{cod} are \code{python}, \code{sys} is \code{console},
while \code{xpro} and \code{xcod} are computer language specific for \code{x}
in \code{f} (Fortran), \code{c} (C), \code{cpp} (C++), \code{pl} (Perl), \code{m} (Matlab),
\code{sh} (Unix shells), \code{cy} (Cython), and \code{py} (Python).

% (Any sphinx code-block comment, whether inside verbatim code
% blocks or outside, yields a mapping between bc arguments
% and computer languages. In case of muliple definitions, the
% first one is used.)

One can also copy computer code directly from files, either the
complete file or specified parts.  Computer code is then never
duplicated in the documentation (important for the principle of
avoiding copying information!). A complete file is typeset
with \code{!bc pro}, while a part of a file is copied into a \code{!bc cod}
environment. What \code{pro} and \code{cod} mean is then defined through
a \code{.ptex2tex.cfg} file for {\LaTeX} and a \code{sphinx code-blocks}
comment for Sphinx.

Another document can be included by writing \code{#include "mynote.do.txt"}
on a line starting with (another) hash sign.  Doconce documents have
extension \code{do.txt}. The \code{do} part stands for doconce, while the
trailing \code{.txt} denotes a text document so that editors gives you the
right writing enviroment for plain text.

\subsection{Macros (Newcommands), Cross-References, Index, and Bibliography}

\label{newcommands}

Doconce supports a type of macros via a LaTeX-style \emph{newcommand}
construction.  The newcommands defined in a file with name
\code{newcommand_replace.tex} are expanded when Doconce is filtered to
other formats, except for {\LaTeX} (since {\LaTeX} performs the expansion
itself).  Newcommands in files with names \code{newcommands.tex} and
\code{newcommands_keep.tex} are kept unaltered when Doconce text is
filtered to other formats, except for the Sphinx format. Since Sphinx
understands {\LaTeX} math, but not newcommands if the Sphinx output is
HTML, it makes most sense to expand all newcommands.  Normally, a user
will put all newcommands that appear in math blocks surrounded by
\code{!bt} and \code{!et} in \code{newcommands_keep.tex} to keep them unchanged, at
least if they contribute to make the raw {\LaTeX} math text easier to
read in the formats that cannot render {\LaTeX}.  Newcommands used
elsewhere throughout the text will usually be placed in
\code{newcommands_replace.tex} and expanded by Doconce.  The definitions of
newcommands in the \code{newcommands*.tex} files \emph{must} appear on a single
line (multi-line newcommands are too hard to parse with regular
expressions).

Recent versions of Doconce also offer cross referencing, typically one
can define labels below (sub)sections, in figure captions, or in
equations, and then refer to these later. Entries in an index can be
defined and result in an index at the end for the {\LaTeX} and Sphinx
formats. Citations to literature, with an accompanying bibliography in
a file, are also supported. The syntax of labels, references,
citations, and the bibliography closely resembles that of {\LaTeX},
making it easy for Doconce documents to be integrated in {\LaTeX}
projects (manuals, books). For further details on functionality and
syntax we refer to the \code{doc/manual/manual.do.txt} file (see the
\href{{https://doconce.googlecode.com/hg/doc/demos/manual/index.html}}{demo page}
for various formats of this document).


% Example on including another Doconce file (using preprocess):


\section{From Doconce to Other Formats}

\label{doconce2formats}

Transformation of a Doconce document \code{mydoc.do.txt} to various other
formats applies the script \code{doconce format}:
\bsys
Terminal> doconce format format mydoc.do.txt
\esys
or just
\bsys
Terminal> doconce format format mydoc
\esys
The \code{mako} or \code{preprocess} programs are always used to preprocess the
file first, and options to \code{mako} or \code{preprocess} can be added after the
filename. For example,
\bsys
Terminal> doconce format latex mydoc -Dextra_sections -DVAR1=5     # preprocess
Terminal> doconce format latex yourdoc extra_sections=True VAR1=5  # mako
\esys
The variable \code{FORMAT} is always defined as the current format when
running \code{preprocess}. That is, in the last example, \code{FORMAT} is
defined as \code{latex}. Inside the Doconce document one can then perform
format specific actions through tests like \code{#if FORMAT == "latex"}.

Inline comments in the text are removed from the output by
\bsys
Terminal> doconce format latex mydoc --skip_inline_comments
\esys
One can also remove all such comments from the original Doconce
file by running:
\bccq
Terminal> doconce remove_inline_comments mydoc
\eccq
This action is convenient when a Doconce document reaches its final form
and comments by different authors should be removed.

\subsection{HTML}

Making an HTML version of a Doconce file \code{mydoc.do.txt}
is performed by
\bsys
Terminal> doconce format html mydoc
\esys
The resulting file \code{mydoc.html} can be loaded into any web browser for viewing.

\subsection{Pandoc}

Output in Pandoc's extended Markdown format results from
\bsys
Terminal> doconce format pandoc mydoc
\esys
The name of the output file is \code{mydoc.mkd}.
From this format one can go to numerous other formats:
\bsys
Terminal> pandoc -R -t mediawiki -o mydoc.mwk mydoc.mkd
\esys
Pandoc supports \code{latex}, \code{html}, \code{odt} (OpenOffice), \code{docx} (Microsoft
Word), \code{rtf}, \code{texinfo}, to mention some. The \code{-R} option makes
Pandoc pass raw HTML or {\LaTeX} to the output format instead of ignoring it.
See the \href{{http://johnmacfarlane.net/pandoc/README.html}}{Pandoc documentation}
for the many features of the \code{pandoc} program.

Pandoc is useful to go from {\LaTeX} mathematics to, e.g., HTML or MS Word.
There are two ways (experiment to find the best one for your document):
\code{doconce format pandoc} and then translating using \code{pandoc}, or
\code{doconce format latex}, and then going from {\LaTeX} to the desired format
using \code{pandoc}.
Here is an example on the latter strategy:
\bsys
Terminal> doconce format latex mydoc
Terminal> doconce ptex2tex mydoc
Terminal> pandoc -f latex -t docx -o mydoc.docx mydoc.tex
\esys
When we go through \code{pandoc}, only single equations or \code{align*}
environments are well understood.

Quite some \code{doconce replace} and \code{doconce subst} edits might be needed
on the \code{.mkd} or \code{.tex} files to successfully have mathematics that is
well translated to MS Word.  Also when going to reStructuredText using
Pandoc, it can be advantageous to go via {\LaTeX}.

Here is an example where we take a Doconce snippet (without title, author,
and date), maybe with some unnumbered equations, and quickly generate
HTML with mathematics displayed my MathJax:
\bsys
Terminal> doconce format pandoc mydoc
Terminal> pandoc -t html -o mydoc.html -s --mathjax mydoc.mkd
\esys
The \code{-s} option adds a proper header and footer to the \code{mydoc.html} file.
This recipe is a quick way of makeing HTML notes with (some) mathematics.

\subsection{{\LaTeX}}

Making a {\LaTeX} file \code{mydoc.tex} from \code{mydoc.do.txt} is done in two steps:
% Note: putting code blocks inside a list is not successful in many
% formats - the text may be messed up. A better choice is a paragraph
% environment, as used here.

\paragraph{Step 1.}
Filter the doconce text to a pre-LaTeX form \code{mydoc.p.tex} for
     \code{ptex2tex}:
\bsys
Terminal> doconce format latex mydoc
\esys
LaTeX-specific commands ("newcommands") in math formulas and similar
can be placed in files \code{newcommands.tex}, \code{newcommands_keep.tex}, or
\code{newcommands_replace.tex} (see Section~\ref{newcommands}).
If these files are present, they are included in the {\LaTeX} document
so that your commands are defined.

\paragraph{Step 2.}
Run \code{ptex2tex} (if you have it) to make a standard {\LaTeX} file,
\bsys
Terminal> ptex2tex mydoc
\esys
In case you do not have \code{ptex2tex}, you may run a (very) simplified version:
\bsys
Terminal> doconce ptex2tex mydoc
\esys

Note that Doconce generates a \code{.p.tex} file with some preprocessor macros
that can be used to steer certain properties of the {\LaTeX} document.
For example, to turn on the Helvetica font instead of the standard
Computer Modern font, run
\bsys
Terminal> ptex2tex -DHELVETICA mydoc
\esys
The title, authors, and date are by default typeset in a non-standard
way to enable a nicer treatment of multiple authors having
institutions in common. However, the standard {\LaTeX} "maketitle" heading
is also available through
\bsys
Terminal> ptex2tex -DLATEX_HEADING=traditional mydoc
\esys
A separate titlepage can be generate by
\bsys
Terminal> ptex2tex -DLATEX_HEADING=titlepage mydoc
\esys

The \code{ptex2tex} tool makes it possible to easily switch between many
different fancy formattings of computer or verbatim code in {\LaTeX}
documents. After any \code{!bc} command in the Doconce source you can
insert verbatim block styles as defined in your \code{.ptex2tex.cfg}
file, e.g., \code{!bc cod} for a code snippet, where \code{cod} is set to
a certain environment in \code{.ptex2tex.cfg} (e.g., \code{CodeIntended}).
There are about 40 styles to choose from.

Also the \code{doconce ptex2tex} command supports preprocessor directives
for processing the \code{.p.tex} file. The command allows specifications
of code environments as well. Here is an example:
\bsys
Terminal> doconce ptex2tex -DLATEX_HEADING=traditional -DMINTED \
          cycod=\begin{quote}\begin{python};\end{python}\end{quote} \
          fpro=minted fcod=minted sys=verbatim
\esys
Note that semicolon must be used to separate the begin and end
commands, unless only the environment name is given (such as
\code{verbatim} above, which implies \code{\begin{verbatim}} and \code{\end{verbatim}}).
The value \code{minted} can be used for code environments where the
language is specified, as in \code{fpro}, where \code{minted} implies
\code{\begin{python}{fortran}}. There is a similar support for \code{ans}:
\code{cppcod=ans} imples \code{\begin{c++}} and \code{\end{c++}} using the
\code{anslistings} package.

\paragraph{Step 2b (optional).}
Edit the \code{mydoc.tex} file to your needs.
For example, you may want to substitute \code{section} by \code{section*} to
avoid numbering of sections, you may want to insert linebreaks
(and perhaps space) in the title, etc. This can be automatically
edited with the aid of the \code{doconce replace} and \code{doconce subst}
commands. The former works with substituting text directly, while the
latter performs substitutions using regular expressions.
Here are some examples:
\bsys
Terminal> doconce replace 'section{' 'section*{' mydoc.tex
Terminal> doconce subst 'title\{(.+)Using (.+)\}' \
          'title{\g<1> \\\\ [1.5mm] Using \g<2>' mydoc.tex
\esys
A lot of tailored fixes to the {\LaTeX} document can be done by
an appropriate set of text replacements and regular expression
substitutions. You are anyway encourged to make a script for
generating PDF from the {\LaTeX} file.

\paragraph{Step 3.}
Compile \code{mydoc.tex}
and create the PDF file:
\bsys
Terminal> latex mydoc
Terminal> latex mydoc
Terminal> makeindex mydoc   # if index
Terminal> bibitem mydoc     # if bibliography
Terminal> latex mydoc
Terminal> dvipdf mydoc
\esys
If one wishes to use the \code{Minted_Python}, \code{Minted_Cpp}, etc.,
environments in \code{ptex2tex} for typesetting code (specified, e.g., in
the \code{*pro} and \code{*cod} environments in \code{.ptex2tex.cfg} or
\code{$HOME/.ptex2tex.cfg}), the \code{minted} {\LaTeX} package is needed.  This
package is included by running \code{doconce format} with the \code{-DMINTED}
option:
\bsys
Terminal> ptex2tex -DMINTED mydoc
\esys
In this case, \code{latex} must be run with the
\code{-shell-escape} option:
\bsys
Terminal> latex -shell-escape mydoc
Terminal> latex -shell-escape mydoc
Terminal> makeindex mydoc   # if index
Terminal> bibitem mydoc     # if bibliography
Terminal> latex -shell-escape mydoc
Terminal> dvipdf mydoc
\esys

\subsection{PDFLaTeX}

Running \code{pdflatex} instead of \code{latex} follows almost the same steps,
but the start is
\bsys
Terminal> doconce format latex mydoc
\esys
Then \code{ptex2tex} is run as explained above, and finally
\bsys
Terminal> pdflatex -shell-escape mydoc
Terminal> makeindex mydoc   # if index
Terminal> bibitem mydoc     # if bibliography
Terminal> pdflatex -shell-escape mydoc
\esys

\subsection{Plain ASCII Text}

We can go from Doconce "back to" plain untagged text suitable for viewing
in terminal windows, inclusion in email text, or for insertion in
computer source code:
\bsys
Terminal> doconce format plain mydoc.do.txt  # results in mydoc.txt
\esys

\subsection{reStructuredText}

Going from Doconce to reStructuredText gives a lot of possibilities to
go to other formats. First we filter the Doconce text to a
reStructuredText file \code{mydoc.rst}:
\bsys
Terminal> doconce format rst mydoc.do.txt
\esys
We may now produce various other formats:
\bsys
Terminal> rst2html.py  mydoc.rst > mydoc.html # html
Terminal> rst2latex.py mydoc.rst > mydoc.tex  # latex
Terminal> rst2xml.py   mydoc.rst > mydoc.xml  # XML
Terminal> rst2odt.py   mydoc.rst > mydoc.odt  # OpenOffice
\esys

The OpenOffice file \code{mydoc.odt} can be loaded into OpenOffice and
saved in, among other things, the RTF format or the Microsoft Word format.
However, it is more convenient to use the program \code{unovonv}
to convert between the many formats OpenOffice supports \emph{on the command line}.
Run
\bsys
Terminal> unoconv --show
\esys
to see all the formats that are supported.
For example, the following commands take
\code{mydoc.odt} to Microsoft Office Open XML format,
classic MS Word format, and PDF:
\bsys
Terminal> unoconv -f ooxml mydoc.odt
Terminal> unoconv -f doc mydoc.odt
Terminal> unoconv -f pdf mydoc.odt
\esys

\paragraph{Remark about Mathematical Typesetting.}
At the time of this writing, there is no easy way to go from Doconce
and {\LaTeX} mathematics to reST and further to OpenOffice and the
"MS Word world". Mathematics is only fully supported by \code{latex} as
output and to a wide extent also supported by the \code{sphinx} output format.
Some links for going from {\LaTeX} to Word are listed below.

\begin{itemize}
 \item \href{{http://ubuntuforums.org/showthread.php?t=1033441}}{\nolinkurl{http://ubuntuforums.org/showthread.php?t=1033441}}

 \item \href{{http://tug.org/utilities/texconv/textopc.html}}{\nolinkurl{http://tug.org/utilities/texconv/textopc.html}}

 \item \href{{http://nileshbansal.blogspot.com/2007/12/latex-to-openofficeword.html}}{\nolinkurl{http://nileshbansal.blogspot.com/2007/12/latex-to-openofficeword.html}}
\end{itemize}

\noindent

\subsection{Sphinx}

Sphinx documents demand quite some steps in their creation. We have automated
most of the steps through the \code{doconce sphinx_dir} command:
\bsys
Terminal> doconce sphinx_dir author="authors' names" \
          title="some title" version=1.0 dirname=sphinxdir \
          theme=mytheme file1 file2 file3 ...
\esys
The keywords \code{author}, \code{title}, and \code{version} are used in the headings
of the Sphinx document. By default, \code{version} is 1.0 and the script
will try to deduce authors and title from the doconce files \code{file1},
\code{file2}, etc. that together represent the whole document. Note that
none of the individual Doconce files \code{file1}, \code{file2}, etc. should
include the rest as their union makes up the whole document.
The default value of \code{dirname} is \code{sphinx-rootdir}. The \code{theme}
keyword is used to set the theme for design of HTML output from
Sphinx (the default theme is \code{'default'}).

With a single-file document in \code{mydoc.do.txt} one often just runs
\bsys
Terminal> doconce sphinx_dir mydoc
\esys
and then an appropriate Sphinx directory \code{sphinx-rootdir} is made with
relevant files.

The \code{doconce sphinx_dir} command generates a script
\code{automake-sphinx.py} for compiling the Sphinx document into an HTML
document.  One can either run \code{automake-sphinx.py} or perform the
steps in the script manually, possibly with necessary modifications.
You should at least read the script prior to executing it to have
some idea of what is done.

Te \code{doconce sphinx_dir} script copies directories named \code{figs} or \code{figures}
over to the Sphinx directory so that figures are accessible in the
Sphinx compilation.  If figures or movies are located in other
directories, \code{automake-sphinx.py} must be edited accordingly.
Links to local files (not \code{http:} or \code{file:} URLs) must be placed
in the \code{_static} subdirectory of the Sphinx directory. The
utility \code{doconce sphinxfix_localURLs} is run to check for local
links: for each such link, say \code{dir1/dir2/myfile.txt} it replaces
the link by \code{_static/myfile.txt} and copies \code{dir1/dir2/myfile.txt}
to a local \code{_static} directory (in the same directory as the
script is run). The user must copy all \code{_static/*} files to the
\code{_static} subdirectory of the Sphinx directory. Links to local
HTML files (say another Sphinx document) may present a problem if they link
to other files: all necessary files must be correctly copied to
the \code{_static} subdirectory of the Sphinx directory.
It may be wise to place relevant files in a \code{_static} directory
and link to these directly from the Doconce document - then links
to not need to be modified when creating  a Sphinx version of
the document.

Doconce comes with a collection of HTML themes for Sphinx documents.
These are packed out in the Sphinx directory, the \code{conf.py}
configuration file for Sphinx is edited accordingly, and a script
\code{make-themes.sh} can make HTML documents with one or more themes.
For example,
to realize the themes \code{fenics} and \code{pyramid}, one writes
\bsys
Terminal> ./make-themes.sh fenics pyramid
\esys
The resulting directories with HTML documents are \code{_build/html_fenics}
and \code{_build/html_pyramid}, respectively. Without arguments,
\code{make-themes.sh} makes all available themes (!).

If it is not desirable to use the autogenerated scripts explained
above, here is the complete manual procedure of generating a
Sphinx document from a file \code{mydoc.do.txt}.

\paragraph{Step 1.}
Translate Doconce into the Sphinx format:
\bsys
Terminal> doconce format sphinx mydoc
\esys

\paragraph{Step 2.}
Create a Sphinx root directory
either manually or by using the interactive \code{sphinx-quickstart}
program. Here is a scripted version of the steps with the latter:
\bsys
mkdir sphinx-rootdir
sphinx-quickstart <<EOF
sphinx-rootdir
n
_
Name of My Sphinx Document
Author
version
version
.rst
index
n
y
n
n
n
n
y
n
n
y
y
y
EOF
\esys
The autogenerated \code{conf.py} file
may need some edits if you want to specific layout (Sphinx themes)
of HTML pages. The \code{doconce sphinx_dir} generator makes an extended \code{conv.py}
file where, among other things, several useful Sphinx extensions
are included.


\paragraph{Step 3.}
Copy the \code{mydoc.rst} file to the Sphinx root directory:
\bsys
Terminal> cp mydoc.rst sphinx-rootdir
\esys
If you have figures in your document, the relative paths to those will
be invalid when you work with \code{mydoc.rst} in the \code{sphinx-rootdir}
directory. Either edit \code{mydoc.rst} so that figure file paths are correct,
or simply copy your figure directories to \code{sphinx-rootdir}.
Links to local files in \code{mydoc.rst} must be modified to links to
files in the \code{_static} directory, see comment above.

\paragraph{Step 4.}
Edit the generated \code{index.rst} file so that \code{mydoc.rst}
is included, i.e., add \code{mydoc} to the \code{toctree} section so that it becomes
\bccq
.. toctree::
   :maxdepth: 2

   mydoc
\eccq
(The spaces before \code{mydoc} are important!)

\paragraph{Step 5.}
Generate, for instance, an HTML version of the Sphinx source:
\bsys
make clean   # remove old versions
make html
\esys

Sphinx can generate a range of different formats:
standalone HTML, HTML in separate directories with \code{index.html} files,
a large single HTML file, JSON files, various help files (the qthelp, HTML,
and Devhelp projects), epub, {\LaTeX}, PDF (via {\LaTeX}), pure text, man pages,
and Texinfo files.

\paragraph{Step 6.}
View the result:
\bsys
Terminal> firefox _build/html/index.html
\esys

Note that verbatim code blocks can be typeset in a variety of ways
depending the argument that follows \code{!bc}: \code{cod} gives Python
(\code{code-block:: python} in Sphinx syntax) and \code{cppcod} gives C++, but
all such arguments can be customized both for Sphinx and {\LaTeX} output.

\subsection{Wiki Formats}

There are many different wiki formats, but Doconce only supports three:
\href{{http://code.google.com/p/support/wiki/WikiSyntax<Google Code>}}{Googlecode wiki}, , MediaWiki, and Creole Wiki. These formats are called
\code{gwiki}, \code{mwiki}, and \code{cwiki}, respectively.
Transformation from Doconce to these formats is done by
\bsys
Terminal> doconce format gwiki mydoc.do.txt
Terminal> doconce format mwiki mydoc.do.txt
Terminal> doconce format cwiki mydoc.do.txt
\esys

The Googlecode wiki document, \code{mydoc.gwiki}, is most conveniently stored
in a directory which is a clone of the wiki part of the Googlecode project.
This is far easier than copying and pasting the entire text into the
wiki editor in a web browser.

When the Doconce file contains figures, each figure filename must in
the \code{.gwiki} file be replaced by a URL where the figure is
available. There are instructions in the file for doing this. Usually,
one performs this substitution automatically (see next section).

From the MediaWiki format one can go to other formats with aid
of \href{{http://pediapress.com/code/}}{mwlib}. This means that one can
easily use Doconce to write \href{{http://en.wikibooks.org}}{Wikibooks}
and publish these in PDF and MediaWiki format.
At the same time, the book can also be published as a
standard {\LaTeX} book or a Sphinx web document.

\subsection{Tweaking the Doconce Output}

Occasionally, one would like to tweak the output in a certain format
from Doconce. One example is figure filenames when transforming
Doconce to reStructuredText. Since Doconce does not know if the
\code{.rst} file is going to be filtered to {\LaTeX} or HTML, it cannot know
if \code{.eps} or \code{.png} is the most appropriate image filename.
The solution is to use a text substitution command or code with, e.g., sed,
perl, python, or scitools subst, to automatically edit the output file
from Doconce. It is then wise to run Doconce and the editing commands
from a script to automate all steps in going from Doconce to the final
format(s). The \code{make.sh} files in \code{docs/manual} and \code{docs/tutorial}
constitute comprehensive examples on how such scripts can be made.

\subsection{Demos}

The current text is generated from a Doconce format stored in the file
\bccq
docs/tutorial/tutorial.do.txt
\eccq
The file \code{make.sh} in the \code{tutorial} directory of the
Doconce source code contains a demo of how to produce a variety of
formats.  The source of this tutorial, \code{tutorial.do.txt} is the
starting point.  Running \code{make.sh} and studying the various generated
files and comparing them with the original \code{tutorial.do.txt} file,
gives a quick introduction to how Doconce is used in a real case.
\href{{https://doconce.googlecode.com/hg/doc/demos/tutorial/index.html}}{Here}
is a sample of how this tutorial looks in different formats.

There is another demo in the \code{docs/manual} directory which
translates the more comprehensive documentation, \code{manual.do.txt}, to
various formats. The \code{make.sh} script runs a set of translations.

\subsection{Dependencies and Installation}

Doconce itself is pure Python code hosted at \href{{http://code.google.com/p/doconce}}{\nolinkurl{http://code.google.com/p/doconce}}.  Its installation from the
Mercurial (\code{hg}) source follows the standard procedure:
\bsys
# Doconce
hg clone https://doconce.googlecode.com/hg/ doconce
cd doconce
sudo python setup.py install
cd ..
\esys

If you make use of the \href{{http://code.google.com/p/preprocess}}{Preprocess}
preprocessor, this program must be installed:
\bsys
svn checkout http://preprocess.googlecode.com/svn/trunk/ preprocess
cd preprocess
cd doconce
sudo python setup.py install
cd ..
\esys
A much more advanced alternative to Preprocess is
\href{{http://www.makotemplates.org}}{Mako}. Its installation is most
conveniently done by \code{pip},
\bsys
pip install Mako
\esys
This command requires \code{pip} to be installed. On Debian Linux systems,
such as Ubuntu, the installation is simply done by
\bsys
sudo apt-get install python-pip
\esys
Alternatively, one can install from the \code{pip} \href{{http://pypi.python.org/pypi/pip}}{source code}.

To make {\LaTeX}
documents (without going through the reStructuredText format) you
need \href{{http://code.google.com/p/ptex2tex}}{ptex2tex}, which is
installed by
\bsys
svn checkout http://ptex2tex.googlecode.com/svn/trunk/ ptex2tex
cd ptex2tex
sudo python setup.py install
cd latex
sh cp2texmf.sh  # copy stylefiles to ~/texmf directory
cd ../..
\esys
As seen, \code{cp2texmf.sh} copies some special stylefiles that
that \code{ptex2tex} potentially makes use of. Some more standard stylefiles
are also needed. These are installed by
\bsys
sudo apt-get install texlive-latex-extra
\esys
on Debian Linux (including Ubuntu) systems. TeXShop on Mac comes with
the necessary stylefiles (if not, they can be found by googling and installed
manually in the \code{~/texmf/tex/latex/misc} directory).

The \emph{minted} {\LaTeX} style is offered by \code{ptex2tex} and popular among
users. This style requires the package \href{{http://pygments.org}}{Pygments}:
\bsys
hg clone ssh://hg@bitbucket.org/birkenfeld/pygments-main pygments
cd pygments
sudo python setup.py install
\esys
If you use the minted style, you have to enable it by running
\code{ptex2tex -DMINTED} and then \code{latex -shell-escape}, see
the Section~\ref{doconce2formats}.

For \code{rst} output and further transformation to {\LaTeX}, HTML, XML,
OpenOffice, and so on, one needs \href{{http://docutils.sourceforge.net}}{docutils}.
The installation can be done by
\bsys
svn checkout http://docutils.svn.sourceforge.net/svnroot/docutils/trunk/docutils
cd docutils
sudo python setup.py install
cd ..
\esys
To use the OpenOffice suite you will typically on Debian systems install
\bsys
sudo apt-get install unovonv libreoffice libreoffice-dmaths
\esys

There is a possibility to create PDF files from reST documents
using ReportLab instead of {\LaTeX}. The enabling software is
\href{{http://code.google.com/p/rst2pdf}}{rst2pdf}. Either download the tarball
or clone the svn repository, go to the \code{rst2pdf} directory and
run \code{sudo python setup.py install}.


Output to \code{sphinx} requires of course \href{{http://sphinx.pocoo.org}}{Sphinx},
installed by
\bsys
hg clone https://bitbucket.org/birkenfeld/sphinx
cd sphinx
sudo python setup.py install
cd ..
\esys

When the output format is \code{epydoc} one needs that program too, installed
by
\bsys
svn co https://epydoc.svn.sourceforge.net/svnroot/epydoc/trunk/epydoc epydoc
cd epydoc
sudo make install
cd ..
\esys

Finally, translation to \code{pandoc} requires the
\href{{http://johnmacfarlane.net/pandoc/}}{Pandoc} program
(written in Haskell) to be installed.
\bsys
sudo apt-get install pandoc
\esys

\paragraph{Remark.}
Several of the packages above installed from source code
are also available in Debian-based system through the
\code{apt-get install} command. However, we recommend installation directly
from the version control system repository as there might be important
updates and bug fixes. For \code{svn} directories, go to the directory,
run \code{svn update}, and then \code{sudo python setup.py install}. For
Mercurial (\code{hg}) directories, go to the directory, run
\code{hg pull; hg update}, and then \code{sudo python setup.py install}.
Doconce itself is frequently updated so these commands should be
run regularly.

\printindex

\end{document}
