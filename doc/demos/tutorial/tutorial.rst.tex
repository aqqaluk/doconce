\documentclass[a4paper]{article}
% generated by Docutils <http://docutils.sourceforge.net/>
\usepackage{fixltx2e} % LaTeX patches, \textsubscript
\usepackage{cmap} % fix search and cut-and-paste in Acrobat
\usepackage{ifthen}
\usepackage[T1]{fontenc}
\usepackage[utf8]{inputenc}
\usepackage{color}
\usepackage{longtable,ltcaption,array}
\setlength{\extrarowheight}{2pt}
\newlength{\DUtablewidth} % internal use in tables

%%% Custom LaTeX preamble
% PDF Standard Fonts
\usepackage{mathptmx} % Times
\usepackage[scaled=.90]{helvet}
\usepackage{courier}

%%% User specified packages and stylesheets

%%% Fallback definitions for Docutils-specific commands

% admonition (specially marked topic)
\providecommand{\DUadmonition}[2][class-arg]{%
  % try \DUadmonition#1{#2}:
  \ifcsname DUadmonition#1\endcsname%
    \csname DUadmonition#1\endcsname{#2}%
  \else
    \begin{center}
      \fbox{\parbox{0.9\textwidth}{#2}}
    \end{center}
  \fi
}

% fieldlist environment
\ifthenelse{\isundefined{\DUfieldlist}}{
  \newenvironment{DUfieldlist}%
    {\quote\description}
    {\enddescription\endquote}
}{}

% title for topics, admonitions and sidebar
\providecommand*{\DUtitle}[2][class-arg]{%
  % call \DUtitle#1{#2} if it exists:
  \ifcsname DUtitle#1\endcsname%
    \csname DUtitle#1\endcsname{#2}%
  \else
    \smallskip\noindent\textbf{#2}\smallskip%
  \fi
}

% titlereference role
\providecommand*{\DUroletitlereference}[1]{\textsl{#1}}

% hyperlinks:
\ifthenelse{\isundefined{\hypersetup}}{
  \usepackage[colorlinks=true,linkcolor=blue,urlcolor=blue]{hyperref}
  \urlstyle{same} % normal text font (alternatives: tt, rm, sf)
}{}


%%% Body
\begin{document}

% Automatically generated reST file from Doconce source
% (http://code.google.com/p/doconce/)


%___________________________________________________________________________

\section*{\phantomsection%
  Doconce: Document Once, Include Anywhere%
  \addcontentsline{toc}{section}{Doconce: Document Once, Include Anywhere}%
  \label{doconce-document-once-include-anywhere}%
}
%
\begin{DUfieldlist}
\item[{Author:}]
Hans Petter Langtangen

\item[{Date:}]
May 29, 2013
%
\begin{itemize}

\item When writing a note, report, manual, etc., do you find it difficult
to choose the typesetting format? That is, to choose between plain
(email-like) text, wiki, Word/OpenOffice, LaTeX, HTML,
reStructuredText, Sphinx, XML, etc.  Would it be convenient to
start with some very simple text-like format that easily converts
to the formats listed above, and then at some later stage
eventually go with a particular format?

\item Do you need to write documents in varying formats but find it
difficult to remember all the typesetting details of various
formats like \href{http://refcards.com/docs/silvermanj/amslatex/LaTeXRefCard.v2.0.pdf}{LaTeX}, \href{http://www.htmlcodetutorial.com/}{HTML}, \href{http://docutils.sourceforge.net/docs/ref/rst/restructuredtext.html}{reStructuredText}, \href{http://sphinx.pocoo.org/contents.html}{Sphinx}, and \href{http://code.google.com/p/support/wiki/WikiSyntax}{wiki}? Would it be convenient
to generate the typesetting details of a particular format from a
very simple text-like format with minimal tagging?

\item Do you have the same information scattered around in different
documents in different typesetting formats? Would it be a good idea
to write things once, in one format, stored in one place, and
include it anywhere?

\end{itemize}

\end{DUfieldlist}

If any of these questions are of interest, you should keep on reading.


%___________________________________________________________________________

\section*{\phantomsection%
  Some Doconce Features%
  \addcontentsline{toc}{section}{Some Doconce Features}%
  \label{some-doconce-features}%
}
%
\begin{quote}
%
\begin{itemize}

\item Strong support for texts with much math and code.

\item Same source can produce a variety of output formats.
The following support LaTeX math and (pygmentized) code:

\end{itemize}
%
\begin{quote}
\newcounter{listcnt0}
\begin{list}{\arabic{listcnt0}.}
{
\usecounter{listcnt0}
\setlength{\rightmargin}{\leftmargin}
}

\item \href{http://hplgit.github.com/teamods/writing_reports/_static/report_4printing.pdf}{traditional LaTeX B/W documents for printing}

\item \href{http://hplgit.github.com/teamods/writing_reports/_static/report.pdf}{color LaTeX PDF documents}

\item \href{http://hplgit.github.com/teamods/writing_reports/_static/report_4phone.pdf}{color LaTeX PDF documents for viewing on small phones}

\item \href{http://hplgit.github.com/teamods/writing_reports/_static/sphinx-fenics_minimal/report.html}{Sphinx HTML documents with 20+ different designs}

\item \href{http://hplgit.github.com/teamods/writing_reports/_static/report.html}{Plain HTML} or with a \href{http://hplgit.github.com/teamods/writing_reports/_static/report_vagrant.html}{template} or \href{http://hplgit.github.com/teamods/writing_reports/_static/report_github_minimal.html}{another template} or \href{http://hplgit.github.com/teamods/writing_reports/_static/report_solarized.html}{solarized}

\item HTML for \href{http://doconce-report-demo.blogspot.no/}{Google} or \href{http://doconcereportdemo.wordpress.com/}{Wordpress} blogs

\item \href{http://doconcedemo.shoutwiki.com/wiki/Doconce_demo_page}{MediaWiki} (Wikipedia, Wikibooks, etc)

\item Markdown

\item IPython notebook
\end{list}
%
\begin{quote}

Other formats include plain untagged text (for email),
Creole wiki (for Bitbucket wikis), Google wiki (for Googlecode),
reStructuredText, and Epytext.

\end{quote}

\end{quote}
%
\begin{itemize}

\item Integration with Mako enables use of variables, functions, if-tests,
and loops to parameterize the text and generate various versions of
the text for different purposes.

\item Computer code can be copied directly from parts of source code files.

\item Running text can quickly be edited to slide formats (reveal.js and deck.js,
based on HTML5+CSS3).

\item Special exercise environments with support for hints, answers,
subexercises, etc.

\item Automatic inline embedding of YouTube and Vimeo movies.

\item Good support for admonitions in various LaTeX and HTML styles
for warnings, questions, hints, remarks, summaries, etc.

\end{itemize}

\end{quote}


%___________________________________________________________________________

\section*{\phantomsection%
  What Does Doconce Look Like?%
  \addcontentsline{toc}{section}{What Does Doconce Look Like?}%
  \label{what-does-doconce-look-like}%
}

Doconce text looks like ordinary text (much like Markdown), but there
are some almost invisible text constructions that allow you to control
the formating. Here are some examples.
%
\begin{quote}
%
\begin{itemize}

\item Bullet lists automatically arise from lines starting with \texttt{*},
or \texttt{o} if the list is to be enumerated.

\item \emph{Emphasized words} are surrounded by \texttt{*}. \textbf{Words in boldface}
are surrounded by underscores.

\item Words from computer code are enclosed in back quotes and
then typeset \texttt{verbatim (in a monospace font)}.

\item Section and paragraph headings are recognied special
decorating characters (\texttt{=} or \texttt{\_}) before and after the heading.
The length of the decoration determines the level of the
section.

\item Blocks of computer code are included by surrounding the blocks with
\texttt{!bc} (begin code) and \texttt{!ec} (end code) tags on separate lines.

\item Blocks of computer code can also be imported from source files.

\item Blocks of LaTeX mathematics are included by surrounding the blocks
with \texttt{!bt} (begin TeX) and \texttt{!et} (end TeX) tags on separate lines.

\item There is support for both LaTeX and text-like inline mathematics
such that formulas make sense also when not rendered by LaTeX
or MathJax.

\item Figures and movies with captions, simple tables,
URLs with links, index list, labels and references are supported.
YouTube and Vimeo videos are automatically embedded in web documents.

\item The abstract of a document starts with \emph{Abstract} as paragraph
heading, and all text up to the next heading makes up the abstract,

\item Special comment lines are not visible in the output.

\item Comments to authors can be inserted throughout the text and
made visible or invisible as desired.

\item There is an exercise environment with many advanced features.

\item With a preprocessor, Preprocess or Mako, one can include other
documents (files), large portions of text can be defined in or out
of the text, and tailored format-specific constructs can easily be
included. With Mako a single text can output its program examples
in two or more languages.

\end{itemize}

\end{quote}


%___________________________________________________________________________

\subsection*{\phantomsection%
  What Can Doconce Be Used For?%
  \addcontentsline{toc}{subsection}{What Can Doconce Be Used For?}%
  \label{what-can-doconce-be-used-for}%
}

LaTeX is ideal for articles, thesis, and books, but not so suited
for web documents. Nice environments for web documents, such as
Sphinx, Markdown, or plain HTML, are not particularly well
suited for thesis and books. IPython notebooks are ideal for
documenting computational experiments, but do not (yet) meet the
requirements of books and thesis.

What about migrating a part of a book for blogging? What about making
an MS Word version of parts or an untagged text for inclusion in
email?  What about efficiently generating slides in modern HTML5
style?  Doconce enables all this with just \emph{one source} (the slogan is
\emph{document once - include anywhere}).
Doconce also
has many extra features for supporting documents with much code and
mathematics, not found in any of the mentioned formats.


%___________________________________________________________________________

\subsection*{\phantomsection%
  Basic Syntax%
  \addcontentsline{toc}{subsection}{Basic Syntax}%
  \label{basic-syntax}%
}

Here is an example of some simple text written in the Doconce format:
%
\begin{quote}{\ttfamily \raggedright \noindent
=======~First~a~Section~Heading~=======\\
~\\
Section~headings~have~7~equality~characters~before~and~after~the~heading.\\
With~9~characters~we~have~a~chapter~heading,~5~gives~a~subsection\\
heading,~and~3~a~subsubsection~heading.\\
~\\
=====~A~Subsection~with~Sample~Text~=====\\
label\{my:first:sec\}\\
~\\
Ordinary~text~looks~like~ordinary~text,~and~the~tags~used~for\\
\_boldface\_~words,~*emphasized*~words,~and~`computer`~words~look\\
natural~in~plain~text.~~Lists~are~typeset~as~you~would~do~in~email,\\
~\\
~~*~item~1\\
~~*~item~2\\
~~*~item~3\\
~\\
Lists~can~also~have~automatically~numbered~items~instead~of~bullets,\\
~\\
~~o~item~1\\
~~o~item~2\\
~~o~item~3\\
~\\
\_\_Hyperlinks.\_\_~Paragraph~headings~are~surrounded~by~double~underscores.\\
URLs~with~a~link~word~are~possible,~as~in~"hpl":~"http://folk.uio.no/hpl".\\
If~the~word~is~URL,~the~URL~itself~becomes~the~link~name,\\
as~in~"URL":~"tutorial.do.txt".~Doconce~distinguishes~between~paper\\
and~screen~output.~In~traditional~paper~output,~in~PDF~generated~from~LaTeX\\
generated~from~Doconce,~the~URLs~of~links~appear~as~footnotes.\\
With~screen~output,~all~links~are~clickable~hyperlinks,~except~in\\
the~plain~text~format~which~does~not~support~hyperlinks.\\
~\\
References~to~sections~may~use~logical~names~as~labels~(e.g.,~a\\
"label"~command~right~after~the~section~title),~as~in~the~reference~to\\
Section~ref\{my:first:sec\}.\\
~\\
Doconce~also~allows~inline~comments~of~the~form~{[}name:~comment{]}~(with\\
a~space~after~`name:`),~e.g.,~such~as~{[}hpl:~here~I~will~make~some\\
remarks~to~the~text{]}.~Inline~comments~can~be~removed~from~the~output\\
by~a~command-line~argument~(see~Section~ref\{doconce2formats\}~for~an\\
example).\\
~\\
Tables~are~also~supperted,~e.g.,\\
~\\
~~|-{}-{}-{}-{}-{}-{}-{}-{}-{}-{}-{}-{}-{}-{}-{}-{}-{}-{}-{}-{}-{}-{}-{}-{}-{}-{}-{}-{}-{}-{}-{}-|\\
~~|time~~|~velocity~|~acceleration~|\\
~~|-{}-{}-r-{}-{}-{}-{}-{}-{}-r-{}-{}-{}-{}-{}-{}-{}-{}-{}-{}-r-{}-{}-{}-{}-{}-{}-{}-|\\
~~|~0.0~~|~1.4186~~~|~-5.01~~~~~~~~|\\
~~|~2.0~~|~1.376512~|~11.919~~~~~~~|\\
~~|~4.0~~|~1.1E+1~~~|~14.717624~~~~|\\
~~|-{}-{}-{}-{}-{}-{}-{}-{}-{}-{}-{}-{}-{}-{}-{}-{}-{}-{}-{}-{}-{}-{}-{}-{}-{}-{}-{}-{}-{}-{}-{}-|\\
~\\
\#~lines~beginning~with~\#~are~comment~lines
}
\end{quote}

The Doconce text above results in the following little document:


%___________________________________________________________________________

\section*{\phantomsection%
  First a Section Heading%
  \addcontentsline{toc}{section}{First a Section Heading}%
  \label{first-a-section-heading}%
}

Section headings have 7 equality characters before and after the heading.
With 9 characters we have a chapter heading, 5 gives a subsection
heading, and 3 a subsubsection heading.


%___________________________________________________________________________

\subsection*{\phantomsection%
  A Subsection with Sample Text%
  \addcontentsline{toc}{subsection}{A Subsection with Sample Text}%
  \label{a-subsection-with-sample-text}%
  \label{my-first-sec}%
}

Ordinary text looks like ordinary text, and the tags used for
\textbf{boldface} words, \emph{emphasized} words, and \texttt{computer} words look
natural in plain text.  Lists are typeset as you would do in an email,
%
\begin{quote}
%
\begin{itemize}

\item item 1

\item item 2

\item item 3

\end{itemize}

\end{quote}

Lists can also have numbered items instead of bullets, just use an \texttt{o}
(for ordered) instead of the asterisk:
%
\begin{quote}
\setcounter{listcnt0}{0}
\begin{list}{\arabic{listcnt0}.}
{
\usecounter{listcnt0}
\setlength{\rightmargin}{\leftmargin}
}

\item item 1

\item item 2

\item item 3
\end{list}

\end{quote}

\emph{Hyperlinks.} Paragraph headings are surrounded by double underscores.
URLs with a link word are possible, as in \href{http://folk.uio.no/hpl}{hpl}.
If the word is URL, the URL itself becomes the link name,
as in \url{tutorial.do.txt}. Doconce distinguishes between paper
and screen output. In traditional paper output, in PDF generated from LaTeX
generated from Doconce, the URLs of links appear as footnotes.
With screen output, all links are clickable hyperlinks, except in
the plain text format which does not support hyperlinks.

References to sections may use logical names as labels (e.g., a
``label'' command right after the section title), as in the reference to
the section \hyperref[a-subsection-with-sample-text]{A Subsection with Sample Text}.

Doconce also allows inline comments such as (\textbf{hpl 1}: here I will make
some remarks to the text) for allowing authors to make notes. Inline
comments can be removed from the output by a command-line argument
(see the section \hyperref[from-doconce-to-other-formats]{From Doconce to Other Formats} for an example). Ordinary comment
lines start with \texttt{\#} and are copied to comment lines in the
output format.

Tables are also supperted, e.g.,

\setlength{\DUtablewidth}{\linewidth}
\begin{longtable*}[c]{|p{0.156\DUtablewidth}|p{0.156\DUtablewidth}|p{0.156\DUtablewidth}|}
\hline
\textbf{%
time
} & \textbf{%
velocity
} & \textbf{%
acceleration
} \\
\hline
\endfirsthead
\hline
\textbf{%
time
} & \textbf{%
velocity
} & \textbf{%
acceleration
} \\
\hline
\endhead
\multicolumn{3}{c}{\hfill ... continued on next page} \\
\endfoot
\endlastfoot

0.0
 & 
1.4186
 & 
-5.01
 \\
\hline

2.0
 & 
1.376512
 & 
11.919
 \\
\hline

4.0
 & 
1.1E+1
 & 
14.717624
 \\
\hline
\end{longtable*}


%___________________________________________________________________________

\subsection*{\phantomsection%
  Mathematics and Computer Code%
  \addcontentsline{toc}{subsection}{Mathematics and Computer Code}%
  \label{mathematics-and-computer-code}%
}

Inline mathematics, such as v = sin(x),
allows the formula to be specified both as LaTeX and as plain text.
This results in a professional LaTeX typesetting, but in formats
not supporting LaTeX mathematics
the text version normally looks better than raw LaTeX mathematics with
backslashes. An inline formula like v = sin(x) is
typeset as:
%
\begin{quote}{\ttfamily \raggedright \noindent
\$\textbackslash{}nu~=~\textbackslash{}sin(x)\$|\$v~=~sin(x)\$
}
\end{quote}

The pipe symbol acts as a delimiter between LaTeX code and the plain text
version of the formula. If you write a lot of mathematics, only the
output formats \texttt{latex}, \texttt{pdflatex}, \texttt{html}, \texttt{sphinx}, and \texttt{pandoc}
are of interest
and all these support inline LaTeX mathematics so then you will naturally
drop the pipe symbol and write just:
%
\begin{quote}{\ttfamily \raggedright \noindent
\$\textbackslash{}nu~=~\textbackslash{}sin(x)\$
}
\end{quote}

However, if you want more textual formats, like plain text or reStructuredText,
the text after the pipe symbol may help to make the math formula more readable
if there are backslahes or other special LaTeX symbols in the LaTeX code.

Blocks of mathematics are typeset with raw LaTeX, inside
\texttt{!bt} and \texttt{!et} (begin TeX, end TeX) instructions:
%
\begin{quote}{\ttfamily \raggedright \noindent
!bt\\
\textbackslash{}begin\{align\}\\
\{\textbackslash{}partial~u\textbackslash{}over\textbackslash{}partial~t\}~\&=~\textbackslash{}nabla\textasciicircum{}2~u~+~f,~label\{myeq1\}\textbackslash{}\textbackslash{}\\
\{\textbackslash{}partial~v\textbackslash{}over\textbackslash{}partial~t\}~\&=~\textbackslash{}nabla\textbackslash{}cdot(q(u)\textbackslash{}nabla~v)~+~g\\
\textbackslash{}end\{align\}\\
!et
}
\end{quote}

The result looks like this:
%
\begin{quote}{\ttfamily \raggedright \noindent
\textbackslash{}begin\{align\}\\
\{\textbackslash{}partial~u\textbackslash{}over\textbackslash{}partial~t\}~\&=~\textbackslash{}nabla\textasciicircum{}2~u~+~f,~label\{myeq1\}\textbackslash{}\textbackslash{}\\
\{\textbackslash{}partial~v\textbackslash{}over\textbackslash{}partial~t\}~\&=~\textbackslash{}nabla\textbackslash{}cdot(q(u)\textbackslash{}nabla~v)~+~g\\
\textbackslash{}end\{align\}
}
\end{quote}

Of course, such blocks only looks nice in formats with support
for LaTeX mathematics, and here the align environment in particular
(this includes \texttt{latex}, \texttt{pdflatex}, \texttt{html}, and \texttt{sphinx}). The raw
LaTeX syntax appears in simpler formats, but can still be useful
for those who can read LaTeX syntax.

You can have blocks of computer code, starting and ending with
\texttt{!bc} and \texttt{!ec} instructions, respectively:
%
\begin{quote}{\ttfamily \raggedright \noindent
!bc~pycod\\
from~math~import~sin,~pi\\
def~myfunc(x):\\
~~~~return~sin(pi*x)\\
~\\
import~integrate\\
I~=~integrate.trapezoidal(myfunc,~0,~pi,~100)\\
!ec
}
\end{quote}

Such blocks are formatted as:
%
\begin{quote}{\ttfamily \raggedright \noindent
from~math~import~sin,~pi\\
def~myfunc(x):\\
~~~~return~sin(pi*x)\\
~\\
import~integrate\\
I~=~integrate.trapezoidal(myfunc,~0,~pi,~100)
}
\end{quote}

A code block must come after some plain sentence (at least for successful
output to \texttt{sphinx}, \texttt{rst}, and formats close to plain text),
not directly after a section/paragraph heading or a table.

One can also copy computer code directly from files, either the
complete file or specified parts.  Computer code is then never
duplicated in the documentation (important for the principle of
avoiding copying information!).

Another document can be included by writing \texttt{\# \#include "mynote.do.txt"}
at the beginning of a line.  Doconce documents have
extension \texttt{do.txt}. The \texttt{do} part stands for doconce, while the
trailing \texttt{.txt} denotes a text document so that editors gives you
plain text editing capabilities.


%___________________________________________________________________________

\subsection*{\phantomsection%
  Macros (Newcommands), Cross-References, Index, and Bibliography%
  \addcontentsline{toc}{subsection}{Macros (Newcommands), Cross-References, Index, and Bibliography}%
  \label{macros-newcommands-cross-references-index-and-bibliography}%
  \label{newcommands}%
}

Doconce supports a type of macros via a LaTeX-style \emph{newcommand}
construction.  The newcommands are defined in files with names
\texttt{newcommands*.tex}, using standard LaTeX syntax. Only newcommands
for use inside math environments are supported.

Labels, corss-references, citations, and support of an index and
bibliography are much inspired by LaTeX syntax, but Doconce features
no backslashes. Use labels for sections and equations only, and
preceed the reference by ``Section'' or ``Chapter'', or in case of
an equation, surround the reference by parenthesis.

Here is an example:
%
\begin{quote}{\ttfamily \raggedright \noindent
=====~My~Section~=====\\
label\{sec:mysec\}\\
~\\
idx\{key~equation\}~idx\{\$\textbackslash{}u\$~conservation\}\\
~\\
We~refer~to~Section~ref\{sec:yoursec\}~for~background~material~on\\
the~*key~equation*.~Here~we~focus~on~the~extension\\
~\\
\#~\textbackslash{}Ddt,~\textbackslash{}u~and~\textbackslash{}mycommand~are~defined~in~newcommands\_keep.tex\\
~\\
!bt\\
\textbackslash{}begin\{equation\}\\
\textbackslash{}Ddt\{\textbackslash{}u\}~=~\textbackslash{}mycommand\{v\},\\
label\{mysec:eq:Dudt\}\\
\textbackslash{}end\{equation\}\\
!et\\
where~\$\textbackslash{}Ddt\{\textbackslash{}u\}\$~is~the~material~derivative~of~\$\textbackslash{}u\$.\\
Equation~(ref\{mysec:eq:Dudt\})~is~important~in~a~number\\
of~contexts,~see~cite\{Larsen\_et\_al\_2002,Johnson\_Friedman\_2010a\}.\\
Also,~cite\{Miller\_2000\}~supports~such~a~view.\\
~\\
As~see~in~Figure~ref\{mysec:fig:myfig\},~the~key~equation\\
features~large,~smooth~regions~*and*~abrupt~changes.\\
~\\
FIGURE:~{[}fig/myfile,~width=600{]}~My~figure.~label\{mysec:fig:myfig\}\\
~\\
=====~References~=====\\
~\\
BIBFILE:~papers.pub
}
\end{quote}

For further details on functionality and
syntax we refer to the \texttt{doc/manual/manual.do.txt} file (see the
\href{https://doconce.googlecode.com/hg/doc/demos/manual/index.html}{demo page}
for various formats of this document).


%___________________________________________________________________________

\section*{\phantomsection%
  From Doconce to Other Formats%
  \addcontentsline{toc}{section}{From Doconce to Other Formats}%
  \label{from-doconce-to-other-formats}%
  \label{doconce2formats}%
}

Transformation of a Doconce document \texttt{mydoc.do.txt} to various other
formats applies the script \texttt{doconce format}:
%
\begin{quote}{\ttfamily \raggedright \noindent
Terminal>~doconce~format~format~mydoc.do.txt
}
\end{quote}

or just:
%
\begin{quote}{\ttfamily \raggedright \noindent
Terminal>~doconce~format~format~mydoc
}
\end{quote}


%___________________________________________________________________________

\subsection*{\phantomsection%
  Generating a makefile%
  \addcontentsline{toc}{subsection}{Generating a makefile}%
  \label{generating-a-makefile}%
}

Producing HTML, Sphinx, and in particular LaTeX documents from
Doconce sources requires a few commands. Often you want to
produce several different formats. The relevant commands should
then be placed in a script that acts as a ``makefile''.

The \texttt{doconce makefile} can be used to automatically generate
such a makefile, more precisely a Bash script \texttt{make.sh}, which
carries out the commands explained below. If our Doconce source
is in \texttt{main\_myproj.do.txt}, we run:
%
\begin{quote}{\ttfamily \raggedright \noindent
doconce~makefile~main\_myproj~html~pdflatex~sphinx
}
\end{quote}

to produce the necessary output for generating HTML, pdfLaTeX, and
Sphinx. Usually, you need to edit \texttt{make.sh} to really fit your
needs. Some examples lines are inserted as comments to show
various options that can be added to the basic commands.
A handy feature of the generated \texttt{make.sh} script is that it
inserts checks for successful runs of the \texttt{doconce format} commands,
and if something goes wrong, the \texttt{make.sh} exits.


%___________________________________________________________________________

\subsection*{\phantomsection%
  Preprocessing%
  \addcontentsline{toc}{subsection}{Preprocessing}%
  \label{preprocessing}%
}

The \texttt{preprocess} and \texttt{mako} programs are used to preprocess the
file, and options to \texttt{preprocess} and/or \texttt{mako} can be added after the
filename. For example:
%
\begin{quote}{\ttfamily \raggedright \noindent
Terminal>~doconce~format~latex~mydoc~-Dextra\_sections~-DVAR1=5~~~~~\#~preprocess\\
Terminal>~doconce~format~latex~yourdoc~extra\_sections=True~VAR1=5~~\#~mako
}
\end{quote}

The variable \texttt{FORMAT} is always defined as the current format when
running \texttt{preprocess} or \texttt{mako}. That is, in the last example, \texttt{FORMAT} is
defined as \texttt{latex}. Inside the Doconce document one can then perform
format specific actions through tests like \texttt{\#if FORMAT == "latex"}
(for \texttt{preprocess}) or \texttt{\% if FORMAT == "latex":} (for \texttt{mako}).


%___________________________________________________________________________

\subsection*{\phantomsection%
  Removal of inline comments%
  \addcontentsline{toc}{subsection}{Removal of inline comments}%
  \label{removal-of-inline-comments}%
}

The command-line arguments \texttt{-{}-no\_preprocess} and \texttt{-{}-no\_mako} turn off
running \texttt{preprocess} and \texttt{mako}, respectively.

Inline comments in the text are removed from the output by:
%
\begin{quote}{\ttfamily \raggedright \noindent
Terminal>~doconce~format~latex~mydoc~-{}-skip\_inline\_comments
}
\end{quote}

One can also remove all such comments from the original Doconce
file by running:
%
\begin{quote}{\ttfamily \raggedright \noindent
Terminal>~doconce~remove\_inline\_comments~mydoc
}
\end{quote}

This action is convenient when a Doconce document reaches its final form
and comments by different authors should be removed.


%___________________________________________________________________________

\subsection*{\phantomsection%
  Notes%
  \addcontentsline{toc}{subsection}{Notes}%
  \label{notes}%
}

Doconce does not have a tag for longer notes, because implementation
of a ``notes feature'' is so easy using the \texttt{preprocess} or \texttt{mako}
programs. Just introduce some variable, say \texttt{NOTES}, that you define
through \texttt{-DNOTES} (or not) when running \texttt{doconce format ...}. Inside
the document you place your notes between \texttt{\# \#ifdef NOTES} and
\texttt{\# \#endif} preprocess tags. Alternatively you use \texttt{\% if NOTES:}
and \texttt{\% endif} that \texttt{mako} will recognize. In the same way you may
encapsulate unfinished material, extra material to be removed
for readers but still nice to archive as part of the document for
future revisions.


%___________________________________________________________________________

\subsection*{\phantomsection%
  Demo of different formats%
  \addcontentsline{toc}{subsection}{Demo of different formats}%
  \label{demo-of-different-formats}%
}

A simple scientific report is available in \href{http://hplgit.github.com/teamods/writing_reports/doconce_commands.html}{a lot of different formats}.
How to create the different formats is explained in more depth
in the coming sections.


%___________________________________________________________________________

\subsection*{\phantomsection%
  HTML%
  \addcontentsline{toc}{subsection}{HTML}%
  \label{id1}%
}

Making an HTML version of a Doconce file \texttt{mydoc.do.txt}
is performed by:
%
\begin{quote}{\ttfamily \raggedright \noindent
Terminal>~doconce~format~html~mydoc
}
\end{quote}

The resulting file \texttt{mydoc.html} can be loaded into any web browser for viewing.

The HTML style can be defined either in the header of the HTML file,
using a named built-in style;
in an external CSS file; or in a template file.

An external CSS file \texttt{filename} used by setting the command-line
argument \texttt{-{}-css=filename}. There available built-in styles are
specified as \texttt{-{}-html\_style=name}, where \texttt{name} can be
%
\begin{quote}
%
\begin{itemize}

\item 
\DUadmonition[system-message]{
\DUtitle[system-message]{system-message}


{\color{red}WARNING/2} in \texttt{tutorial.rst}, line~429

\hyperlink{id2}{
Duplicate explicit target name: ``solarized''.
}}

\texttt{solarized}: the famous \href{http://ethanschoonover.com/solarized}{solarized}
style (yellowish),

\item \texttt{blueish}: a simple style with blue headings (default),

\item \texttt{blueish2}: a variant of \emph{bluish},

\item \texttt{bloodish}: as \texttt{bluish}, but dark read as color.

\end{itemize}

\end{quote}

Using \texttt{-{}-css=filename} where \texttt{filename} is a non-existing file makes
Doconce write the built-in style to that file. Otherwise the HTML
links to the CSS stylesheet in \texttt{filename}. Several stylesheets can
be specified: \texttt{-{}-ccs=file1.css,file2.css,file3.css}.

Templates are HTML files with ``slots'' \texttt{\%(main)s} for the main body
of text, \texttt{\%(title)s} for the title, and \texttt{\%(date)s} for the date.
Doconce comes with a few templates. The usage of templates is
described in a \href{https://doconce.googlecode.com/hg/doc/design/wrapper_tech.html}{separate document}. That document describes how you your Doconce-generated
HTML file can have any specified layout.

If the Pygments package (including the \texttt{pygmentize} program)
is installed, code blocks are typeset with
aid of this package. The command-line argument \texttt{-{}-no\_pygments\_html}
turns off the use of Pygments and makes code blocks appear with
plain (\texttt{pre}) HTML tags. The option \texttt{-{}-pygments\_html\_linenos} turns
on line numbers in Pygments-formatted code blocks. A specific
Pygments style is set by \texttt{-{}-pygments\_html\_style=style}, where \texttt{style}
can be \texttt{default}, \texttt{emacs}, \texttt{perldoc}, and other valid names for
Pygments styles.

The HTML file can be embedded in a template with your own tailored
design, see a ``tutorial'': `` \url{https://doconce.googlecode.com/hg/doc/design/wrapper_tech.html}'' on this topic. The template file must contain
valid HTML code and can have three ``slots'': \texttt{\%(title)s} for a title,
\texttt{\%(date)s} for a date, and \texttt{\%(main)s} for the main body of text. The
latter is the
Doconce document translated to HTML. The title becomes the first
heading in the Doconce document, or the title (but a title is not
recommended when using templates). The date is extracted from the
\texttt{DATE:} line. With the template feature one can easily embed
the text in the look and feel of a website. Doconce comes with
two templates in \texttt{bundled/html\_styles}. Just copy the directory
containing the template and the CSS and JavaScript files to your
document directory, edit the template as needed (also check that
paths to the \texttt{css} and \texttt{js} subdirectories are correct - according
to how you store the template files), and run:
%
\begin{quote}{\ttfamily \raggedright \noindent
Terminal>~doconce~format~html~mydoc~-{}-html\_template=mytemplate.html
}
\end{quote}

The template in \texttt{style\_vagrant} also needs an extra option
\texttt{-{}-html\_style=vagrant}. With this style, one has nice navigation buttons
that are used if the document contains \texttt{!split} commands for splitting
it into many pages.


%___________________________________________________________________________

\subsection*{\phantomsection%
  Blog Posts%
  \addcontentsline{toc}{subsection}{Blog Posts}%
  \label{blog-posts}%
}

Doconce can be used for writing blog posts provided the blog site accepts
raw HTML code. Google's Blogger service (\texttt{blogger.com} or
\texttt{blogname.blogspot.com}) is particularly well suited since it also
allows extensive LaTeX mathematics via MathJax.
\setcounter{listcnt0}{0}
\begin{list}{\arabic{listcnt0}.}
{
\usecounter{listcnt0}
\setlength{\rightmargin}{\leftmargin}
}

\item Write the blog text as a Doconce document without any
title, author, and date.

\item Generate HTML as described above.

\item Copy the text and paste it into the
text area in the blog (just delete the HTML code that initially
pops up in the text area). Make sure the input format is HTML.
\end{list}

See a \href{http://doconce.blogspot.no}{simple blog example} and
a \href{http://doconce-report-demo.blogspot.no/}{scientific report}
for demonstrations of blogs at \texttt{blogspot.no}.

\DUadmonition[warning]{
\DUtitle[warning]{Warning}

In the readers' comments after the blog one cannot paste raw HTML
code with MathJax
scripts so there is no support for mathematics in the discussion forum.
}

\DUadmonition[note]{
\DUtitle[note]{Note}

Figure files must be uploaded to some web site and the filenames name must
be replaced by the relevant URL. This can be automatically edited:
%
\begin{quote}{\ttfamily \raggedright \noindent
cp~mydoc.do.txt~mydoc2.do.txt\\
url="https//raw.github.com/someuser/someuser.github.com"\\
dir="master/project/dir1/dir2"\\
for~figname~in~fig1~fig2~fig3;~do\\
~~doconce~replace~"{[}\$figname,"~"{[}\$site/\$dir/\$figname.png,"~\textbackslash{}\\
~~~~~~~~~~mydoc2.do.txt\\
done\\
doconce~format~html~mydoc2\\
\#~Paste~mydoc2.html~into~a~new~blog~page
}
\end{quote}
}

Blog posts at Google can also be published \href{http://support.google.com/blogger/bin/answer.py?hl=en&answer=41452}{automatically through email}.
A Python program can send the contents of the HTML file
to the blog's email address using the packages  \texttt{smtplib} and \texttt{email}.

\DUadmonition[system-message]{
\DUtitle[system-message]{system-message}


{\color{red}WARNING/2} in \texttt{tutorial.rst}, line~429

\hyperlink{id3}{
Duplicate explicit target name: ``scientific report''.
}}

WordPress (\texttt{wordpress.com}) allows raw HTML code in blogs,
but has very limited
LaTeX support, basically only formulas. The \texttt{-{}-wordpress} option to
\texttt{doconce} modifies the HTML code such that all equations are typeset
in a way that is acceptable to WordPress.
Look at a \href{http://doconce.wordpress.com}{simple doconce example}
and a \href{http://doconcereportdemo.wordpress.com/}{scientific report}
to see blogging with mathematics and code on WordPress.

Speaking of WordPress, the related project \url{http://pressbooks.com} can take raw HTML code (from Doconce, for
instance) and produce very nice-looking books.  There is no support
for mathematics in the text, though.


%___________________________________________________________________________

\subsection*{\phantomsection%
  Pandoc and Markdown%
  \addcontentsline{toc}{subsection}{Pandoc and Markdown}%
  \label{pandoc-and-markdown}%
}

Output in Pandoc's extended Markdown format results from:
%
\begin{quote}{\ttfamily \raggedright \noindent
Terminal>~doconce~format~pandoc~mydoc
}
\end{quote}

The name of the output file is \texttt{mydoc.mkd}.
From this format one can go to numerous other formats:
%
\begin{quote}{\ttfamily \raggedright \noindent
Terminal>~pandoc~-R~-t~mediawiki~-o~mydoc.mwk~-{}-toc~mydoc.mkd
}
\end{quote}

Pandoc supports \texttt{latex}, \texttt{html}, \texttt{odt} (OpenOffice), \texttt{docx} (Microsoft
Word), \texttt{rtf}, \texttt{texinfo}, to mention some. The \texttt{-R} option makes
Pandoc pass raw HTML or LaTeX to the output format instead of ignoring it,
while the \texttt{-{}-toc} option generates a table of contents.
See the \href{http://johnmacfarlane.net/pandoc/README.html}{Pandoc documentation}
for the many features of the \texttt{pandoc} program. The HTML output from
\texttt{pandoc} needs adjustments to provide full support for MathJax LaTeX
mathematics, and for this purpose one should use \texttt{doconce md2html}:
%
\begin{quote}{\ttfamily \raggedright \noindent
Terminal>~doconce~format~pandoc~mydoc\\
Terminal>~doconce~m2html~mydoc
}
\end{quote}

The result \texttt{mydoc.html} can be viewed in a browser.

Pandoc is useful to go from LaTeX mathematics to, e.g., HTML or MS
Word.  There are two ways (experiment to find the best one for your
document): \texttt{doconce format pandoc} and then translating using \texttt{doconce
md2latex} (which runs \texttt{pandoc}), or \texttt{doconce format latex}, and then
going from LaTeX to the desired format using \texttt{pandoc}.
Here is an example on the latter strategy:
%
\begin{quote}{\ttfamily \raggedright \noindent
Terminal>~doconce~format~latex~mydoc\\
Terminal>~doconce~ptex2tex~mydoc\\
Terminal>~doconce~replace~'\textbackslash{}Verb!'~'\textbackslash{}verb!'~mydoc.tex\\
Terminal>~pandoc~-f~latex~-t~docx~-o~mydoc.docx~mydoc.tex
}
\end{quote}

When we go through \texttt{pandoc}, only single equations, \texttt{align}, or \texttt{align*}
environments are well understood for output to HTML.

Note that Doconce applies the \texttt{Verb} macro from the \texttt{fancyvrb} package
while \texttt{pandoc} only supports the standard \texttt{verb} construction for
inline verbatim text.  Moreover, quite some additional \texttt{doconce
replace} and \texttt{doconce subst} edits might be needed on the \texttt{.mkd} or
\texttt{.tex} files to successfully have mathematics that is well translated
to MS Word.  Also when going to reStructuredText using Pandoc, it can
be advantageous to go via LaTeX.

Here is an example where we take a Doconce snippet (without title, author,
and date), maybe with some unnumbered equations, and quickly generate
HTML with mathematics displayed my MathJax:
%
\begin{quote}{\ttfamily \raggedright \noindent
Terminal>~doconce~format~pandoc~mydoc\\
Terminal>~pandoc~-t~html~-o~mydoc.html~-s~-{}-mathjax~mydoc.mkd
}
\end{quote}

The \texttt{-s} option adds a proper header and footer to the \texttt{mydoc.html} file.
This recipe is a quick way of makeing HTML notes with (some) mathematics.


%___________________________________________________________________________

\subsection*{\phantomsection%
  LaTeX%
  \addcontentsline{toc}{subsection}{LaTeX}%
  \label{id4}%
}

Making a LaTeX file \texttt{mydoc.tex} from \texttt{mydoc.do.txt} is done in two steps:

% Note: putting code blocks inside a list is not successful in many

% formats - the text may be messed up. A better choice is a paragraph

% environment, as used here.

\emph{Step 1.} Filter the doconce text to a pre-LaTeX form \texttt{mydoc.p.tex} for
the \texttt{ptex2tex} program (or \texttt{doconce ptex2tex}):
%
\begin{quote}{\ttfamily \raggedright \noindent
Terminal>~doconce~format~latex~mydoc
}
\end{quote}

LaTeX-specific commands (``newcommands'') in math formulas and similar
can be placed in files \texttt{newcommands.tex}, \texttt{newcommands\_keep.tex}, or
\texttt{newcommands\_replace.tex} (see the section \hyperref[macros-newcommands-cross-references-index-and-bibliography]{Macros (Newcommands), Cross-References, Index, and Bibliography}).
If these files are present, they are included in the LaTeX document
so that your commands are defined.

An option \texttt{-{}-latex\_printed} makes some adjustments for documents
aimed at being printed. For example, links to web resources are
associated with a footnote listing the complete web address (URL).

\emph{Step 2.} Run \texttt{ptex2tex} (if you have it) to make a standard LaTeX file:
%
\begin{quote}{\ttfamily \raggedright \noindent
Terminal>~ptex2tex~mydoc
}
\end{quote}

In case you do not have \texttt{ptex2tex}, you may run a (very) simplified version:
%
\begin{quote}{\ttfamily \raggedright \noindent
Terminal>~doconce~ptex2tex~mydoc
}
\end{quote}

Note that Doconce generates a \texttt{.p.tex} file with some preprocessor macros
that can be used to steer certain properties of the LaTeX document.
For example, to turn on the Helvetica font instead of the standard
Computer Modern font, run:
%
\begin{quote}{\ttfamily \raggedright \noindent
Terminal>~ptex2tex~-DHELVETICA~mydoc\\
Terminal>~doconce~ptex2tex~mydoc~-DHELVETICA~~\#~alternative
}
\end{quote}

The title, authors, and date are by default typeset in a non-standard
way to enable a nicer treatment of multiple authors having
institutions in common. However, the standard LaTeX ``maketitle'' heading
is also available through \texttt{-DLATEX\_HEADING=traditional}.
A separate titlepage can be generate by
\texttt{-DLATEX\_HEADING=titlepage}.

Preprocessor variables to be defined or undefined are
%
\begin{quote}
%
\begin{itemize}

\item \texttt{XELATEX} for processing by \texttt{xelatex}

\item \texttt{PALATINO} for the Palatino font

\item \texttt{HELVETICA} for the Helvetica font

\item \texttt{A4PAPER} for A4 paper size

\item \texttt{A6PAPER} for A6 paper size (suitable for reading PDFs on phones)

\item \texttt{MOVIE} for specifying how movies are handled: the value \texttt{media9}
implies the \texttt{media9} package and the \texttt{\textbackslash{}includemedia} command (default),
while other values are \texttt{movie15} (\texttt{\textbackslash{}includemovie} command),
\texttt{multimedia} (for Beamer-style \texttt{\textbackslash{}movie} command),
or \texttt{href-run} (for the plain \DUroletitlereference{h`run:file`\_} command)

\item \texttt{PREAMBLE} to turn the LaTeX preamble on or off (i.e., complete document
or document to be included elsewhere - and note that
the preamble is only included
if the document has a title, author, and date)

\item \texttt{MINTED} for inclusion of the minted package for typesetting of
code with the Pygments tool (which requires \texttt{latex}
or \texttt{pdflatex} to be run with the \texttt{-shell-escape} option)

\item \texttt{TODONOTES} for using the fancy \texttt{todonotes} package for typesetting
inline comments (looks much like track changes in MS Word). This
macro has only effect if inline comments are used (name, colon,
and comment inside brackets).

\item \texttt{LINENUMBERS} for inclusion of line numbers in the text.

\item \texttt{AMON} for setting the type of admonitions: \texttt{"colors"} for colored
boxes with icons, \texttt{"graybox1"} for gray frame boxes with rounded
corners (default), \texttt{"graybox2"} for narrower square gray frame boxes
(except for summary, which for A4 format is small and with wrapped
text around if it does not contain verbatim code),
or \texttt{"paragraph"} for simple, plain paragraph headings and ordinary text

\item \texttt{COLORED\_TABLE\_ROWS} for coloring every other table rows (set this
variable to \texttt{gray} or \texttt{blue})

\item \texttt{BLUE\_SECTION\_HEADINGS} for blue section and subsection headings

\item \texttt{LATEX\_HEADING} for the typesetting of the title, author, parts of
preamble (values: \texttt{traditional} for traditional LaTeX heading,
\texttt{titlepage} for a separate titlepage, \texttt{Springer\_collection} for
edited volumes on Springer, \texttt{beamer} for Beamer slides, \texttt{doconce\_heading}
(default) for listing institutions after names)

\end{itemize}

\end{quote}

If you are not satisfied with the Doconce preamble, you can provide
your own preamble by adding the command-line option \texttt{-{}-latex\_preamble=myfile}.
In case \texttt{myfile} contains a documentclass definition, Doconce assumes
that the file contains the \emph{complete} preamble you want (not that all
the packages listed in the default preamble are required and must be
present in \texttt{myfile}). Otherwise, \texttt{myfile} is assumed to contain
\emph{additional} LaTeX code to be added to the Doconce default preamble.

The \texttt{ptex2tex} tool makes it possible to easily switch between many
different fancy formattings of computer or verbatim code in LaTeX
documents. After any \texttt{!bc} command in the Doconce source you can
insert verbatim block styles as defined in your \texttt{.ptex2tex.cfg}
file, e.g., \texttt{!bc sys} for a terminal session, where \texttt{sys} is set to
a certain environment in \texttt{.ptex2tex.cfg} (e.g., \texttt{CodeTerminal}).
There are about 40 styles to choose from, and you can easily add
new ones.

Also the \texttt{doconce ptex2tex} command supports preprocessor directives
for processing the \texttt{.p.tex} file. The command allows specifications
of code environments as well. Here is an example:
%
\begin{quote}{\ttfamily \raggedright \noindent
Terminal>~doconce~ptex2tex~mydoc~-DLATEX\_HEADING=traditional~\textbackslash{}\\
~~~~~~~~~~-DPALATINO~-DA6PAPER~\textbackslash{}\\
~~~~~~~~~~"sys=\textbackslash{}begin\{quote\}\textbackslash{}begin\{verbatim\}@\textbackslash{}end\{verbatim\}\textbackslash{}end\{quote\}"~\textbackslash{}\\
~~~~~~~~~~fpro=minted~fcod=minted~shcod=Verbatim~envir=ans:nt
}
\end{quote}

Note that \texttt{@} must be used to separate the begin and end LaTeX
commands, unless only the environment name is given (such as \texttt{minted}
above, which implies \texttt{\textbackslash{}begin\{minted\}\{fortran\}} and \texttt{\textbackslash{}end\{minted\}} as
begin and end for blocks inside \texttt{!bc fpro} and \texttt{!ec}).  Specifying
\texttt{envir=ans:nt} means that all other environments are typeset with the
\texttt{anslistings.sty} package, e.g., \texttt{!bc cppcod} will then result in
\texttt{\textbackslash{}begin\{c++\}}. If no environments like \texttt{sys}, \texttt{fpro}, or the common
\texttt{envir} are defined on the command line, the plain \texttt{\textbackslash{}begin\{verbatim\}}
and \texttt{\textbackslash{}end\{verbatim\}} used.

\emph{Step 2b (optional).} Edit the \texttt{mydoc.tex} file to your needs.
For example, you may want to substitute \texttt{section} by \texttt{section*} to
avoid numbering of sections, you may want to insert linebreaks
(and perhaps space) in the title, etc. This can be automatically
edited with the aid of the \texttt{doconce replace} and \texttt{doconce subst}
commands. The former works with substituting text directly, while the
latter performs substitutions using regular expressions.
You will use \texttt{doconce replace} to edit \texttt{section\{} to \texttt{section*\{}:
%
\begin{quote}{\ttfamily \raggedright \noindent
Terminal>~doconce~replace~'section\{'~'section*\{'~mydoc.tex
}
\end{quote}

For fixing the line break of a title, you may pick a word in the
title, say ``Using'', and insert a break after than word. With
\texttt{doconce subst} this is easy employing regular expressions with
a group before ``Using'' and a group after:
%
\begin{quote}{\ttfamily \raggedright \noindent
Terminal>~doconce~subst~'title\textbackslash{}\{(.+)Using~(.+)\textbackslash{}\}'~\textbackslash{}\\
~~~~~~~~~~'title\{\textbackslash{}g<1>~\textbackslash{}\textbackslash{}\textbackslash{}\textbackslash{}~{[}1.5mm{]}~Using~\textbackslash{}g<2>'~mydoc.tex
}
\end{quote}

A lot of tailored fixes to the LaTeX document can be done by
an appropriate set of text replacements and regular expression
substitutions. You are anyway encourged to make a script for
generating PDF from the LaTeX file so the \texttt{doconce subst} or
\texttt{doconce replace} commands can be put inside the script.

\emph{Step 3.} Compile \texttt{mydoc.tex}
and create the PDF file:
%
\begin{quote}{\ttfamily \raggedright \noindent
Terminal>~latex~mydoc\\
Terminal>~latex~mydoc\\
Terminal>~makeindex~mydoc~~~\#~if~index\\
Terminal>~bibitem~mydoc~~~~~\#~if~bibliography\\
Terminal>~latex~mydoc\\
Terminal>~dvipdf~mydoc
}
\end{quote}

If one wishes to run \texttt{ptex2tex} and use the minted LaTeX package for
typesetting code blocks (\texttt{Minted\_Python}, \texttt{Minted\_Cpp}, etc., in
\texttt{ptex2tex} specified through the \texttt{*pro} and \texttt{*cod} variables in
\texttt{.ptex2tex.cfg} or \texttt{\$HOME/.ptex2tex.cfg}), the minted LaTeX package is
needed.  This package is included by running \texttt{ptex2tex} with the
\texttt{-DMINTED} option:
%
\begin{quote}{\ttfamily \raggedright \noindent
Terminal>~ptex2tex~-DMINTED~mydoc
}
\end{quote}

In this case, \texttt{latex} must be run with the
\texttt{-shell-escape} option:
%
\begin{quote}{\ttfamily \raggedright \noindent
Terminal>~latex~-shell-escape~mydoc\\
Terminal>~latex~-shell-escape~mydoc\\
Terminal>~makeindex~mydoc~~~\#~if~index\\
Terminal>~bibitem~mydoc~~~~~\#~if~bibliography\\
Terminal>~latex~-shell-escape~mydoc\\
Terminal>~dvipdf~mydoc
}
\end{quote}

When running \texttt{doconce ptex2tex mydoc envir=minted} (or other minted
specifications with \texttt{doconce ptex2tex}), the minted package is automatically
included so there is no need for the \texttt{-DMINTED} option.


%___________________________________________________________________________

\subsection*{\phantomsection%
  PDFLaTeX%
  \addcontentsline{toc}{subsection}{PDFLaTeX}%
  \label{pdflatex}%
}

Running \texttt{pdflatex} instead of \texttt{latex} follows almost the same steps,
but the start is:
%
\begin{quote}{\ttfamily \raggedright \noindent
Terminal>~doconce~format~latex~mydoc
}
\end{quote}

Then \texttt{ptex2tex} is run as explained above, and finally:
%
\begin{quote}{\ttfamily \raggedright \noindent
Terminal>~pdflatex~-shell-escape~mydoc\\
Terminal>~makeindex~mydoc~~~\#~if~index\\
Terminal>~bibitem~mydoc~~~~~\#~if~bibliography\\
Terminal>~pdflatex~-shell-escape~mydoc
}
\end{quote}


%___________________________________________________________________________

\subsection*{\phantomsection%
  XeLaTeX%
  \addcontentsline{toc}{subsection}{XeLaTeX}%
  \label{xelatex}%
}

XeLaTeX is an alternative to pdfLaTeX and is run in almost the
same way, except for the \texttt{-DXELATEX} flag to ptex2tex:
%
\begin{quote}{\ttfamily \raggedright \noindent
Terminal>~doconce~format~pdflatex~mydoc\\
Terminal>~doconce~ptex2tex~mydoc~-DXELATEX\\
Terminal>~ptex2tex~-DXELATEX~mydoc~~\#~alternative\\
Terminal>~xelatex~mydoc
}
\end{quote}


%___________________________________________________________________________

\subsection*{\phantomsection%
  Plain ASCII Text%
  \addcontentsline{toc}{subsection}{Plain ASCII Text}%
  \label{plain-ascii-text}%
}

We can go from Doconce ``back to'' plain untagged text suitable for viewing
in terminal windows, inclusion in email text, or for insertion in
computer source code:
%
\begin{quote}{\ttfamily \raggedright \noindent
Terminal>~doconce~format~plain~mydoc.do.txt~~\#~results~in~mydoc.txt
}
\end{quote}


%___________________________________________________________________________

\subsection*{\phantomsection%
  reStructuredText%
  \addcontentsline{toc}{subsection}{reStructuredText}%
  \label{id5}%
}

Going from Doconce to reStructuredText gives a lot of possibilities to
go to other formats. First we filter the Doconce text to a
reStructuredText file \texttt{mydoc.rst}:
%
\begin{quote}{\ttfamily \raggedright \noindent
Terminal>~doconce~format~rst~mydoc.do.txt
}
\end{quote}

We may now produce various other formats:
%
\begin{quote}{\ttfamily \raggedright \noindent
Terminal>~rst2html.py~~mydoc.rst~>~mydoc.html~\#~html\\
Terminal>~rst2latex.py~mydoc.rst~>~mydoc.tex~~\#~latex\\
Terminal>~rst2xml.py~~~mydoc.rst~>~mydoc.xml~~\#~XML\\
Terminal>~rst2odt.py~~~mydoc.rst~>~mydoc.odt~~\#~OpenOffice
}
\end{quote}

The OpenOffice file \texttt{mydoc.odt} can be loaded into OpenOffice and
saved in, among other things, the RTF format or the Microsoft Word format.
However, it is more convenient to use the program \texttt{unovonv}
to convert between the many formats OpenOffice supports \emph{on the command line}.
Run:
%
\begin{quote}{\ttfamily \raggedright \noindent
Terminal>~unoconv~-{}-show
}
\end{quote}

to see all the formats that are supported.
For example, the following commands take
\texttt{mydoc.odt} to Microsoft Office Open XML format,
classic MS Word format, and PDF:
%
\begin{quote}{\ttfamily \raggedright \noindent
Terminal>~unoconv~-f~ooxml~mydoc.odt\\
Terminal>~unoconv~-f~doc~mydoc.odt\\
Terminal>~unoconv~-f~pdf~mydoc.odt
}
\end{quote}

\emph{Remark about Mathematical Typesetting.} At the time of this writing, there is no easy way to go from Doconce
and LaTeX mathematics to reST and further to OpenOffice and the
``MS Word world''. Mathematics is only fully supported by \texttt{latex} as
output and to a wide extent also supported by the \texttt{sphinx} output format.
Some links for going from LaTeX to Word are listed below.
%
\begin{quote}
%
\begin{itemize}

\item \url{http://ubuntuforums.org/showthread.php?t=1033441}

\item \url{http://tug.org/utilities/texconv/textopc.html}

\item \url{http://nileshbansal.blogspot.com/2007/12/latex-to-openofficeword.html}

\end{itemize}

\end{quote}


%___________________________________________________________________________

\subsection*{\phantomsection%
  Sphinx%
  \addcontentsline{toc}{subsection}{Sphinx}%
  \label{id6}%
}

Sphinx documents demand quite some steps in their creation. We have automated
most of the steps through the \texttt{doconce sphinx\_dir} command:
%
\begin{quote}{\ttfamily \raggedright \noindent
Terminal>~doconce~sphinx\_dir~author="authors'~names"~\textbackslash{}\\
~~~~~~~~~~title="some~title"~version=1.0~dirname=sphinxdir~\textbackslash{}\\
~~~~~~~~~~theme=mytheme~file1~file2~file3~...
}
\end{quote}

The keywords \texttt{author}, \texttt{title}, and \texttt{version} are used in the headings
of the Sphinx document. By default, \texttt{version} is 1.0 and the script
will try to deduce authors and title from the doconce files \texttt{file1},
\texttt{file2}, etc. that together represent the whole document. Note that
none of the individual Doconce files \texttt{file1}, \texttt{file2}, etc. should
include the rest as their union makes up the whole document.
The default value of \texttt{dirname} is \texttt{sphinx-rootdir}. The \texttt{theme}
keyword is used to set the theme for design of HTML output from
Sphinx (the default theme is \texttt{'default'}).

With a single-file document in \texttt{mydoc.do.txt} one often just runs:
%
\begin{quote}{\ttfamily \raggedright \noindent
Terminal>~doconce~sphinx\_dir~mydoc
}
\end{quote}

and then an appropriate Sphinx directory \texttt{sphinx-rootdir} is made with
relevant files.

The \texttt{doconce sphinx\_dir} command generates a script
\texttt{automake\_sphinx.py} for compiling the Sphinx document into an HTML
document.  One can either run \texttt{automake\_sphinx.py} or perform the
steps in the script manually, possibly with necessary modifications.
Normally, executing the script works well, but if you are new
to Sphinx and end up producing quite some Sphinx documents, I encourave
you to read the Sphinx documentation and study the \texttt{automake\_sphinx.py}
file.

\emph{Links.} The \texttt{automake\_sphinx.py} script copies directories named \texttt{fig*}
over to the Sphinx directory so that figures are accessible
in the Sphinx compilation.  It also examines \texttt{MOVIE:} and \texttt{FIGURE:}
commands in the Doconce file to find other image files and copies
these too. I strongly recommend to put files
to which there are local links (not \texttt{http:} or \texttt{file:} URLs) in
a directory named \texttt{\_static}. The \texttt{automake\_sphinx.py} copies
\texttt{\_static*} to the Sphinx directory, which guarantees that the links
to the local files will work in the Sphinx document.

There is a utility \texttt{doconce sphinxfix\_localURLs} for checking links to
local files and moving the files to \texttt{\_static} and changing the links
accordingly. For example, a link to \texttt{dir1/dir2/myfile.txt} is changed
to \texttt{\_static/myfile.txt} and \texttt{myfile.txt} is copied to \texttt{\_static}.
However, I recommend instead that you manually copy
files to \texttt{\_static} when you want to link to them, or let your
script which compiles the Doconce document do it automatically.

\emph{Themes.} Doconce comes with a rich collection of HTML themes for Sphinx documents,
much larger than what is found in the standard Sphinx distribution.
Additional themes include
\texttt{agni},
\texttt{basicstrap},
\texttt{bootstrap},
\texttt{cloud},
\texttt{fenics},
\texttt{fenics\_minimal},
\texttt{flask},
\texttt{haiku},
\texttt{impressjs},
\texttt{jal},
\texttt{pylons},
\texttt{redcloud},
\texttt{scipy\_lectures},
\texttt{slim-agogo}, and
\texttt{vlinux-theme}.

All the themes are packed out in the Sphinx directory, and the
\texttt{doconce sphinx\_dir} insert lots of extra code in the \texttt{conf.py}
file to enable easy specification and customization of themes.
For example, modules are loaded for the additional themes that
come with Doconce, code is inserted to allow customization of
the look and feel of themes, etc. The \texttt{conf.py} file is a
good starting point for fine-tuning your favorite team, and your
own \texttt{conf.py} file can later be supplied and used when running
\texttt{doconce sphinx\_dir}: simply add the command-line option
\texttt{conf.py=conf.py}.

A script
\texttt{make-themes.sh} can make HTML documents with one or more themes.
For example,
to realize the themes \texttt{fenics}, \texttt{pyramid}, and \texttt{pylon} one writes:
%
\begin{quote}{\ttfamily \raggedright \noindent
Terminal>~./make-themes.sh~fenics~pyramid~pylon
}
\end{quote}

The resulting directories with HTML documents are \texttt{\_build/html\_fenics}
and \texttt{\_build/html\_pyramid}, respectively. Without arguments,
\texttt{make-themes.sh} makes all available themes (!). With \texttt{make-themes.sh}
it is easy to check out various themes to find the one that is most
attractive for your document.

You may supply your own theme and avoid copying all the themes
that come with Doconce into the Sphinx directory. Just specify
\texttt{theme\_dir=path} on the command line, where \texttt{path} is the relative
path to the directory containing the Sphinx theme. You must also
specify a configure file by \texttt{conf.py=path}, where \texttt{path} is the
relative path to your \texttt{conf.py} file.

\emph{Example.} Say you like the \texttt{scipy\_lectures} theme, but you want
a table of contents to appear \emph{to the right}, much in the same style
as in the \texttt{default} theme (where the table of contents is to the left).
You can then run \texttt{doconce sphinx\_dir}, invoke a text editor with the
\texttt{conf.py} file, find the line \texttt{html\_theme == 'scipy\_lectures'},
edit the following \texttt{nosidebar} to \texttt{false} and \texttt{rightsidebar} to \texttt{true}.
Alternatively, you may write a little script using \texttt{doconce replace}
to replace a portion of text in \texttt{conf.py} by a new one:
%
\begin{quote}{\ttfamily \raggedright \noindent
doconce~replace~"elif~html\_theme~==~'scipy\_lectures':\\
~~~~html\_theme\_options~=~\{\\
~~~~~~~~'nosidebar':~'true',\\
~~~~~~~~'rightsidebar':~'false',\\
~~~~~~~~'sidebarbgcolor':~'\#f2f2f2',\\
~~~~~~~~'sidebartextcolor':~'\#20435c',\\
~~~~~~~~'sidebarlinkcolor':~'\#20435c',\\
~~~~~~~~'footerbgcolor':~'\#000000',\\
~~~~~~~~'relbarbgcolor':~'\#000000',\\
~~~~\}"~"elif~html\_theme~==~'scipy\_lectures':\\
~~~~html\_theme\_options~=~\{\\
~~~~~~~~'nosidebar':~'false',\\
~~~~~~~~'rightsidebar':~'true',\\
~~~~~~~~'sidebarbgcolor':~'\#f2f2f2',\\
~~~~~~~~'sidebartextcolor':~'\#20435c',\\
~~~~~~~~'sidebarlinkcolor':~'\#20435c',\\
~~~~~~~~'footerbgcolor':~'\#000000',\\
~~~~~~~~'relbarbgcolor':~'\#000000',\\
~~~~\}"~conf.py
}
\end{quote}

Obviously, we could also have changed colors in the edit above.
The final alternative is to save the edited \texttt{conf.py} file somewhere
and reuse it the next time \texttt{doconce sphinx\_dir} is run:
%
\begin{quote}{\ttfamily \raggedright \noindent
doconce~sphinx\_dir~theme=scipy\_lectures~\textbackslash{}\\
~~~~~~~~~~~~~~~~~~~conf.py=../some/path/conf.py~mydoc
}
\end{quote}


%___________________________________________________________________________

\subsubsection*{\phantomsection%
  The manual Sphinx procedure%
  \addcontentsline{toc}{subsubsection}{The manual Sphinx procedure}%
  \label{the-manual-sphinx-procedure}%
}

If it is not desirable to use the autogenerated scripts explained
above, here is the complete manual procedure of generating a
Sphinx document from a file \texttt{mydoc.do.txt}.

\emph{Step 1.} Translate Doconce into the Sphinx format:
%
\begin{quote}{\ttfamily \raggedright \noindent
Terminal>~doconce~format~sphinx~mydoc
}
\end{quote}

\emph{Step 2.} Create a Sphinx root directory
either manually or by using the interactive \texttt{sphinx-quickstart}
program. Here is a scripted version of the steps with the latter:
%
\begin{quote}{\ttfamily \raggedright \noindent
mkdir~sphinx-rootdir\\
sphinx-quickstart~<{}<EOF\\
sphinx-rootdir\\
n\\
\_\\
Name~of~My~Sphinx~Document\\
Author\\
version\\
version\\
.rst\\
index\\
n\\
y\\
n\\
n\\
n\\
n\\
y\\
n\\
n\\
y\\
y\\
y\\
EOF
}
\end{quote}

The autogenerated \texttt{conf.py} file
may need some edits if you want to specific layout (Sphinx themes)
of HTML pages. The \texttt{doconce sphinx\_dir} generator makes an extended \texttt{conv.py}
file where, among other things, several useful Sphinx extensions
are included.

\emph{Step 3.} Copy the \texttt{mydoc.rst} file to the Sphinx root directory:
%
\begin{quote}{\ttfamily \raggedright \noindent
Terminal>~cp~mydoc.rst~sphinx-rootdir
}
\end{quote}

If you have figures in your document, the relative paths to those will
be invalid when you work with \texttt{mydoc.rst} in the \texttt{sphinx-rootdir}
directory. Either edit \texttt{mydoc.rst} so that figure file paths are correct,
or simply copy your figure directories to \texttt{sphinx-rootdir}.
Links to local files in \texttt{mydoc.rst} must be modified to links to
files in the \texttt{\_static} directory, see comment above.

\emph{Step 4.} Edit the generated \texttt{index.rst} file so that \texttt{mydoc.rst}
is included, i.e., add \texttt{mydoc} to the \texttt{toctree} section so that it becomes:
%
\begin{quote}{\ttfamily \raggedright \noindent
..~toctree::\\
~~~:maxdepth:~2\\
~\\
~~~mydoc
}
\end{quote}

(The spaces before \texttt{mydoc} are important!)

\emph{Step 5.} Generate, for instance, an HTML version of the Sphinx source:
%
\begin{quote}{\ttfamily \raggedright \noindent
make~clean~~~\#~remove~old~versions\\
make~html
}
\end{quote}

Sphinx can generate a range of different formats:
standalone HTML, HTML in separate directories with \texttt{index.html} files,
a large single HTML file, JSON files, various help files (the qthelp, HTML,
and Devhelp projects), epub, LaTeX, PDF (via LaTeX), pure text, man pages,
and Texinfo files.

\emph{Step 6.} View the result:
%
\begin{quote}{\ttfamily \raggedright \noindent
Terminal>~firefox~\_build/html/index.html
}
\end{quote}

Note that verbatim code blocks can be typeset in a variety of ways
depending the argument that follows \texttt{!bc}: \texttt{cod} gives Python
(\texttt{code-block:: python} in Sphinx syntax) and \texttt{cppcod} gives C++, but
all such arguments can be customized both for Sphinx and LaTeX output.


%___________________________________________________________________________

\subsection*{\phantomsection%
  Wiki Formats%
  \addcontentsline{toc}{subsection}{Wiki Formats}%
  \label{wiki-formats}%
}

\DUadmonition[system-message]{
\DUtitle[system-message]{system-message}


{\color{red}WARNING/2} in \texttt{tutorial.rst}, line~429

\hyperlink{id7}{
Duplicate explicit target name: ``mediawiki''.
}}

There are many different wiki formats, but Doconce only supports three:
\href{http://code.google.com/p/support/wiki/WikiSyntax}{Googlecode wiki},
\href{http://www.mediawiki.org/wiki/Help:Formatting}{MediaWiki}, and
\href{http://www.wikicreole.org/wiki/Creole1.0}{Creole Wiki}.
These formats are called
\texttt{gwiki}, \texttt{mwiki}, and \texttt{cwiki}, respectively.
Transformation from Doconce to these formats is done by:
%
\begin{quote}{\ttfamily \raggedright \noindent
Terminal>~doconce~format~gwiki~mydoc.do.txt\\
Terminal>~doconce~format~mwiki~mydoc.do.txt\\
Terminal>~doconce~format~cwiki~mydoc.do.txt
}
\end{quote}

The produced MediaWiki can be tested in the \href{http://en.wikibooks.org/wiki/Sandbox}{sandbox of
wikibooks.org}. The format
works well with Wikipedia, Wikibooks, and
\href{http://doconcedemo.shoutwiki.com/wiki/Doconce_demo_page}{ShoutWiki},
but not always well elsewhere
(see \href{http://doconcedemo.jumpwiki.com/wiki/First_demo}{this example}).

Large MediaWiki documents can be made with the
\href{http://en.wikipedia.org/w/index.php?title=Special:Book&bookcmd=book_creator}{Book creator}.
From the MediaWiki format one can go to other formats with aid
of \href{http://pediapress.com/code/}{mwlib}. This means that one can
easily use Doconce to write \href{http://en.wikibooks.org}{Wikibooks}
and publish these in PDF and MediaWiki format, while
at the same time, the book can also be published as a
standard LaTeX book, a Sphinx web document, or a collection of HTML files.

The Googlecode wiki document, \texttt{mydoc.gwiki}, is most conveniently stored
in a directory which is a clone of the wiki part of the Googlecode project.
This is far easier than copying and pasting the entire text into the
wiki editor in a web browser.

When the Doconce file contains figures, each figure filename must in
the \texttt{.gwiki} file be replaced by a URL where the figure is
available. There are instructions in the file for doing this. Usually,
one performs this substitution automatically (see next section).


%___________________________________________________________________________

\subsection*{\phantomsection%
  Tweaking the Doconce Output%
  \addcontentsline{toc}{subsection}{Tweaking the Doconce Output}%
  \label{tweaking-the-doconce-output}%
}

Occasionally, one would like to tweak the output in a certain format
from Doconce. One example is figure filenames when transforming
Doconce to reStructuredText. Since Doconce does not know if the
\texttt{.rst} file is going to be filtered to LaTeX or HTML, it cannot know
if \texttt{.eps} or \texttt{.png} is the most appropriate image filename.
The solution is to use a text substitution command or code with, e.g., sed,
perl, python, or scitools subst, to automatically edit the output file
from Doconce. It is then wise to run Doconce and the editing commands
from a script to automate all steps in going from Doconce to the final
format(s). The \texttt{make.sh} files in \texttt{docs/manual} and \texttt{docs/tutorial}
constitute comprehensive examples on how such scripts can be made.


%___________________________________________________________________________

\subsection*{\phantomsection%
  Demos%
  \addcontentsline{toc}{subsection}{Demos}%
  \label{demos}%
}

The current text is generated from a Doconce format stored in the file:
%
\begin{quote}{\ttfamily \raggedright \noindent
doc/tutorial/tutorial.do.txt
}
\end{quote}

The file \texttt{make.sh} in the \texttt{tutorial} directory of the
Doconce source code contains a demo of how to produce a variety of
formats.  The source of this tutorial, \texttt{tutorial.do.txt} is the
starting point.  Running \texttt{make.sh} and studying the various generated
files and comparing them with the original \texttt{tutorial.do.txt} file,
gives a quick introduction to how Doconce is used in a real case.
\href{https://doconce.googlecode.com/hg/doc/demos/tutorial/index.html}{Here}
is a sample of how this tutorial looks in different formats.

There is another demo in the \texttt{docs/manual} directory which
translates the more comprehensive documentation, \texttt{manual.do.txt}, to
various formats. The \texttt{make.sh} script runs a set of translations.


%___________________________________________________________________________

\section*{\phantomsection%
  Installation of Doconce and its Dependencies%
  \addcontentsline{toc}{section}{Installation of Doconce and its Dependencies}%
  \label{installation-of-doconce-and-its-dependencies}%
}

Below, we explain the manual installation of all software that may be
needed when working with Doconce documents.
The impatient way to install what is needed is to run the
\url{doconce_install_all.sh} script.


%___________________________________________________________________________

\subsection*{\phantomsection%
  Doconce%
  \addcontentsline{toc}{subsection}{Doconce}%
  \label{doconce}%
}

Doconce itself is pure Python code hosted at \url{http://code.google.com/p/doconce}.  Its installation from the
Mercurial (\texttt{hg}) source follows the standard procedure:
%
\begin{quote}{\ttfamily \raggedright \noindent
\#~Doconce\\
hg~clone~https://code.google.com/p/doconce/~doconce\\
cd~doconce\\
sudo~python~setup.py~install\\
cd~..
}
\end{quote}

Since Doconce is frequently updated, it is recommended to use the
above procedure and whenever a problem occurs, make sure to
update to the most recent version:
%
\begin{quote}{\ttfamily \raggedright \noindent
cd~doconce\\
hg~pull\\
hg~update\\
sudo~python~setup.py~install
}
\end{quote}


%___________________________________________________________________________

\subsection*{\phantomsection%
  Dependencies%
  \addcontentsline{toc}{subsection}{Dependencies}%
  \label{dependencies}%
}

Producing HTML documents, plain text, pandoc-extended Markdown,
and wikis can be done without installing any other
software. However, if you want other formats as output
(LaTeX, Sphinx, reStructuredText) and assisting utilities such
as preprocesors, spellcheck, file differences, bibliographies,
and so on, the software below must be installed.

% Make a debpkg_doconce.txt file with everything that is needed on Debian


%___________________________________________________________________________

\subsubsection*{\phantomsection%
  Preprocessors%
  \addcontentsline{toc}{subsubsection}{Preprocessors}%
  \label{preprocessors}%
}

If you make use of the \href{http://code.google.com/p/preprocess}{Preprocess}
preprocessor, this program must be installed:
%
\begin{quote}{\ttfamily \raggedright \noindent
svn~checkout~http://preprocess.googlecode.com/svn/trunk/~preprocess\\
cd~preprocess\\
cd~doconce\\
sudo~python~setup.py~install\\
cd~..
}
\end{quote}

A much more advanced alternative to Preprocess is
\href{http://www.makotemplates.org}{Mako}. Its installation is most
conveniently done by \texttt{pip}:
%
\begin{quote}{\ttfamily \raggedright \noindent
pip~install~Mako
}
\end{quote}

This command requires \texttt{pip} to be installed. On Debian Linux systems,
such as Ubuntu, the installation is simply done by:
%
\begin{quote}{\ttfamily \raggedright \noindent
sudo~apt-get~install~python-pip
}
\end{quote}

Alternatively, one can install from the \texttt{pip} \href{http://pypi.python.org/pypi/pip}{source code}.

Mako can also be installed directly from
\href{http://www.makotemplates.org/download.html}{source}: download the
tarball, pack it out, go to the directory and run
the usual \texttt{sudo python setup.py install}.


%___________________________________________________________________________

\subsubsection*{\phantomsection%
  Image file handling%
  \addcontentsline{toc}{subsubsection}{Image file handling}%
  \label{image-file-handling}%
}

Different output formats require different formats of image files.
For example, PostScript or Encapuslated PostScript is required for \texttt{latex}
output, while HTML needs JPEG, GIF, or PNG formats.
Doconce calls up programs from the ImageMagick suite for converting
image files to a proper format if needed. The \href{http://www.imagemagick.org/script/index.php}{ImageMagick suite} can be installed on all major platforms.
On Debian Linux (including Ubuntu) systems one can simply write:
%
\begin{quote}{\ttfamily \raggedright \noindent
sudo~apt-get~install~imagemagick
}
\end{quote}

The convenience program \texttt{doconce combine\_images}, for combining several
images into one, will use \texttt{montage} and \texttt{convert} from ImageMagick and
the \texttt{pdftk}, \texttt{pdfnup}, and \texttt{pdfcrop} programs from the \texttt{texlive-extra-utils}
Debian package. The latter gets installed by:
%
\begin{quote}{\ttfamily \raggedright \noindent
sudo~apt-get~install~texlive-extra-utils
}
\end{quote}

Automatic image conversion from EPS to PDF calls up \texttt{epstopdf}, which
can be installed by:
%
\begin{quote}{\ttfamily \raggedright \noindent
sudo~apt-get~install~texlive-font-utils
}
\end{quote}


%___________________________________________________________________________

\subsubsection*{\phantomsection%
  Spellcheck%
  \addcontentsline{toc}{subsubsection}{Spellcheck}%
  \label{spellcheck}%
}

The utility \texttt{doconce spellcheck} applies the \texttt{ispell} program for
spellcheck. On Debian (including Ubuntu) it is installed by:
%
\begin{quote}{\ttfamily \raggedright \noindent
sudo~apt-get~install~ispell
}
\end{quote}


%___________________________________________________________________________

\subsubsection*{\phantomsection%
  Bibliography%
  \addcontentsline{toc}{subsubsection}{Bibliography}%
  \label{bibliography}%
}

The Python package \href{https://bitbucket.org/logg/publish}{Publish} is needed if you use a bibliography
in your document. On the website, click on \emph{Clone}, copy the
command and run it:
%
\begin{quote}{\ttfamily \raggedright \noindent
hg~clone~https://bitbucket.org/logg/publish
}
\end{quote}

Thereafter go to the \texttt{publish} directory and run the \texttt{setup.py} script
for installing Publish:
%
\begin{quote}{\ttfamily \raggedright \noindent
cd~publish\\
sudo~python~setup.py
}
\end{quote}


%___________________________________________________________________________

\subsubsection*{\phantomsection%
  Ptex2tex for LaTeX Output%
  \addcontentsline{toc}{subsubsection}{Ptex2tex for LaTeX Output}%
  \label{ptex2tex-for-latex-output}%
}

To make LaTeX documents with very flexible choice of typesetting of
verbatim code blocks you need \href{http://code.google.com/p/ptex2tex}{ptex2tex},
which is installed by:
%
\begin{quote}{\ttfamily \raggedright \noindent
svn~checkout~http://ptex2tex.googlecode.com/svn/trunk/~ptex2tex\\
cd~ptex2tex\\
sudo~python~setup.py~install
}
\end{quote}

It may happen that you need additional style files, you can run
a script, \texttt{cp2texmf.sh}:
%
\begin{quote}{\ttfamily \raggedright \noindent
cd~latex\\
sh~cp2texmf.sh~~\#~copy~stylefiles~to~\textasciitilde{}/texmf~directory\\
cd~../..
}
\end{quote}

This script copies some special stylefiles that
that \texttt{ptex2tex} potentially makes use of. Some more standard stylefiles
are also needed. These are installed by:
%
\begin{quote}{\ttfamily \raggedright \noindent
sudo~apt-get~install~texlive
}
\end{quote}

on Debian Linux (including Ubuntu) systems. TeXShop on Mac comes with
the necessary stylefiles (if not, they can be found by googling and installed
manually in the \texttt{\textasciitilde{}/texmf/tex/latex/misc} directory).

Note that the \texttt{doconce ptex2tex} command, which needs no installation
beyond Doconce itself, can be used as a simpler alternative to the \texttt{ptex2tex}
program.

The \emph{minted} LaTeX style is offered by \texttt{ptex2tex} and \texttt{doconce ptext2tex}
and popular among many
users. This style requires the package \href{http://pygments.org}{Pygments}
to be installed. On Debian Linux:
%
\begin{quote}{\ttfamily \raggedright \noindent
sudo~apt-get~install~python-pygments
}
\end{quote}

Alternatively, the package can be installed manually:
%
\begin{quote}{\ttfamily \raggedright \noindent
hg~clone~ssh://hg@bitbucket.org/birkenfeld/pygments-main~pygments\\
cd~pygments\\
sudo~python~setup.py~install
}
\end{quote}

One can also do the simple:
%
\begin{quote}{\ttfamily \raggedright \noindent
pip~install~sphinx
}
\end{quote}

which also installs pygments.

If you use the minted style together with \texttt{ptex2tex}, you have to
enable it by the \texttt{-DMINTED} command-line argument to \texttt{ptex2tex}.
This is not necessary if you run the alternative \texttt{doconce ptex2tex} program.

All use of the minted style requires the \texttt{-shell-escape} command-line
argument when running LaTeX, i.e., \texttt{latex -shell-escape} or \texttt{pdflatex
-shell-escape}.

Inline comments apply the \texttt{todonotes} LaTeX package if the \texttt{ptex2tex}
or \texttt{doconce ptex2tex} command is run with \texttt{-DTODONOTES}.  The
\texttt{todonotes} package requires several other packages: \texttt{xcolor},
\texttt{ifthen}, \texttt{xkeyval}, \texttt{tikz}, \texttt{calc}, \texttt{graphicx}, and \texttt{setspace}. The
relevant Debian packages for installing all this are listed below.


%___________________________________________________________________________

\subsubsection*{\phantomsection%
  LaTeX packages%
  \addcontentsline{toc}{subsubsection}{LaTeX packages}%
  \label{latex-packages}%
}

Many LaTeX packages are potentially needed (depending on various
preprocessor variables given to \texttt{ptex2tex} or \texttt{doconce ptex2tex}.  The
standard packages always included are \texttt{relsize}, \texttt{epsfig}, \texttt{makeidx},
\texttt{setspace}, \texttt{color}, \texttt{amsmath}, \texttt{amsfonts}, \texttt{xcolor}, \texttt{bm},
\texttt{microtype}, \texttt{titlesec}, and \texttt{hyperref}.  The \texttt{ptex2tex} package (from
\href{http://code.google.com/p/ptex2tex}{ptex2tex}) is also included, but
removed again if \texttt{doconce ptex2tex} is run instead of the \texttt{ptex2tex}
program, meaning that if you do not use \texttt{ptex2tex}, you do not need
\texttt{ptex2tex.sty}. Optional packages that might be included are \texttt{minted},
\texttt{fontspec}, \texttt{xunicode}, \texttt{inputenc}, \texttt{helvet}, \texttt{mathpazo}, \texttt{wrapfig},
\texttt{calc}, \texttt{ifthen}, \texttt{xkeyval}, \texttt{tikz}, \texttt{graphicx}, \texttt{setspace}, \texttt{shadow},
\texttt{disable}, \texttt{todonotes}, \texttt{lineno}, \texttt{xr}, \texttt{framed}, \texttt{mdframe},
\texttt{movie15}, \texttt{a4paper}, and \texttt{a6paper}.

Relevant Debian packages that gives you all of these LaTeX packages are:
%
\begin{quote}{\ttfamily \raggedright \noindent
texlive\\
texlive-extra-utils\\
texlive-latex-extra\\
texlive-font-utils
}
\end{quote}

On old Ubuntu 12.04 one has to do \texttt{sudo add-apt-repository ppa:texlive-backports/ppa} and \texttt{sudo apt-get update} first, or alternatively install these as well:
%
\begin{quote}{\ttfamily \raggedright \noindent
texlive-math-extra\\
texlive-bibtex-extra\\
texlive-xetex\\
texlive-humanities\\
texlive-pictures
}
\end{quote}

Alternatively, one may pull in \texttt{texlive-full} to get all available
style files.

If you want to use the \emph{anslistings} code environment with \texttt{ptex2tex}
(\texttt{.ptex2tex.cfg} styles \texttt{Python\_ANS}, \texttt{Python\_ANSt}, \texttt{Cpp\_ANS}, etc.) or
\texttt{doconce ptex2tex} (\texttt{envir=ans} or \texttt{envir=ans:nt}), you need the
\texttt{anslistings.sty} file. It can be obtained from
the \href{https://code.google.com/p/ptex2tex/source/browse/trunk/latex/styles/with_license/anslistings.sty}{ptex2tex source}. It should get installed
by the \texttt{cp2texmf.sh} script executed above.


%___________________________________________________________________________

\subsubsection*{\phantomsection%
  reStructuredText (reST) Output%
  \addcontentsline{toc}{subsubsection}{reStructuredText (reST) Output}%
  \label{restructuredtext-rest-output}%
}

The \texttt{rst} output from Doconce allows further transformation to LaTeX,
HTML, XML, OpenOffice, and so on, through the \href{http://docutils.sourceforge.net}{docutils} package.  The installation of the
most recent version can be done by:
%
\begin{quote}{\ttfamily \raggedright \noindent
svn~checkout~\textbackslash{}\\
~~http://docutils.svn.sourceforge.net/svnroot/docutils/trunk/docutils\\
cd~docutils\\
sudo~python~setup.py~install\\
cd~..
}
\end{quote}

The command:
%
\begin{quote}{\ttfamily \raggedright \noindent
pip~install~sphinx
}
\end{quote}

installs Docutils along with Sphinx and Pygments.

To use the OpenOffice suite you will typically on Debian systems install:
%
\begin{quote}{\ttfamily \raggedright \noindent
sudo~apt-get~install~unovonv~libreoffice~libreoffice-dmaths
}
\end{quote}

There is a possibility to create PDF files from reST documents
using ReportLab instead of LaTeX. The enabling software is
\href{http://code.google.com/p/rst2pdf}{rst2pdf}. Either download the tarball
or clone the svn repository, go to the \texttt{rst2pdf} directory and
run the usual \texttt{sudo python setup.py install}.


%___________________________________________________________________________

\subsubsection*{\phantomsection%
  Sphinx Output%
  \addcontentsline{toc}{subsubsection}{Sphinx Output}%
  \label{sphinx-output}%
}

Output to \texttt{sphinx} requires of course the
\href{http://sphinx.pocoo.org}{Sphinx software},
installed by:
%
\begin{quote}{\ttfamily \raggedright \noindent
hg~clone~https://bitbucket.org/birkenfeld/sphinx\\
cd~sphinx\\
sudo~python~setup.py~install\\
cd~..
}
\end{quote}

An alternative is:
%
\begin{quote}{\ttfamily \raggedright \noindent
pip~install~sphinx
}
\end{quote}

Doconce comes with many Sphinx themes that are not part of the
standard Sphinx source distribution. Some of these themes require
additional Python/Sphinx modules to be installed:
%
\begin{quote}
%
\begin{itemize}

\item cloud and redcloud: \url{https://bitbucket.org/ecollins/cloud_sptheme}

\item bootstrap: \url{https://github.com/ryan-roemer/sphinx-bootstrap-theme}

\item solarized: \url{https://bitbucket.org/miiton/sphinxjp.themes.solarized}

\item impressjs: \url{https://github.com/shkumagai/sphinxjp.themes.impressjs}

\end{itemize}

\end{quote}

These must be downloaded or cloned, and \texttt{setup.py} must be run as shown
above.


%___________________________________________________________________________

\subsubsection*{\phantomsection%
  Markdown and Pandoc Output%
  \addcontentsline{toc}{subsubsection}{Markdown and Pandoc Output}%
  \label{markdown-and-pandoc-output}%
}

The Doconce format \texttt{pandoc} outputs the document in the Pandoc
extended Markdown format, which via the \texttt{pandoc} program can be
translated to a range of other formats. Installation of \href{http://johnmacfarlane.net/pandoc/}{Pandoc}, written in Haskell, is most
easily done by:
%
\begin{quote}{\ttfamily \raggedright \noindent
sudo~apt-get~install~pandoc
}
\end{quote}

on Debian (Ubuntu) systems.


%___________________________________________________________________________

\subsubsection*{\phantomsection%
  Epydoc Output%
  \addcontentsline{toc}{subsubsection}{Epydoc Output}%
  \label{epydoc-output}%
}

When the output format is \texttt{epydoc} one needs that program too, installed
by:
%
\begin{quote}{\ttfamily \raggedright \noindent
svn~co~https://epydoc.svn.sourceforge.net/svnroot/epydoc/trunk/epydoc~epydoc\\
cd~epydoc\\
sudo~make~install\\
cd~..
}
\end{quote}

\emph{Remark.} Several of the packages above installed from source code
are also available in Debian-based system through the
\texttt{apt-get install} command. However, we recommend installation directly
from the version control system repository as there might be important
updates and bug fixes. For \texttt{svn} directories, go to the directory,
run \texttt{svn update}, and then \texttt{sudo python setup.py install}. For
Mercurial (\texttt{hg}) directories, go to the directory, run
\texttt{hg pull; hg update}, and then \texttt{sudo python setup.py install}.


%___________________________________________________________________________

\subsubsection*{\phantomsection%
  The \texttt{doconce diff} command%
  \addcontentsline{toc}{subsubsection}{The doconce diff command}%
  \label{the-doconce-diff-command}%
}

The \texttt{doconce diff file1 file prog} command for illustrating differences between
two files \texttt{file1} and \texttt{file2} using the program \texttt{prog} requires \texttt{prog}
to be installed. By default, \texttt{prog} is \texttt{difflib} which comes with Python
and is always present if you have Doconce installed. Another choice, \texttt{diff},
should be available on all Unix/Linux systems. Other choices, their
URL, and their \texttt{sudo apt-get install} command on Debian (Ubuntu) systems
appear in the table below.

\setlength{\DUtablewidth}{\linewidth}
\begin{longtable*}[c]{|p{0.315\DUtablewidth}|p{0.315\DUtablewidth}|p{0.315\DUtablewidth}|}
\hline
\textbf{%
Program
} & \textbf{%
URL
} & \textbf{%
Debian/Ubuntu install
} \\
\hline
\endfirsthead
\hline
\textbf{%
Program
} & \textbf{%
URL
} & \textbf{%
Debian/Ubuntu install
} \\
\hline
\endhead
\multicolumn{3}{c}{\hfill ... continued on next page} \\
\endfoot
\endlastfoot

\texttt{pdiff}
 & 
\href{http://www.gnu.org/software/a2ps/}{a2ps} \href{http://www.gnu.org/software/wdiff/}{wdiff}
 & 
\texttt{sudo apt-get install a2ps wdiff texlive-latex-extra texlive-latex-recommended}
 \\
\hline

\texttt{latexdiff}
 & 
\href{http://www.ctan.org/pkg/latexdiff}{latexdiff}
 & 
\texttt{sudo apt-get install latexdiff}
 \\
\hline

\texttt{kdiff3}
 & 
\href{http://kdiff3.sourceforge.net/}{kdiff3}
 & 
\texttt{sudo apt-get install kdiff3}
 \\
\hline

\texttt{diffuse}
 & 
\href{http://diffuse.sourceforge.net/}{diffuse}
 & 
\texttt{sudo apt-get install diffuse}
 \\
\hline

\texttt{xxdiff}
 & 
\href{http://xxdiff.sourceforge.net/local/}{xxdiff}
 & 
\texttt{sudo apt-get install xxdiff}
 \\
\hline

\texttt{meld}
 & 
\href{http://meldmerge.org/}{meld}
 & 
\texttt{sudo apt-get install meld}
 \\
\hline

\texttt{tkdiff.tcl}
 & 
\href{https://sourceforge.net/projects/tkdiff/}{tkdiff}
 & 
not in Debian
 \\
\hline
\end{longtable*}


%___________________________________________________________________________

\subsection*{\phantomsection%
  Quick Debian/Ubuntu Install%
  \addcontentsline{toc}{subsection}{Quick Debian/Ubuntu Install}%
  \label{quick-debian-ubuntu-install}%
}

On Debian (including Ubuntu) systems, it is straightforward to install the
long series of Doconce dependencies:
%
\begin{quote}{\ttfamily \raggedright \noindent
\#~Version~control~systems\\
sudo~apt-get~install~-y~mercurial~git~subversion\\
~\\
\#~Python\\
sudo~apt-get~install~-y~idle~ipython~python-pip~python-pdftools~texinfo\\
~\\
\#~These~lines~are~only~necessary~for~Ubuntu~12.04~to~install~texlive~2012\\
ubuntu\_version=`lsb\_release~-r~|~awl~'\{print~\$2\}'`\\
if~{[}~\$ubuntu\_version~=~"12.04"~{]};~then\\
~~sudo~add-apt-repository~ppa:texlive-backports/ppa\\
~~sudo~apt-get~update\\
fi\\
\#~LaTeX\\
sudo~apt-get~install~-y~texlive~texlive-extra-utils~texlive-latex-extra~texlive-math-extra~texlive-font-utils\\
\#~or~sudo~apt-get~install~-y~texlive-full~~\#~get~everything\\
sudo~apt-get~install~-y~latexdiff~auctex\\
~\\
\#~Image~and~movie~tools\\
sudo~apt-get~install~-y~imagemagick~netpbm~mjpegtools~pdftk~giftrans~gv~evince~smpeg-plaympeg~mplayer~totem~ffmpeg~libav-tools\\
~\\
\#~Misc\\
sudo~apt-get~install~-y~ispell~pandoc~libreoffice~unoconv~libreoffice-dmaths~curl~a2ps~wdiff~meld~xxdiff~diffpdf~kdiff3~diffuse\\
~\\
\#~More~Python~software\\
pip~install~sphinx~~\#~install~pygments~and~docutils~too\\
pip~install~mako\\
pip~install~-e~svn+http://preprocess.googlecode.com/svn/trunk\#egg=preprocess\\
pip~install~-e~hg+https://bitbucket.org/logg/publish\#egg=publish\\
~\\
pip~install~-e~hg+https://bitbucket.org/ecollins/cloud\_sptheme\#egg=cloud\_sptheme\\
pip~install~-e~git+https://github.com/ryan-roemer/sphinx-bootstrap-theme\#egg=sphinx-bootstrap-theme\\
pip~install~-e~hg+https://bitbucket.org/miiton/sphinxjp.themes.solarized\#egg=sphinxjp.themes.solarized\\
pip~install~-e~git+https://github.com/shkumagai/sphinxjp.themes.impressjs\#egg=sphinxjp.themes.impressjs\\
pip~install~-e~svn+https://epydoc.svn.sourceforge.net/svnroot/epydoc/trunk/epydoc\#egg=epydoc\\
~\\
\#~Doconce~itself\\
rm~-rf~srclib~~~\#~put~downloaded~software~in~srclib\\
mkdir~srclib\\
cd~srclib\\
hg~clone~https://code.google.com/p/doconce/\\
cd~doconce\\
sudo~python~setup.py~install~-y\\
cd~../..\\
~\\
\#~Ptex2tex\\
cd~srclib\\
svn~checkout~http://ptex2tex.googlecode.com/svn/trunk/~ptex2tex\\
cd~ptex2tex\\
sudo~python~setup.py~install~-y\\
cd~latex\\
sh~cp2texmf.sh~~\#~copy~stylefiles~to~\textasciitilde{}/texmf~directory\\
cd~../../..
}
\end{quote}

\end{document}
