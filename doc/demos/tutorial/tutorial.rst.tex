\documentclass[a4paper]{article}
% generated by Docutils <http://docutils.sourceforge.net/>
\usepackage{fixltx2e} % LaTeX patches, \textsubscript
\usepackage{cmap} % fix search and cut-and-paste in Acrobat
\usepackage{ifthen}
\usepackage[T1]{fontenc}
\usepackage[utf8]{inputenc}
\usepackage{longtable,ltcaption,array}
\setlength{\extrarowheight}{2pt}
\newlength{\DUtablewidth} % internal use in tables

%%% Custom LaTeX preamble
% PDF Standard Fonts
\usepackage{mathptmx} % Times
\usepackage[scaled=.90]{helvet}
\usepackage{courier}

%%% User specified packages and stylesheets

%%% Fallback definitions for Docutils-specific commands

% admonition (specially marked topic)
\providecommand{\DUadmonition}[2][class-arg]{%
  % try \DUadmonition#1{#2}:
  \ifcsname DUadmonition#1\endcsname%
    \csname DUadmonition#1\endcsname{#2}%
  \else
    \begin{center}
      \fbox{\parbox{0.9\textwidth}{#2}}
    \end{center}
  \fi
}

% fieldlist environment
\ifthenelse{\isundefined{\DUfieldlist}}{
  \newenvironment{DUfieldlist}%
    {\quote\description}
    {\enddescription\endquote}
}{}

% title for topics, admonitions and sidebar
\providecommand*{\DUtitle}[2][class-arg]{%
  % call \DUtitle#1{#2} if it exists:
  \ifcsname DUtitle#1\endcsname%
    \csname DUtitle#1\endcsname{#2}%
  \else
    \smallskip\noindent\textbf{#2}\smallskip%
  \fi
}

% hyperlinks:
\ifthenelse{\isundefined{\hypersetup}}{
  \usepackage[colorlinks=true,linkcolor=blue,urlcolor=blue]{hyperref}
  \urlstyle{same} % normal text font (alternatives: tt, rm, sf)
}{}


%%% Body
\begin{document}

% Automatically generated reST file from Doconce source
% (http://code.google.com/p/doconce/)

% Missing: FIGURE, MOVIE, environments


%___________________________________________________________________________

\section*{\phantomsection%
  Doconce: Document Once, Include Anywhere%
  \addcontentsline{toc}{section}{Doconce: Document Once, Include Anywhere}%
  \label{doconce-document-once-include-anywhere}%
}
%
\begin{DUfieldlist}
\item[{Author:}]
Hans Petter Langtangen

\item[{Date:}]
Feb 8, 2013
%
\begin{itemize}

\item When writing a note, report, manual, etc., do you find it difficult
to choose the typesetting format? That is, to choose between plain
(email-like) text, wiki, Word/OpenOffice, LaTeX, HTML,
reStructuredText, Sphinx, XML, etc.  Would it be convenient to
start with some very simple text-like format that easily converts
to the formats listed above, and then at some later stage
eventually go with a particular format?

\item Do you need to write documents in varying formats but find it
difficult to remember all the typesetting details of various
formats like \href{http://refcards.com/docs/silvermanj/amslatex/LaTeXRefCard.v2.0.pdf}{LaTeX}, \href{http://www.htmlcodetutorial.com/}{HTML}, \href{http://docutils.sourceforge.net/docs/ref/rst/restructuredtext.html}{reStructuredText}, \href{http://sphinx.pocoo.org/contents.html}{Sphinx}, and \href{http://code.google.com/p/support/wiki/WikiSyntax}{wiki}? Would it be convenient
to generate the typesetting details of a particular format from a
very simple text-like format with minimal tagging?

\item Do you have the same information scattered around in different
documents in different typesetting formats? Would it be a good idea
to write things once, in one format, stored in one place, and
include it anywhere?

\end{itemize}

\end{DUfieldlist}

If any of these questions are of interest, you should keep on reading.


%___________________________________________________________________________

\section*{\phantomsection%
  What Does Doconce Look Like?%
  \addcontentsline{toc}{section}{What Does Doconce Look Like?}%
  \label{what-does-doconce-look-like}%
}

Doconce text looks like ordinary text, but there are some almost invisible
text constructions that allow you to control the formating. Here are
som examples.
%
\begin{quote}
%
\begin{itemize}

\item Bullet lists arise from lines starting with \texttt{*}.

\item \emph{Emphasized words} are surrounded by \texttt{*}.

\item \textbf{Words in boldface} are surrounded by underscores.

\item Words from computer code are enclosed in back quotes and
then typeset \texttt{verbatim (in a monospace font)}.

\item Section headings are recognied by equality (\texttt{=}) signs before
and after the title, and the number of \texttt{=} signs indicates the
level of the section: 7 for main section, 5 for subsection, and
3 for subsubsection.

\item Paragraph headings are recognized by a double underscore
before and after the heading.

\item The abstract of a document starts with \emph{Abstract} as paragraph
heading, and all text up to the next heading makes up the abstract,

\item Blocks of computer code can easily be included by placing
\texttt{!bc} (begin code) and \texttt{!ec} (end code) commands at separate lines
before and after the code block.

\item Blocks of computer code can also be imported from source files.

\item Blocks of LaTeX mathematics can easily be included by placing
\texttt{!bt} (begin TeX) and \texttt{!et} (end TeX) commands at separate lines
before and after the math block.

\item There is support for both LaTeX and text-like inline mathematics.

\item Figures and movies with captions, simple tables,
URLs with links, index list, labels and references are supported.

\item Invisible comments in the output format can be inserted throughout
the text.

\item Visible comments can be inserted so that authors and readers can
comment upon the text (and at any time turn on/off output of such
comments).

\item There is an exercise environment with many advanced features.

\item With a preprocessor, Preprocess or Mako, one can include
other documents (files) and large portions of text can be defined
in or out of the text.

\item With Mako one can also have Python code
embedded in the Doconce document and thereby parameterize the
text (e.g., one text can describe programming in two languages).

\end{itemize}

\end{quote}

Here is an example of some simple text written in the Doconce format:
%
\begin{quote}{\ttfamily \raggedright \noindent
=====~A~Subsection~with~Sample~Text~=====\\
label\{my:first:sec\}\\
~\\
Ordinary~text~looks~like~ordinary~text,~and~the~tags~used~for\\
\_boldface\_~words,~*emphasized*~words,~and~`computer`~words~look\\
natural~in~plain~text.~~Lists~are~typeset~as~you~would~do~in~email,\\
~\\
~~*~item~1\\
~~*~item~2\\
~~*~item~3\\
~\\
Lists~can~also~have~automatically~numbered~items~instead~of~bullets,\\
~\\
~~o~item~1\\
~~o~item~2\\
~~o~item~3\\
~\\
URLs~with~a~link~word~are~possible,~as~in~"hpl":~"http://folk.uio.no/hpl".\\
If~the~word~is~URL,~the~URL~itself~becomes~the~link~name,\\
as~in~"URL":~"tutorial.do.txt".\\
~\\
References~to~sections~may~use~logical~names~as~labels~(e.g.,~a\\
"label"~command~right~after~the~section~title),~as~in~the~reference~to\\
Section~ref\{my:first:sec\}.\\
~\\
Doconce~also~allows~inline~comments~of~the~form~{[}name:~comment{]}~(with\\
a~space~after~`name:`),~e.g.,~such~as~{[}hpl:~here~I~will~make~some\\
remarks~to~the~text{]}.~Inline~comments~can~be~removed~from~the~output\\
by~a~command-line~argument~(see~Section~ref\{doconce2formats\}~for~an\\
example).\\
~\\
Tables~are~also~supperted,~e.g.,\\
~\\
~~|-{}-{}-{}-{}-{}-{}-{}-{}-{}-{}-{}-{}-{}-{}-{}-{}-{}-{}-{}-{}-{}-{}-{}-{}-{}-{}-{}-{}-{}-{}-{}-|\\
~~|time~~|~velocity~|~acceleration~|\\
~~|-{}-{}-r-{}-{}-{}-{}-{}-{}-r-{}-{}-{}-{}-{}-{}-{}-{}-{}-{}-r-{}-{}-{}-{}-{}-{}-{}-|\\
~~|~0.0~~|~1.4186~~~|~-5.01~~~~~~~~|\\
~~|~2.0~~|~1.376512~|~11.919~~~~~~~|\\
~~|~4.0~~|~1.1E+1~~~|~14.717624~~~~|\\
~~|-{}-{}-{}-{}-{}-{}-{}-{}-{}-{}-{}-{}-{}-{}-{}-{}-{}-{}-{}-{}-{}-{}-{}-{}-{}-{}-{}-{}-{}-{}-{}-|\\
~\\
\#~lines~beginning~with~\#~are~comment~lines
}
\end{quote}

The Doconce text above results in the following little document:


%___________________________________________________________________________

\subsection*{\phantomsection%
  A Subsection with Sample Text%
  \addcontentsline{toc}{subsection}{A Subsection with Sample Text}%
  \label{a-subsection-with-sample-text}%
  \label{my-first-sec}%
}

Ordinary text looks like ordinary text, and the tags used for
\textbf{boldface} words, \emph{emphasized} words, and \texttt{computer} words look
natural in plain text.  Lists are typeset as you would do in an email,
%
\begin{quote}
%
\begin{itemize}

\item item 1

\item item 2

\item item 3

\end{itemize}

\end{quote}

Lists can also have numbered items instead of bullets, just use an \texttt{o}
(for ordered) instead of the asterisk:
%
\begin{quote}
\newcounter{listcnt0}
\begin{list}{\arabic{listcnt0}.}
{
\usecounter{listcnt0}
\setlength{\rightmargin}{\leftmargin}
}

\item item 1

\item item 2

\item item 3
\end{list}

\end{quote}

URLs with a link word are possible, as in \href{http://folk.uio.no/hpl}{hpl}.
If the word is URL, the URL itself becomes the link name,
as in \url{tutorial.do.txt}.

References to sections may use logical names as labels (e.g., a
``label'' command right after the section title), as in the reference to
the section \hyperref[a-subsection-with-sample-text]{A Subsection with Sample Text}.

Doconce also allows inline comments such as (\textbf{hpl}: here I will make
some remarks to the text) for allowing authors to make notes. Inline
comments can be removed from the output by a command-line argument
(see the section \hyperref[from-doconce-to-other-formats]{From Doconce to Other Formats} for an example).

Tables are also supperted, e.g.,

\setlength{\DUtablewidth}{\linewidth}
\begin{longtable*}[c]{|p{0.156\DUtablewidth}|p{0.156\DUtablewidth}|p{0.156\DUtablewidth}|}
\hline
\textbf{%
time
} & \textbf{%
velocity
} & \textbf{%
acceleration
} \\
\hline
\endfirsthead
\hline
\textbf{%
time
} & \textbf{%
velocity
} & \textbf{%
acceleration
} \\
\hline
\endhead
\multicolumn{3}{c}{\hfill ... continued on next page} \\
\endfoot
\endlastfoot

0.0
 & 
1.4186
 & 
-5.01
 \\
\hline

2.0
 & 
1.376512
 & 
11.919
 \\
\hline

4.0
 & 
1.1E+1
 & 
14.717624
 \\
\hline
\end{longtable*}


%___________________________________________________________________________

\subsection*{\phantomsection%
  Mathematics and Computer Code%
  \addcontentsline{toc}{subsection}{Mathematics and Computer Code}%
  \label{mathematics-and-computer-code}%
}

Inline mathematics, such as v = sin(x),
allows the formula to be specified both as LaTeX and as plain text.
This results in a professional LaTeX typesetting, but in other formats
the text version normally looks better than raw LaTeX mathematics with
backslashes. An inline formula like v = sin(x) is
typeset as:
%
\begin{quote}{\ttfamily \raggedright \noindent
\$\textbackslash{}nu~=~\textbackslash{}sin(x)\$|\$v~=~sin(x)\$
}
\end{quote}

The pipe symbol acts as a delimiter between LaTeX code and the plain text
version of the formula. If you write a lot of mathematics, only the
output formats \texttt{latex}, \texttt{pdflatex}, \texttt{html}, \texttt{sphinx}, and \texttt{pandoc}
are of interest
and all these support inline LaTeX mathematics so then you will naturally
drop the pipe symbol and write just:
%
\begin{quote}{\ttfamily \raggedright \noindent
\$\textbackslash{}nu~=~\textbackslash{}sin(x)\$
}
\end{quote}

However, if you want more textual formats, like plain text or reStructuredText,
the text after the pipe symbol may help to make the math formula more readable
if there are backslahes or other special LaTeX symbols in the LaTeX code.

Blocks of mathematics are typeset with raw LaTeX, inside
\texttt{!bt} and \texttt{!et} (begin TeX, end TeX) instructions:
%
\begin{quote}{\ttfamily \raggedright \noindent
!bt\\
\textbackslash{}begin\{align\}\\
\{\textbackslash{}partial~u\textbackslash{}over\textbackslash{}partial~t\}~\&=~\textbackslash{}nabla\textasciicircum{}2~u~+~f,~label\{myeq1\}\textbackslash{}\textbackslash{}\\
\{\textbackslash{}partial~v\textbackslash{}over\textbackslash{}partial~t\}~\&=~\textbackslash{}nabla\textbackslash{}cdot(q(u)\textbackslash{}nabla~v)~+~g\\
\textbackslash{}end\{align\}\\
!et
}
\end{quote}

% Note: !bt and !et (and !bc and !ec below) are used to illustrate

% tex and code blocks in inside verbatim blocks and are replaced

% by !bt, !et, !bc, and !ec after all other formatting is finished.

The result looks like this:
%
\begin{quote}{\ttfamily \raggedright \noindent
\textbackslash{}begin\{align\}\\
\{\textbackslash{}partial~u\textbackslash{}over\textbackslash{}partial~t\}~\&=~\textbackslash{}nabla\textasciicircum{}2~u~+~f,~label\{myeq1\}\textbackslash{}\textbackslash{}\\
\{\textbackslash{}partial~v\textbackslash{}over\textbackslash{}partial~t\}~\&=~\textbackslash{}nabla\textbackslash{}cdot(q(u)\textbackslash{}nabla~v)~+~g\\
\textbackslash{}end\{align\}
}
\end{quote}

Of course, such blocks only looks nice in formats with support
for LaTeX mathematics, and here the align environment in particular
(this includes \texttt{latex}, \texttt{pdflatex}, \texttt{html}, and \texttt{sphinx}). The raw
LaTeX syntax appears in simpler formats, but can still be useful
for those who can read LaTeX syntax.

You can have blocks of computer code, starting and ending with
\texttt{!bc} and \texttt{!ec} instructions, respectively:
%
\begin{quote}{\ttfamily \raggedright \noindent
!bc~pycod\\
from~math~import~sin,~pi\\
def~myfunc(x):\\
~~~~return~sin(pi*x)\\
~\\
import~integrate\\
I~=~integrate.trapezoidal(myfunc,~0,~pi,~100)\\
!ec
}
\end{quote}

Such blocks are formatted as:
%
\begin{quote}{\ttfamily \raggedright \noindent
from~math~import~sin,~pi\\
def~myfunc(x):\\
~~~~return~sin(pi*x)\\
~\\
import~integrate\\
I~=~integrate.trapezoidal(myfunc,~0,~pi,~100)
}
\end{quote}

A code block must come after some plain sentence (at least for successful
output to \texttt{sphinx}, \texttt{rst}, and ASCII-close formats),
not directly after a section/paragraph heading or a table.

One can also copy computer code directly from files, either the
complete file or specified parts.  Computer code is then never
duplicated in the documentation (important for the principle of
avoiding copying information!).

Another document can be included by writing \texttt{\# \#include "mynote.do.txt"}
at the beginning of a line.  Doconce documents have
extension \texttt{do.txt}. The \texttt{do} part stands for doconce, while the
trailing \texttt{.txt} denotes a text document so that editors gives you
plain text editing capabilities.


%___________________________________________________________________________

\subsection*{\phantomsection%
  Macros (Newcommands), Cross-References, Index, and Bibliography%
  \addcontentsline{toc}{subsection}{Macros (Newcommands), Cross-References, Index, and Bibliography}%
  \label{macros-newcommands-cross-references-index-and-bibliography}%
  \label{newcommands}%
}

Doconce supports a type of macros via a LaTeX-style \emph{newcommand}
construction.  The newcommands defined in a file with name
\texttt{newcommand\_replace.tex} are expanded when Doconce is filtered to
other formats, except for LaTeX (since LaTeX performs the expansion
itself).  Newcommands in files with names \texttt{newcommands.tex} and
\texttt{newcommands\_keep.tex} are kept unaltered when Doconce text is
filtered to other formats, except for the Sphinx format. Since Sphinx
understands LaTeX math, but not newcommands if the Sphinx output is
HTML, it makes most sense to expand all newcommands.  Normally, a user
will put all newcommands that appear in math blocks surrounded by
\texttt{!bt} and \texttt{!et} in \texttt{newcommands\_keep.tex} to keep them unchanged, at
least if they contribute to make the raw LaTeX math text easier to
read in the formats that cannot render LaTeX.  Newcommands used
elsewhere throughout the text will usually be placed in
\texttt{newcommands\_replace.tex} and expanded by Doconce.  The definitions of
newcommands in the \texttt{newcommands*.tex} files \emph{must} appear on a single
line (multi-line newcommands are too hard to parse with regular
expressions).

Recent versions of Doconce also offer cross referencing, typically one
can define labels below (sub)sections, in figure captions, or in
equations, and then refer to these later. Entries in an index can be
defined and result in an index at the end for the LaTeX and Sphinx
formats. Citations to literature, with an accompanying bibliography in
a file, are also supported. The syntax of labels, references,
citations, and the bibliography closely resembles that of LaTeX,
making it easy for Doconce documents to be integrated in LaTeX
projects (manuals, books). For further details on functionality and
syntax we refer to the \texttt{doc/manual/manual.do.txt} file (see the
\href{https://doconce.googlecode.com/hg/doc/demos/manual/index.html}{demo page}
for various formats of this document).

% Example on including another Doconce file (using preprocess):


%___________________________________________________________________________

\section*{\phantomsection%
  From Doconce to Other Formats%
  \addcontentsline{toc}{section}{From Doconce to Other Formats}%
  \label{from-doconce-to-other-formats}%
  \label{doconce2formats}%
}

Transformation of a Doconce document \texttt{mydoc.do.txt} to various other
formats applies the script \texttt{doconce format}:
%
\begin{quote}{\ttfamily \raggedright \noindent
Terminal>~doconce~format~format~mydoc.do.txt
}
\end{quote}

or just:
%
\begin{quote}{\ttfamily \raggedright \noindent
Terminal>~doconce~format~format~mydoc
}
\end{quote}


%___________________________________________________________________________

\subsection*{\phantomsection%
  Preprocessing%
  \addcontentsline{toc}{subsection}{Preprocessing}%
  \label{preprocessing}%
}

The \texttt{preprocess} and \texttt{mako} programs are used to preprocess the
file, and options to \texttt{preprocess} and/or \texttt{mako} can be added after the
filename. For example:
%
\begin{quote}{\ttfamily \raggedright \noindent
Terminal>~doconce~format~latex~mydoc~-Dextra\_sections~-DVAR1=5~~~~~\#~preprocess\\
Terminal>~doconce~format~latex~yourdoc~extra\_sections=True~VAR1=5~~\#~mako
}
\end{quote}

The variable \texttt{FORMAT} is always defined as the current format when
running \texttt{preprocess} or \texttt{mako}. That is, in the last example, \texttt{FORMAT} is
defined as \texttt{latex}. Inside the Doconce document one can then perform
format specific actions through tests like \texttt{\#if FORMAT == "latex"}
(for \texttt{preprocess}) or \texttt{\% if FORMAT == "latex":} (for \texttt{mako}).


%___________________________________________________________________________

\subsection*{\phantomsection%
  Removal of inline comments%
  \addcontentsline{toc}{subsection}{Removal of inline comments}%
  \label{removal-of-inline-comments}%
}

% mention notes also

The command-line arguments \texttt{-{}-no-preprocess} and \texttt{-{}-no-mako} turn off
running \texttt{preprocess} and \texttt{mako}, respectively.

Inline comments in the text are removed from the output by:
%
\begin{quote}{\ttfamily \raggedright \noindent
Terminal>~doconce~format~latex~mydoc~-{}-skip\_inline\_comments
}
\end{quote}

One can also remove all such comments from the original Doconce
file by running:
%
\begin{quote}{\ttfamily \raggedright \noindent
Terminal>~doconce~remove\_inline\_comments~mydoc
}
\end{quote}

This action is convenient when a Doconce document reaches its final form
and comments by different authors should be removed.


%___________________________________________________________________________

\subsection*{\phantomsection%
  HTML%
  \addcontentsline{toc}{subsection}{HTML}%
  \label{id1}%
}

Making an HTML version of a Doconce file \texttt{mydoc.do.txt}
is performed by:
%
\begin{quote}{\ttfamily \raggedright \noindent
Terminal>~doconce~format~html~mydoc
}
\end{quote}

The resulting file \texttt{mydoc.html} can be loaded into any web browser for viewing.

The HTML style can be defined either in the header of the HTML file or
in an external CSS file. The latter is enabled by the command-line
argument \texttt{-{}-css=filename}. There is a default style with blue headings,
and a style with the \href{http://ethanschoonover.com/solarized}{solarized}
color palette, specified by the \texttt{-{}-html-solarized} command line
argument. If there is no file with name \texttt{filename} in the \texttt{-{}-css=filename}
specification, the blue or solarized styles are written to \texttt{filename}
and linked from the HTML document. You can provide your own style sheet
either by replacing the content inside the \texttt{style} tags or by
specifying a CSS file through the \texttt{-{}-css=filename} option.

If the Pygments package (including the \texttt{pygmentize} program)
is installed, code blocks are typeset with
aid of this package. The command-line argument \texttt{-{}-no-pygments-html}
turns off the use of Pygments and makes code blocks appear with
plain (\texttt{pre}) HTML tags. The option \texttt{-{}-pygments-html-linenos} turns
on line numbers in Pygments-formatted code blocks.

The HTML file can be embedded in a template if the Doconce document
does not have a title (because then there will be
no header and footer in the HTML file). The template file must contain
valid HTML code and can have three ``slots'': \texttt{\%(title)s} for a title,
\texttt{\%(date)s} for a date, and \texttt{\%(main)s} for the main body of text, i.e., the
Doconce document translated to HTML. The title becomes the first
heading in the Doconce document, and the date is extracted from the
\texttt{DATE:} line, if present. With the template feature one can easily embed
the text in the look and feel of a website. The template can be extracted
from the source code of a page at the site; just insert \texttt{\%(title)s} and
\texttt{\%(date)s} at appropriate places and replace the main bod of text
by \texttt{\%(main)s}. Here is an example:
%
\begin{quote}{\ttfamily \raggedright \noindent
Terminal>~doconce~format~html~mydoc~-{}-html-template=mytemplate.html
}
\end{quote}


%___________________________________________________________________________

\subsection*{\phantomsection%
  Blogs%
  \addcontentsline{toc}{subsection}{Blogs}%
  \label{blogs}%
}

Doconce can be used for writing blogs provided the blog site accepts
raw HTML code. Google's Blogger service (\texttt{blogger.com} or
\texttt{blogname.blogspot.com})
is particularly well suited since it also allows extensive LaTeX mathematics via
MathJax.
Write the blog text as a Doconce document without any title, author, and
date. Then generate HTML as described above. Copy the text and paste it
into the text area in the blog, making sure the input format is HTML.
On Google's Blogger service you can use Doconce to generate blogs with
LaTeX mathematics and pretty (pygmentized) blocks of computer code.
See a \href{http://doconce.blogspot.no}{blog example} for details on blogging.

\DUadmonition[warning]{
\DUtitle[warning]{Warning}

In the comments after the blog one cannot paste raw HTML code with MathJax
scripts so there is no support for mathematics in the comments.
}

WordPress (\texttt{wordpress.com}) allows raw HTML code in blogs,
but has very limited
LaTeX support, basically only formulas. The \texttt{-{}-wordpress} option to
\texttt{doconce} modifies the HTML code such that all equations are typeset
in a way that is acceptable to WordPress.
There is a \href{http://doconce.wordpress.com}{doconce example}
on blogging with mathematics and code on WordPress.


%___________________________________________________________________________

\subsection*{\phantomsection%
  Pandoc and Markdown%
  \addcontentsline{toc}{subsection}{Pandoc and Markdown}%
  \label{pandoc-and-markdown}%
}

Output in Pandoc's extended Markdown format results from:
%
\begin{quote}{\ttfamily \raggedright \noindent
Terminal>~doconce~format~pandoc~mydoc
}
\end{quote}

The name of the output file is \texttt{mydoc.mkd}.
From this format one can go to numerous other formats:
%
\begin{quote}{\ttfamily \raggedright \noindent
Terminal>~pandoc~-R~-t~mediawiki~-o~mydoc.mwk~-{}-toc~mydoc.mkd
}
\end{quote}

Pandoc supports \texttt{latex}, \texttt{html}, \texttt{odt} (OpenOffice), \texttt{docx} (Microsoft
Word), \texttt{rtf}, \texttt{texinfo}, to mention some. The \texttt{-R} option makes
Pandoc pass raw HTML or LaTeX to the output format instead of ignoring it,
while the \texttt{-{}-toc} option generates a table of contents.
See the \href{http://johnmacfarlane.net/pandoc/README.html}{Pandoc documentation}
for the many features of the \texttt{pandoc} program.

Pandoc is useful to go from LaTeX mathematics to, e.g., HTML or MS Word.
There are two ways (experiment to find the best one for your document):
\texttt{doconce format pandoc} and then translating using \texttt{pandoc}, or
\texttt{doconce format latex}, and then going from LaTeX to the desired format
using \texttt{pandoc}.
Here is an example on the latter strategy:
%
\begin{quote}{\ttfamily \raggedright \noindent
Terminal>~doconce~format~latex~mydoc\\
Terminal>~doconce~ptex2tex~mydoc\\
Terminal>~doconce~replace~'\textbackslash{}Verb!'~'\textbackslash{}verb!'~mydoc.tex\\
Terminal>~pandoc~-f~latex~-t~docx~-o~mydoc.docx~mydoc.tex
}
\end{quote}

When we go through \texttt{pandoc}, only single equations or \texttt{align*}
environments are well understood.

Note that Doconce applies the \texttt{Verb} macro from the \texttt{fancyvrb} package
while \texttt{pandoc} only supports the standard \texttt{verb} construction for
inline verbatim text.  Moreover, quite some additional \texttt{doconce
replace} and \texttt{doconce subst} edits might be needed on the \texttt{.mkd} or
\texttt{.tex} files to successfully have mathematics that is well translated
to MS Word.  Also when going to reStructuredText using Pandoc, it can
be advantageous to go via LaTeX.

Here is an example where we take a Doconce snippet (without title, author,
and date), maybe with some unnumbered equations, and quickly generate
HTML with mathematics displayed my MathJax:
%
\begin{quote}{\ttfamily \raggedright \noindent
Terminal>~doconce~format~pandoc~mydoc\\
Terminal>~pandoc~-t~html~-o~mydoc.html~-s~-{}-mathjax~mydoc.mkd
}
\end{quote}

The \texttt{-s} option adds a proper header and footer to the \texttt{mydoc.html} file.
This recipe is a quick way of makeing HTML notes with (some) mathematics.


%___________________________________________________________________________

\subsection*{\phantomsection%
  LaTeX%
  \addcontentsline{toc}{subsection}{LaTeX}%
  \label{id2}%
}

Making a LaTeX file \texttt{mydoc.tex} from \texttt{mydoc.do.txt} is done in two steps:
.. Note: putting code blocks inside a list is not successful in many

% formats - the text may be messed up. A better choice is a paragraph

% environment, as used here.

\emph{Step 1.} Filter the doconce text to a pre-LaTeX form \texttt{mydoc.p.tex} for
the \texttt{ptex2tex} program (or \texttt{doconce ptex2tex}):
%
\begin{quote}{\ttfamily \raggedright \noindent
Terminal>~doconce~format~latex~mydoc
}
\end{quote}

LaTeX-specific commands (``newcommands'') in math formulas and similar
can be placed in files \texttt{newcommands.tex}, \texttt{newcommands\_keep.tex}, or
\texttt{newcommands\_replace.tex} (see the section \hyperref[macros-newcommands-cross-references-index-and-bibliography]{Macros (Newcommands), Cross-References, Index, and Bibliography}).
If these files are present, they are included in the LaTeX document
so that your commands are defined.

An option \texttt{-{}-latex-printed} makes some adjustments for documents
aimed at being printed. For example, links to web resources are
associated with a footnote listing the complete web address (URL).

\emph{Step 2.} Run \texttt{ptex2tex} (if you have it) to make a standard LaTeX file:
%
\begin{quote}{\ttfamily \raggedright \noindent
Terminal>~ptex2tex~mydoc
}
\end{quote}

In case you do not have \texttt{ptex2tex}, you may run a (very) simplified version:
%
\begin{quote}{\ttfamily \raggedright \noindent
Terminal>~doconce~ptex2tex~mydoc
}
\end{quote}

Note that Doconce generates a \texttt{.p.tex} file with some preprocessor macros
that can be used to steer certain properties of the LaTeX document.
For example, to turn on the Helvetica font instead of the standard
Computer Modern font, run:
%
\begin{quote}{\ttfamily \raggedright \noindent
Terminal>~ptex2tex~-DHELVETICA~mydoc\\
Terminal>~doconce~ptex2tex~mydoc~-DHELVETICA~~\#~alternative
}
\end{quote}

The title, authors, and date are by default typeset in a non-standard
way to enable a nicer treatment of multiple authors having
institutions in common. However, the standard LaTeX ``maketitle'' heading
is also available through \texttt{-DLATEX\_HEADING=traditional}.
A separate titlepage can be generate by
\texttt{-DLATEX\_HEADING=titlepage}.

Preprocessor variables to be defined or undefined are
%
\begin{quote}
%
\begin{itemize}

\item \texttt{BOOK} for the ``book'' documentclass rather than the standard
``article'' class (necessary if you apply chapter headings)

\item \texttt{PALATINO} for the Palatino font

\item \texttt{HELVETIA} for the Helvetica font

\item \texttt{A4PAPER} for A4 paper size

\item \texttt{A6PAPER} for A6 paper size (suitable for reading on small devices)

\item \texttt{MOVIE15} for using the movie15 LaTeX package to display movies

\item \texttt{PREAMBLE} to turn the LaTeX preamble on or off (i.e., complete document
or document to be included elsewhere)

\item \texttt{MINTED} for inclusion of the minted package (which requires \texttt{latex}
or \texttt{pdflatex} to be run with the \texttt{-shell-escape} option)

\end{itemize}

\end{quote}

If you are not satisfied with the Doconce preamble, you can provide
your own preamble by adding the command-line option \texttt{-{}-latex-preamble=myfile}.
In case \texttt{myfile} contains a documentclass definition, Doconce assumes
that the file contains the \emph{complete} preamble you want (not that all
the packages listed in the default preamble are required and must be
present in \texttt{myfile}). Otherwise, \texttt{myfile} is assumed to contain
\emph{additional} LaTeX code to be added to the Doconce default preamble.

The \texttt{ptex2tex} tool makes it possible to easily switch between many
different fancy formattings of computer or verbatim code in LaTeX
documents. After any \texttt{!bc} command in the Doconce source you can
insert verbatim block styles as defined in your \texttt{.ptex2tex.cfg}
file, e.g., \texttt{!bc sys} for a terminal session, where \texttt{sys} is set to
a certain environment in \texttt{.ptex2tex.cfg} (e.g., \texttt{CodeTerminal}).
There are about 40 styles to choose from, and you can easily add
new ones.

Also the \texttt{doconce ptex2tex} command supports preprocessor directives
for processing the \texttt{.p.tex} file. The command allows specifications
of code environments as well. Here is an example:
%
\begin{quote}{\ttfamily \raggedright \noindent
Terminal>~doconce~ptex2tex~mydoc~-DLATEX\_HEADING=traditional~\textbackslash{}\\
~~~~~~~~~~-DPALATINO~-DA6PAPER~\textbackslash{}\\
~~~~~~~~~~"sys=\textbackslash{}begin\{quote\}\textbackslash{}begin\{verbatim\}@\textbackslash{}end\{verbatim\}\textbackslash{}end\{quote\}"~\textbackslash{}\\
~~~~~~~~~~fpro=minted~fcod=minted~shcod=Verbatim~envir=ans:nt
}
\end{quote}

Note that \texttt{@} must be used to separate the begin and end LaTeX
commands, unless only the environment name is given (such as \texttt{minted}
above, which implies \texttt{\textbackslash{}begin\{minted\}\{fortran\}} and \texttt{\textbackslash{}end\{minted\}} as
begin and end for blocks inside \texttt{!bc fpro} and \texttt{!ec}).  Specifying
\texttt{envir=ans:nt} means that all other environments are typeset with the
\texttt{anslistings.sty} package, e.g., \texttt{!bc cppcod} will then result in
\texttt{\textbackslash{}begin\{c++\}}. If no environments like \texttt{sys}, \texttt{fpro}, or the common
\texttt{envir} are defined on the command line, the plain \texttt{\textbackslash{}begin\{verbatim\}}
and \texttt{\textbackslash{}end\{verbatim\}} used.

\emph{Step 2b (optional).} Edit the \texttt{mydoc.tex} file to your needs.
For example, you may want to substitute \texttt{section} by \texttt{section*} to
avoid numbering of sections, you may want to insert linebreaks
(and perhaps space) in the title, etc. This can be automatically
edited with the aid of the \texttt{doconce replace} and \texttt{doconce subst}
commands. The former works with substituting text directly, while the
latter performs substitutions using regular expressions.
Here are two examples:
%
\begin{quote}{\ttfamily \raggedright \noindent
Terminal>~doconce~replace~'section\{'~'section*\{'~mydoc.tex\\
Terminal>~doconce~subst~'title\textbackslash{}\{(.+)Using~(.+)\textbackslash{}\}'~\textbackslash{}\\
~~~~~~~~~~'title\{\textbackslash{}g<1>~\textbackslash{}\textbackslash{}\textbackslash{}\textbackslash{}~{[}1.5mm{]}~Using~\textbackslash{}g<2>'~mydoc.tex
}
\end{quote}

A lot of tailored fixes to the LaTeX document can be done by
an appropriate set of text replacements and regular expression
substitutions. You are anyway encourged to make a script for
generating PDF from the LaTeX file.

\emph{Step 3.} Compile \texttt{mydoc.tex}
and create the PDF file:
%
\begin{quote}{\ttfamily \raggedright \noindent
Terminal>~latex~mydoc\\
Terminal>~latex~mydoc\\
Terminal>~makeindex~mydoc~~~\#~if~index\\
Terminal>~bibitem~mydoc~~~~~\#~if~bibliography\\
Terminal>~latex~mydoc\\
Terminal>~dvipdf~mydoc
}
\end{quote}

If one wishes to run \texttt{ptex2tex} and use the minted LaTeX package for
typesetting code blocks (\texttt{Minted\_Python}, \texttt{Minted\_Cpp}, etc., in
\texttt{ptex2tex} specified through the \texttt{*pro} and \texttt{*cod} variables in
\texttt{.ptex2tex.cfg} or \texttt{\$HOME/.ptex2tex.cfg}), the minted LaTeX package is
needed.  This package is included by running \texttt{ptex2tex} with the
\texttt{-DMINTED} option:
%
\begin{quote}{\ttfamily \raggedright \noindent
Terminal>~ptex2tex~-DMINTED~mydoc
}
\end{quote}

In this case, \texttt{latex} must be run with the
\texttt{-shell-escape} option:
%
\begin{quote}{\ttfamily \raggedright \noindent
Terminal>~latex~-shell-escape~mydoc\\
Terminal>~latex~-shell-escape~mydoc\\
Terminal>~makeindex~mydoc~~~\#~if~index\\
Terminal>~bibitem~mydoc~~~~~\#~if~bibliography\\
Terminal>~latex~-shell-escape~mydoc\\
Terminal>~dvipdf~mydoc
}
\end{quote}

When running \texttt{doconce ptex2tex mydoc envir=minted} (or other minted
specifications with \texttt{doconce ptex2tex}), the minted package is automatically
included so there is no need for the \texttt{-DMINTED} option.


%___________________________________________________________________________

\subsection*{\phantomsection%
  PDFLaTeX%
  \addcontentsline{toc}{subsection}{PDFLaTeX}%
  \label{pdflatex}%
}

Running \texttt{pdflatex} instead of \texttt{latex} follows almost the same steps,
but the start is:
%
\begin{quote}{\ttfamily \raggedright \noindent
Terminal>~doconce~format~latex~mydoc
}
\end{quote}

Then \texttt{ptex2tex} is run as explained above, and finally:
%
\begin{quote}{\ttfamily \raggedright \noindent
Terminal>~pdflatex~-shell-escape~mydoc\\
Terminal>~makeindex~mydoc~~~\#~if~index\\
Terminal>~bibitem~mydoc~~~~~\#~if~bibliography\\
Terminal>~pdflatex~-shell-escape~mydoc
}
\end{quote}


%___________________________________________________________________________

\subsection*{\phantomsection%
  Plain ASCII Text%
  \addcontentsline{toc}{subsection}{Plain ASCII Text}%
  \label{plain-ascii-text}%
}

We can go from Doconce ``back to'' plain untagged text suitable for viewing
in terminal windows, inclusion in email text, or for insertion in
computer source code:
%
\begin{quote}{\ttfamily \raggedright \noindent
Terminal>~doconce~format~plain~mydoc.do.txt~~\#~results~in~mydoc.txt
}
\end{quote}


%___________________________________________________________________________

\subsection*{\phantomsection%
  reStructuredText%
  \addcontentsline{toc}{subsection}{reStructuredText}%
  \label{id3}%
}

Going from Doconce to reStructuredText gives a lot of possibilities to
go to other formats. First we filter the Doconce text to a
reStructuredText file \texttt{mydoc.rst}:
%
\begin{quote}{\ttfamily \raggedright \noindent
Terminal>~doconce~format~rst~mydoc.do.txt
}
\end{quote}

We may now produce various other formats:
%
\begin{quote}{\ttfamily \raggedright \noindent
Terminal>~rst2html.py~~mydoc.rst~>~mydoc.html~\#~html\\
Terminal>~rst2latex.py~mydoc.rst~>~mydoc.tex~~\#~latex\\
Terminal>~rst2xml.py~~~mydoc.rst~>~mydoc.xml~~\#~XML\\
Terminal>~rst2odt.py~~~mydoc.rst~>~mydoc.odt~~\#~OpenOffice
}
\end{quote}

The OpenOffice file \texttt{mydoc.odt} can be loaded into OpenOffice and
saved in, among other things, the RTF format or the Microsoft Word format.
However, it is more convenient to use the program \texttt{unovonv}
to convert between the many formats OpenOffice supports \emph{on the command line}.
Run:
%
\begin{quote}{\ttfamily \raggedright \noindent
Terminal>~unoconv~-{}-show
}
\end{quote}

to see all the formats that are supported.
For example, the following commands take
\texttt{mydoc.odt} to Microsoft Office Open XML format,
classic MS Word format, and PDF:
%
\begin{quote}{\ttfamily \raggedright \noindent
Terminal>~unoconv~-f~ooxml~mydoc.odt\\
Terminal>~unoconv~-f~doc~mydoc.odt\\
Terminal>~unoconv~-f~pdf~mydoc.odt
}
\end{quote}

\emph{Remark about Mathematical Typesetting.} At the time of this writing, there is no easy way to go from Doconce
and LaTeX mathematics to reST and further to OpenOffice and the
``MS Word world''. Mathematics is only fully supported by \texttt{latex} as
output and to a wide extent also supported by the \texttt{sphinx} output format.
Some links for going from LaTeX to Word are listed below.
%
\begin{quote}
%
\begin{itemize}

\item \url{http://ubuntuforums.org/showthread.php?t=1033441}

\item \url{http://tug.org/utilities/texconv/textopc.html}

\item \url{http://nileshbansal.blogspot.com/2007/12/latex-to-openofficeword.html}

\end{itemize}

\end{quote}


%___________________________________________________________________________

\subsection*{\phantomsection%
  Sphinx%
  \addcontentsline{toc}{subsection}{Sphinx}%
  \label{id4}%
}

Sphinx documents demand quite some steps in their creation. We have automated
most of the steps through the \texttt{doconce sphinx\_dir} command:
%
\begin{quote}{\ttfamily \raggedright \noindent
Terminal>~doconce~sphinx\_dir~author="authors'~names"~\textbackslash{}\\
~~~~~~~~~~title="some~title"~version=1.0~dirname=sphinxdir~\textbackslash{}\\
~~~~~~~~~~theme=mytheme~file1~file2~file3~...
}
\end{quote}

The keywords \texttt{author}, \texttt{title}, and \texttt{version} are used in the headings
of the Sphinx document. By default, \texttt{version} is 1.0 and the script
will try to deduce authors and title from the doconce files \texttt{file1},
\texttt{file2}, etc. that together represent the whole document. Note that
none of the individual Doconce files \texttt{file1}, \texttt{file2}, etc. should
include the rest as their union makes up the whole document.
The default value of \texttt{dirname} is \texttt{sphinx-rootdir}. The \texttt{theme}
keyword is used to set the theme for design of HTML output from
Sphinx (the default theme is \texttt{'default'}).

With a single-file document in \texttt{mydoc.do.txt} one often just runs:
%
\begin{quote}{\ttfamily \raggedright \noindent
Terminal>~doconce~sphinx\_dir~mydoc
}
\end{quote}

and then an appropriate Sphinx directory \texttt{sphinx-rootdir} is made with
relevant files.

The \texttt{doconce sphinx\_dir} command generates a script
\texttt{automake\_sphinx.py} for compiling the Sphinx document into an HTML
document.  One can either run \texttt{automake\_sphinx.py} or perform the
steps in the script manually, possibly with necessary modifications.
You should at least read the script prior to executing it to have
some idea of what is done.

The \texttt{doconce sphinx\_dir} script copies directories named \texttt{figs} or
\texttt{figures} over to the Sphinx directory so that figures are accessible
in the Sphinx compilation.  If figures or movies are located in other
directories, \texttt{automake\_sphinx.py} must be edited accordingly.  Files,
to which there are local links (not \texttt{http:} or \texttt{file:} URLs), must be
placed in the \texttt{\_static} subdirectory of the Sphinx directory. The
utility \texttt{doconce sphinxfix\_localURLs} is run to check for local links
in the Doconce file: for each such link, say \texttt{dir1/dir2/myfile.txt} it
replaces the link by \texttt{\_static/myfile.txt} and copies
\texttt{dir1/dir2/myfile.txt} to a local \texttt{\_static} directory (in the same
directory as the script is run).  However, we recommend instead that
the writer of the document places files in \texttt{\_static} or lets a script
do it automatically. The user must copy all \texttt{\_static/*} files to the
\texttt{\_static} subdirectory of the Sphinx directory.  It may be wise to
always put files, to which there are local links in the Doconce
document, in a \texttt{\_static} or \texttt{\_static-name} directory and use these
local links. Then links do not need to be modified when creating a
Sphinx version of the document.

Doconce comes with a collection of HTML themes for Sphinx documents.
These are packed out in the Sphinx directory, the \texttt{conf.py}
configuration file for Sphinx is edited accordingly, and a script
\texttt{make-themes.sh} can make HTML documents with one or more themes.
For example,
to realize the themes \texttt{fenics} and \texttt{pyramid}, one writes:
%
\begin{quote}{\ttfamily \raggedright \noindent
Terminal>~./make-themes.sh~fenics~pyramid
}
\end{quote}

The resulting directories with HTML documents are \texttt{\_build/html\_fenics}
and \texttt{\_build/html\_pyramid}, respectively. Without arguments,
\texttt{make-themes.sh} makes all available themes (!).

If it is not desirable to use the autogenerated scripts explained
above, here is the complete manual procedure of generating a
Sphinx document from a file \texttt{mydoc.do.txt}.

\emph{Step 1.} Translate Doconce into the Sphinx format:
%
\begin{quote}{\ttfamily \raggedright \noindent
Terminal>~doconce~format~sphinx~mydoc
}
\end{quote}

\emph{Step 2.} Create a Sphinx root directory
either manually or by using the interactive \texttt{sphinx-quickstart}
program. Here is a scripted version of the steps with the latter:
%
\begin{quote}{\ttfamily \raggedright \noindent
mkdir~sphinx-rootdir\\
sphinx-quickstart~<{}<EOF\\
sphinx-rootdir\\
n\\
\_\\
Name~of~My~Sphinx~Document\\
Author\\
version\\
version\\
.rst\\
index\\
n\\
y\\
n\\
n\\
n\\
n\\
y\\
n\\
n\\
y\\
y\\
y\\
EOF
}
\end{quote}

The autogenerated \texttt{conf.py} file
may need some edits if you want to specific layout (Sphinx themes)
of HTML pages. The \texttt{doconce sphinx\_dir} generator makes an extended \texttt{conv.py}
file where, among other things, several useful Sphinx extensions
are included.

\emph{Step 3.} Copy the \texttt{mydoc.rst} file to the Sphinx root directory:
%
\begin{quote}{\ttfamily \raggedright \noindent
Terminal>~cp~mydoc.rst~sphinx-rootdir
}
\end{quote}

If you have figures in your document, the relative paths to those will
be invalid when you work with \texttt{mydoc.rst} in the \texttt{sphinx-rootdir}
directory. Either edit \texttt{mydoc.rst} so that figure file paths are correct,
or simply copy your figure directories to \texttt{sphinx-rootdir}.
Links to local files in \texttt{mydoc.rst} must be modified to links to
files in the \texttt{\_static} directory, see comment above.

\emph{Step 4.} Edit the generated \texttt{index.rst} file so that \texttt{mydoc.rst}
is included, i.e., add \texttt{mydoc} to the \texttt{toctree} section so that it becomes:
%
\begin{quote}{\ttfamily \raggedright \noindent
..~toctree::\\
~~~:maxdepth:~2\\
~\\
~~~mydoc
}
\end{quote}

(The spaces before \texttt{mydoc} are important!)

\emph{Step 5.} Generate, for instance, an HTML version of the Sphinx source:
%
\begin{quote}{\ttfamily \raggedright \noindent
make~clean~~~\#~remove~old~versions\\
make~html
}
\end{quote}

Sphinx can generate a range of different formats:
standalone HTML, HTML in separate directories with \texttt{index.html} files,
a large single HTML file, JSON files, various help files (the qthelp, HTML,
and Devhelp projects), epub, LaTeX, PDF (via LaTeX), pure text, man pages,
and Texinfo files.

\emph{Step 6.} View the result:
%
\begin{quote}{\ttfamily \raggedright \noindent
Terminal>~firefox~\_build/html/index.html
}
\end{quote}

Note that verbatim code blocks can be typeset in a variety of ways
depending the argument that follows \texttt{!bc}: \texttt{cod} gives Python
(\texttt{code-block:: python} in Sphinx syntax) and \texttt{cppcod} gives C++, but
all such arguments can be customized both for Sphinx and LaTeX output.


%___________________________________________________________________________

\subsection*{\phantomsection%
  Wiki Formats%
  \addcontentsline{toc}{subsection}{Wiki Formats}%
  \label{wiki-formats}%
}

There are many different wiki formats, but Doconce only supports three:
\href{http://code.google.com/p/support/wiki/WikiSyntax}{Googlecode wiki},
\href{http://www.mediawiki.org/wiki/Help:Formatting}{MediaWiki}, and
\href{http://www.wikicreole.org/wiki/Creole1.0}{Creole Wiki}.
These formats are called
\texttt{gwiki}, \texttt{mwiki}, and \texttt{cwiki}, respectively.
Transformation from Doconce to these formats is done by:
%
\begin{quote}{\ttfamily \raggedright \noindent
Terminal>~doconce~format~gwiki~mydoc.do.txt\\
Terminal>~doconce~format~mwiki~mydoc.do.txt\\
Terminal>~doconce~format~cwiki~mydoc.do.txt
}
\end{quote}

The produced MediaWiki can be tested in the \href{http://en.wikibooks.org/wiki/Sandbox}{sandbox of
wikibooks.org}. The format
works well with Wikipedia, Wikibooks, and
\href{http://doconcedemo.shoutwiki.com/wiki/Doconce_demo_page}{ShoutWiki},
but not always well elsewhere
(see \href{http://doconcedemo.jumpwiki.com/wiki/First_demo}{this example}).

Large MediaWiki documents can be made with the
\href{http://en.wikipedia.org/w/index.php?title=Special:Book&bookcmd=book_creator}{Book creator}.
From the MediaWiki format one can go to other formats with aid
of \href{http://pediapress.com/code/}{mwlib}. This means that one can
easily use Doconce to write \href{http://en.wikibooks.org}{Wikibooks}
and publish these in PDF and MediaWiki format, while
at the same time, the book can also be published as a
standard LaTeX book, a Sphinx web document, or a collection of HTML files.

The Googlecode wiki document, \texttt{mydoc.gwiki}, is most conveniently stored
in a directory which is a clone of the wiki part of the Googlecode project.
This is far easier than copying and pasting the entire text into the
wiki editor in a web browser.

When the Doconce file contains figures, each figure filename must in
the \texttt{.gwiki} file be replaced by a URL where the figure is
available. There are instructions in the file for doing this. Usually,
one performs this substitution automatically (see next section).


%___________________________________________________________________________

\subsection*{\phantomsection%
  Tweaking the Doconce Output%
  \addcontentsline{toc}{subsection}{Tweaking the Doconce Output}%
  \label{tweaking-the-doconce-output}%
}

Occasionally, one would like to tweak the output in a certain format
from Doconce. One example is figure filenames when transforming
Doconce to reStructuredText. Since Doconce does not know if the
\texttt{.rst} file is going to be filtered to LaTeX or HTML, it cannot know
if \texttt{.eps} or \texttt{.png} is the most appropriate image filename.
The solution is to use a text substitution command or code with, e.g., sed,
perl, python, or scitools subst, to automatically edit the output file
from Doconce. It is then wise to run Doconce and the editing commands
from a script to automate all steps in going from Doconce to the final
format(s). The \texttt{make.sh} files in \texttt{docs/manual} and \texttt{docs/tutorial}
constitute comprehensive examples on how such scripts can be made.


%___________________________________________________________________________

\subsection*{\phantomsection%
  Demos%
  \addcontentsline{toc}{subsection}{Demos}%
  \label{demos}%
}

The current text is generated from a Doconce format stored in the file:
%
\begin{quote}{\ttfamily \raggedright \noindent
docs/tutorial/tutorial.do.txt
}
\end{quote}

The file \texttt{make.sh} in the \texttt{tutorial} directory of the
Doconce source code contains a demo of how to produce a variety of
formats.  The source of this tutorial, \texttt{tutorial.do.txt} is the
starting point.  Running \texttt{make.sh} and studying the various generated
files and comparing them with the original \texttt{tutorial.do.txt} file,
gives a quick introduction to how Doconce is used in a real case.
\href{https://doconce.googlecode.com/hg/doc/demos/tutorial/index.html}{Here}
is a sample of how this tutorial looks in different formats.

There is another demo in the \texttt{docs/manual} directory which
translates the more comprehensive documentation, \texttt{manual.do.txt}, to
various formats. The \texttt{make.sh} script runs a set of translations.


%___________________________________________________________________________

\section*{\phantomsection%
  Installation of Doconce and its Dependencies%
  \addcontentsline{toc}{section}{Installation of Doconce and its Dependencies}%
  \label{installation-of-doconce-and-its-dependencies}%
}


%___________________________________________________________________________

\subsection*{\phantomsection%
  Doconce%
  \addcontentsline{toc}{subsection}{Doconce}%
  \label{doconce}%
}

Doconce itself is pure Python code hosted at \url{http://code.google.com/p/doconce}.  Its installation from the
Mercurial (\texttt{hg}) source follows the standard procedure:
%
\begin{quote}{\ttfamily \raggedright \noindent
\#~Doconce\\
hg~clone~https://code.google.com/p/doconce/~doconce\\
cd~doconce\\
sudo~python~setup.py~install\\
cd~..
}
\end{quote}

Since Doconce is frequently updated, it is recommended to use the
above procedure and whenever a problem occurs, make sure to
update to the most recent version:
%
\begin{quote}{\ttfamily \raggedright \noindent
cd~doconce\\
hg~pull\\
hg~update\\
sudo~python~setup.py~install
}
\end{quote}

Debian GNU/Linux users can also run:
%
\begin{quote}{\ttfamily \raggedright \noindent
sudo~apt-get~install~doconce
}
\end{quote}

This installs the latest release and not the most updated and bugfixed
version.
On Ubuntu one needs to run:
%
\begin{quote}{\ttfamily \raggedright \noindent
sudo~add-apt-repository~ppa:scitools/ppa\\
sudo~apt-get~update\\
sudo~apt-get~install~doconce
}
\end{quote}


%___________________________________________________________________________

\subsection*{\phantomsection%
  Dependencies%
  \addcontentsline{toc}{subsection}{Dependencies}%
  \label{dependencies}%
}


%___________________________________________________________________________

\subsubsection*{\phantomsection%
  Preprocessors%
  \addcontentsline{toc}{subsubsection}{Preprocessors}%
  \label{preprocessors}%
}

If you make use of the \href{http://code.google.com/p/preprocess}{Preprocess}
preprocessor, this program must be installed:
%
\begin{quote}{\ttfamily \raggedright \noindent
svn~checkout~http://preprocess.googlecode.com/svn/trunk/~preprocess\\
cd~preprocess\\
cd~doconce\\
sudo~python~setup.py~install\\
cd~..
}
\end{quote}

A much more advanced alternative to Preprocess is
\href{http://www.makotemplates.org}{Mako}. Its installation is most
conveniently done by \texttt{pip}:
%
\begin{quote}{\ttfamily \raggedright \noindent
pip~install~Mako
}
\end{quote}

This command requires \texttt{pip} to be installed. On Debian Linux systems,
such as Ubuntu, the installation is simply done by:
%
\begin{quote}{\ttfamily \raggedright \noindent
sudo~apt-get~install~python-pip
}
\end{quote}

Alternatively, one can install from the \texttt{pip} \href{http://pypi.python.org/pypi/pip}{source code}.

Mako can also be installed directly from
\href{http://www.makotemplates.org/download.html}{source}: download the
tarball, pack it out, go to the directory and run
the usual \texttt{sudo python setup.py install}.


%___________________________________________________________________________

\subsubsection*{\phantomsection%
  Image file handling%
  \addcontentsline{toc}{subsubsection}{Image file handling}%
  \label{image-file-handling}%
}

Different output formats require different formats of image files.
For example, PostScript or Encapuslated PostScript is required for \texttt{latex}
output, while HTML needs JPEG, GIF, or PNG formats.
Doconce calls up programs from the ImageMagick suite for converting
image files to a proper format if needed. The \href{http://www.imagemagick.org/script/index.php}{ImageMagick suite} can be installed on all major platforms.
On Debian Linux (including Ubuntu) systems one can simply write:
%
\begin{quote}{\ttfamily \raggedright \noindent
sudo~apt-get~install~imagemagick
}
\end{quote}

The convenience program \texttt{doconce combine\_images}, for combining several
images into one, will use \texttt{montage} and \texttt{convert} from ImageMagick and
the \texttt{pdftk}, \texttt{pdfnup}, and \texttt{pdfcrop} programs from the \texttt{texlive-extra-utils}
Debian package. The latter gets installed by:
%
\begin{quote}{\ttfamily \raggedright \noindent
sudo~apt-get~install~texlive-extra-utils
}
\end{quote}


%___________________________________________________________________________

\subsubsection*{\phantomsection%
  Spellcheck%
  \addcontentsline{toc}{subsubsection}{Spellcheck}%
  \label{spellcheck}%
}

The utility \texttt{doconce spellcheck} applies the \texttt{ispell} program for
spellcheck. On Debian (including Ubuntu) it is installed by:
%
\begin{quote}{\ttfamily \raggedright \noindent
sudo~apt-get~install~ispell
}
\end{quote}


%___________________________________________________________________________

\subsubsection*{\phantomsection%
  Ptex2tex for LaTeX Output%
  \addcontentsline{toc}{subsubsection}{Ptex2tex for LaTeX Output}%
  \label{ptex2tex-for-latex-output}%
}

To make LaTeX documents with very flexible choice of typesetting of
verbatim code blocks you need \href{http://code.google.com/p/ptex2tex}{ptex2tex},
which is installed by:
%
\begin{quote}{\ttfamily \raggedright \noindent
svn~checkout~http://ptex2tex.googlecode.com/svn/trunk/~ptex2tex\\
cd~ptex2tex\\
sudo~python~setup.py~install
}
\end{quote}

It may happen that you need additional style files, you can run
a script, \texttt{cp2texmf.sh}:
%
\begin{quote}{\ttfamily \raggedright \noindent
cd~latex\\
sh~cp2texmf.sh~~\#~copy~stylefiles~to~\textasciitilde{}/texmf~directory\\
cd~../..
}
\end{quote}

This script copies some special stylefiles that
that \texttt{ptex2tex} potentially makes use of. Some more standard stylefiles
are also needed. These are installed by:
%
\begin{quote}{\ttfamily \raggedright \noindent
sudo~apt-get~install~texlive-latex-recommended~texlive-latex-extra
}
\end{quote}

on Debian Linux (including Ubuntu) systems. TeXShop on Mac comes with
the necessary stylefiles (if not, they can be found by googling and installed
manually in the \texttt{\textasciitilde{}/texmf/tex/latex/misc} directory).

Note that the \texttt{doconce ptex2tex} command, which needs no installation
beyond Doconce itself, can be used as a simpler alternative to the \texttt{ptex2tex}
program.

The \emph{minted} LaTeX style is offered by \texttt{ptex2tex} and \texttt{doconce ptext2tex}
and popular among many
users. This style requires the package \href{http://pygments.org}{Pygments}
to be installed. On Debian Linux:
%
\begin{quote}{\ttfamily \raggedright \noindent
sudo~apt-get~install~python-pygments
}
\end{quote}

Alternatively, the package can be installed manually:
%
\begin{quote}{\ttfamily \raggedright \noindent
hg~clone~ssh://hg@bitbucket.org/birkenfeld/pygments-main~pygments\\
cd~pygments\\
sudo~python~setup.py~install
}
\end{quote}

If you use the minted style together with \texttt{ptex2tex}, you have to
enable it by the \texttt{-DMINTED} command-line argument to \texttt{ptex2tex}.
This is not necessary if you run the alternative \texttt{doconce ptex2tex} program.

All
use of the minted style requires the \texttt{-shell-escape} command-line
argument when running LaTeX, i.e., \texttt{latex -shell-escape} or \texttt{pdflatex
-shell-escape}.

% Say something about anslistings.sty


%___________________________________________________________________________

\subsubsection*{\phantomsection%
  reStructuredText (reST) Output%
  \addcontentsline{toc}{subsubsection}{reStructuredText (reST) Output}%
  \label{restructuredtext-rest-output}%
}

The \texttt{rst} output from Doconce allows further transformation to LaTeX,
HTML, XML, OpenOffice, and so on, through the \href{http://docutils.sourceforge.net}{docutils} package.  The installation of the
most recent version can be done by:
%
\begin{quote}{\ttfamily \raggedright \noindent
svn~checkout~http://docutils.svn.sourceforge.net/svnroot/docutils/trunk/docutils\\
cd~docutils\\
sudo~python~setup.py~install\\
cd~..
}
\end{quote}

To use the OpenOffice suite you will typically on Debian systems install:
%
\begin{quote}{\ttfamily \raggedright \noindent
sudo~apt-get~install~unovonv~libreoffice~libreoffice-dmaths
}
\end{quote}

There is a possibility to create PDF files from reST documents
using ReportLab instead of LaTeX. The enabling software is
\href{http://code.google.com/p/rst2pdf}{rst2pdf}. Either download the tarball
or clone the svn repository, go to the \texttt{rst2pdf} directory and
run the usual \texttt{sudo python setup.py install}.

Output to \texttt{sphinx} requires of course the
\href{http://sphinx.pocoo.org}{Sphinx software},
installed by:
%
\begin{quote}{\ttfamily \raggedright \noindent
hg~clone~https://bitbucket.org/birkenfeld/sphinx\\
cd~sphinx\\
sudo~python~setup.py~install\\
cd~..
}
\end{quote}


%___________________________________________________________________________

\subsubsection*{\phantomsection%
  Markdown and Pandoc Output%
  \addcontentsline{toc}{subsubsection}{Markdown and Pandoc Output}%
  \label{markdown-and-pandoc-output}%
}

The Doconce format \texttt{pandoc} outputs the document in the Pandoc
extended Markdown format, which via the \texttt{pandoc} program can be
translated to a range of other formats. Installation of \href{http://johnmacfarlane.net/pandoc/}{Pandoc}, written in Haskell, is most
easily done by:
%
\begin{quote}{\ttfamily \raggedright \noindent
sudo~apt-get~install~pandoc
}
\end{quote}

on Debian (Ubuntu) systems.


%___________________________________________________________________________

\subsubsection*{\phantomsection%
  Epydoc Output%
  \addcontentsline{toc}{subsubsection}{Epydoc Output}%
  \label{epydoc-output}%
}

When the output format is \texttt{epydoc} one needs that program too, installed
by:
%
\begin{quote}{\ttfamily \raggedright \noindent
svn~co~https://epydoc.svn.sourceforge.net/svnroot/epydoc/trunk/epydoc~epydoc\\
cd~epydoc\\
sudo~make~install\\
cd~..
}
\end{quote}

\emph{Remark.} Several of the packages above installed from source code
are also available in Debian-based system through the
\texttt{apt-get install} command. However, we recommend installation directly
from the version control system repository as there might be important
updates and bug fixes. For \texttt{svn} directories, go to the directory,
run \texttt{svn update}, and then \texttt{sudo python setup.py install}. For
Mercurial (\texttt{hg}) directories, go to the directory, run
\texttt{hg pull; hg update}, and then \texttt{sudo python setup.py install}.

\end{document}
