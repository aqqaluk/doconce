\documentclass[a4paper]{article}
% generated by Docutils <http://docutils.sourceforge.net/>
\usepackage{fixltx2e} % LaTeX patches, \textsubscript
\usepackage{cmap} % fix search and cut-and-paste in Acrobat
\usepackage{ifthen}
\usepackage[T1]{fontenc}
\usepackage[utf8]{inputenc}
\usepackage{color}
\usepackage{longtable,ltcaption,array}
\setlength{\extrarowheight}{2pt}
\newlength{\DUtablewidth} % internal use in tables

%%% Custom LaTeX preamble
% PDF Standard Fonts
\usepackage{mathptmx} % Times
\usepackage[scaled=.90]{helvet}
\usepackage{courier}

%%% User specified packages and stylesheets

%%% Fallback definitions for Docutils-specific commands

% admonition (specially marked topic)
\providecommand{\DUadmonition}[2][class-arg]{%
  % try \DUadmonition#1{#2}:
  \ifcsname DUadmonition#1\endcsname%
    \csname DUadmonition#1\endcsname{#2}%
  \else
    \begin{center}
      \fbox{\parbox{0.9\textwidth}{#2}}
    \end{center}
  \fi
}

% fieldlist environment
\ifthenelse{\isundefined{\DUfieldlist}}{
  \newenvironment{DUfieldlist}%
    {\quote\description}
    {\enddescription\endquote}
}{}

% title for topics, admonitions and sidebar
\providecommand*{\DUtitle}[2][class-arg]{%
  % call \DUtitle#1{#2} if it exists:
  \ifcsname DUtitle#1\endcsname%
    \csname DUtitle#1\endcsname{#2}%
  \else
    \smallskip\noindent\textbf{#2}\smallskip%
  \fi
}

% hyperlinks:
\ifthenelse{\isundefined{\hypersetup}}{
  \usepackage[colorlinks=true,linkcolor=blue,urlcolor=blue]{hyperref}
  \urlstyle{same} % normal text font (alternatives: tt, rm, sf)
}{}


%%% Body
\begin{document}

% Automatically generated reST file from Doconce source
% (http://code.google.com/p/doconce/)


%___________________________________________________________________________

\section*{\phantomsection%
  Doconce: Document Once, Include Anywhere%
  \addcontentsline{toc}{section}{Doconce: Document Once, Include Anywhere}%
  \label{doconce-document-once-include-anywhere}%
}
%
\begin{DUfieldlist}
\item[{Author:}]
Hans Petter Langtangen

\item[{Date:}]
Oct 29, 2011
%
\begin{itemize}

\item When writing a note, report, manual, etc., do you find it difficult
to choose the typesetting format? That is, to choose between plain
(email-like) text, wiki, Word/OpenOffice, LaTeX, HTML,
reStructuredText, Sphinx, XML, etc.  Would it be convenient to
start with some very simple text-like format that easily converts
to the formats listed above, and then at some later stage
eventually go with a particular format?

\item Do you need to write documents in varying formats but find it
difficult to remember all the typesetting details of various
formats like \href{http://refcards.com/docs/silvermanj/amslatex/LaTeXRefCard.v2.0.pdf}{LaTeX}, \href{http://www.htmlcodetutorial.com/}{HTML}, \href{http://docutils.sourceforge.net/docs/ref/rst/restructuredtext.html}{reStructuredText}, \href{http://sphinx.pocoo.org/contents.html}{Sphinx}, and \href{http://code.google.com/p/support/wiki/WikiSyntax}{wiki}? Would it be convenient
to generate the typesetting details of a particular format from a
very simple text-like format with minimal tagging?

\item Do you have the same information scattered around in different
documents in different typesetting formats? Would it be a good idea
to write things once, in one format, stored in one place, and
include it anywhere?

\end{itemize}

\end{DUfieldlist}

If any of these questions are of interest, you should keep on reading.


%___________________________________________________________________________

\section*{\phantomsection%
  The Doconce Concept%
  \addcontentsline{toc}{section}{The Doconce Concept}%
  \label{the-doconce-concept}%
}

Doconce is two things:
%
\begin{quote}
\newcounter{listcnt0}
\begin{list}{\arabic{listcnt0}.}
{
\usecounter{listcnt0}
\setlength{\rightmargin}{\leftmargin}
}

\item Doconce is a very simple and minimally tagged markup language that
looks like ordinary ASCII text (much like what you would use in an
email), but the text can be transformed to numerous other formats,
including HTML, wiki, LaTeX, PDF, reStructuredText (reST), Sphinx,
Epytext, and also plain text (where non-obvious formatting/tags are
removed for clear reading in, e.g., emails). From reStructuredText
you can go to XML, HTML, LaTeX, PDF, OpenOffice, and from the
latter to RTF and MS Word.
(An experimental translator to Pandoc is under development, and from
Pandoc one can generate Markdown, reST, LaTeX, HTML, PDF, DocBook XML,
OpenOffice, GNU Texinfo, MediaWiki, RTF, Groff, and other formats.)

\item Doconce is a working strategy for never duplicating information.
Text is written in a single place and then transformed to
a number of different destinations of diverse type (software
source code, manuals, tutorials, books, wikis, memos, emails, etc.).
The Doconce markup language support this working strategy.
The slogan is: ``Write once, include anywhere''.
\end{list}

\end{quote}

Here are some Doconce features:
%
\begin{quote}
%
\begin{itemize}

\item Doconce markup does include tags, so the format is more tagged than
Markdown and Pandoc, but less than reST, and very much less than
LaTeX and HTML.

\item Doconce can be converted to plain \emph{untagged} text,
often desirable for computer programs and email.

\item Doconce has good support for copying in parts of computer code,
say in examples, directly from the source code files.

\item Doconce has full support for LaTeX math, and integrates very well
with big LaTeX projects (books).

\item Doconce is almost self-explanatory and is a handy starting point
for generating documents in more complicated markup languages, such
as Google wiki, LaTeX, and Sphinx. A primary application of Doconce
is just to make the initial versions of a Sphinx or wiki document.

\item Contrary to the similar Pandoc translator, Doconce integrates with
Sphinx and Google wiki. However, if these formats are not of interest,
Pandoc is obviously a superior tool.

\end{itemize}

\end{quote}

Doconce was particularly written for the following sample applications:
%
\begin{quote}
%
\begin{itemize}

\item Large books written in LaTeX, but where many pieces (computer demos,
projects, examples) can be written in Doconce to appear in other
contexts in other formats, including plain HTML, Sphinx, wiki, or MS Word.

\item Software documentation, primarily Python doc strings, which one wants
to appear as plain untagged text for viewing in Pydoc, as reStructuredText
for use with Sphinx, as wiki text when publishing the software at
web sites, and as LaTeX integrated in, e.g., a thesis.

\item Quick memos, which start as plain text in email, then some small
amount of Doconce tagging is added, before the memos can appear as
Sphinx web pages, MS Word documents, or in wikis.

\end{itemize}

\end{quote}

History: Doconce was developed in 2006 at a time when most popular
markup languages used quite some tagging.  Later, almost untagged
markup languages like Markdown and Pandoc became popular. Doconce is
not a replacement of Pandoc, which is a considerably more
sophisticated project. Moreover, Doconce was developed mainly to
fulfill the needs for a flexible source code base for books with much
mathematics and computer code.

Disclaimer: Doconce is a simple tool, largely based on interpreting
and handling text through regular expressions. The possibility for
tweaking the layout is obviously limited since the text can go to
all sorts of sophisticated markup languages. Moreover, because of
limitations of regular expressions, some formatting of Doconce syntax
may face problems when transformed to HTML, LaTeX, Sphinx, and similar
formats.


%___________________________________________________________________________

\section*{\phantomsection%
  What Does Doconce Look Like?%
  \addcontentsline{toc}{section}{What Does Doconce Look Like?}%
  \label{what-does-doconce-look-like}%
}

Doconce text looks like ordinary text, but there are some almost invisible
text constructions that allow you to control the formating. For example,
%
\begin{quote}
%
\begin{itemize}

\item bullet lists arise from lines starting with an asterisk,

\item \emph{emphasized words} are surrounded by asterisks,

\item \textbf{words in boldface} are surrounded by underscores,

\item words from computer code are enclosed in back quotes and
then typeset verbatim (monospace font),

\item section headings are recognied by equality (\texttt{=}) signs before
and after the text, and the number of \texttt{=} signs indicates the
level of the section (7 for main section, 5 for subsection,
3 for subsubsection),

\item paragraph headings are recognized by a double underscore
before and after the heading,

\item blocks of computer code can easily be included by placing
\texttt{!bc} (begin code) and \texttt{!ec} (end code) commands at separate lines
before and after the code block,

\item blocks of computer code can also be imported from source files,

\item blocks of LaTeX mathematics can easily be included by placing
\texttt{!bt} (begin TeX) and \texttt{!et} (end TeX) commands at separate lines
before and after the math block,

\item there is support for both LaTeX and text-like inline mathematics,

\item figures and movies with captions, simple tables,
URLs with links, index list, labels and references are supported,

\item comments can be inserted throughout the text (\texttt{\#} at the beginning
of a line),

\item with a simple preprocessor, Preprocess or Mako, one can include
other documents (files) and large portions of text can be defined
in or out of the text,

\item with the Mako preprocessor one can even embed Python
code and use this to steer generation of Doconce text.

\end{itemize}

\end{quote}

Here is an example of some simple text written in the Doconce format:
%
\begin{quote}{\ttfamily \raggedright \noindent
=====~A~Subsection~with~Sample~Text~=====\\
label\{my:first:sec\}\\
~\\
Ordinary~text~looks~like~ordinary~text,~and~the~tags~used~for\\
\_boldface\_~words,~*emphasized*~words,~and~`computer`~words~look\\
natural~in~plain~text.~~Lists~are~typeset~as~you~would~do~in~an~email,\\
~\\
~~*~item~1\\
~~*~item~2\\
~~*~item~3\\
~\\
Lists~can~also~have~automatically~numbered~items~instead~of~bullets,\\
~\\
~~o~item~1\\
~~o~item~2\\
~~o~item~3\\
~\\
URLs~with~a~link~word~are~possible,~as~in~"hpl":"http://folk.uio.no/hpl".\\
If~the~word~is~URL,~the~URL~itself~becomes~the~link~name,\\
as~in~"URL":"tutorial.do.txt".\\
~\\
References~to~sections~may~use~logical~names~as~labels~(e.g.,~a\\
"label"~command~right~after~the~section~title),~as~in~the~reference~to\\
Section~ref\{my:first:sec\}.\\
~\\
Doconce~also~allows~inline~comments~such~as~{[}hpl:~here~I~will~make\\
some~remarks~to~the~text{]}~for~allowing~authors~to~make~notes.~Inline\\
comments~can~be~removed~from~the~output~by~a~command-line~argument\\
(see~Section~ref\{doconce2formats\}~for~an~example).\\
~\\
Tables~are~also~supperted,~e.g.,\\
~\\
~~|-{}-{}-{}-{}-{}-{}-{}-{}-{}-{}-{}-{}-{}-{}-{}-{}-{}-{}-{}-{}-{}-{}-{}-{}-{}-{}-{}-{}-{}-{}-{}-|\\
~~|time~~|~velocity~|~acceleration~|\\
~~|-{}-{}-r-{}-{}-{}-{}-{}-{}-r-{}-{}-{}-{}-{}-{}-{}-{}-{}-{}-r-{}-{}-{}-{}-{}-{}-{}-|\\
~~|~0.0~~|~1.4186~~~|~-5.01~~~~~~~~|\\
~~|~2.0~~|~1.376512~|~11.919~~~~~~~|\\
~~|~4.0~~|~1.1E+1~~~|~14.717624~~~~|\\
~~|-{}-{}-{}-{}-{}-{}-{}-{}-{}-{}-{}-{}-{}-{}-{}-{}-{}-{}-{}-{}-{}-{}-{}-{}-{}-{}-{}-{}-{}-{}-{}-|\\
~\\
\#~lines~beginning~with~\#~are~comment~lines
}
\end{quote}

The Doconce text above results in the following little document:


%___________________________________________________________________________

\subsection*{\phantomsection%
  A Subsection with Sample Text%
  \addcontentsline{toc}{subsection}{A Subsection with Sample Text}%
  \label{a-subsection-with-sample-text}%
  \label{my-first-sec}%
}

Ordinary text looks like ordinary text, and the tags used for
\textbf{boldface} words, \emph{emphasized} words, and \texttt{computer} words look
natural in plain text.  Lists are typeset as you would do in an email,
%
\begin{quote}
%
\begin{itemize}

\item item 1

\item item 2

\item item 3

\end{itemize}

\end{quote}

Lists can also have numbered items instead of bullets, just use an \texttt{o}
(for ordered) instead of the asterisk:
%
\begin{quote}
\setcounter{listcnt0}{0}
\begin{list}{\arabic{listcnt0}.}
{
\usecounter{listcnt0}
\setlength{\rightmargin}{\leftmargin}
}

\item item 1

\item item 2

\item item 3
\end{list}

\end{quote}

URLs with a link word are possible, as in \href{http://folk.uio.no/hpl}{hpl}.
If the word is URL, the URL itself becomes the link name,
as in \url{tutorial.do.txt}.

References to sections may use logical names as labels (e.g., a
``label'' command right after the section title), as in the reference to
the section \hyperref[a-subsection-with-sample-text]{A Subsection with Sample Text}.

Doconce also allows inline comments such as (\textbf{hpl}: here I will make
some remarks to the text) for allowing authors to make notes. Inline
comments can be removed from the output by a command-line argument
(see the section \hyperref[from-doconce-to-other-formats]{From Doconce to Other Formats} for an example).

Tables are also supperted, e.g.,

\setlength{\DUtablewidth}{\linewidth}
\begin{longtable*}[c]{|p{0.156\DUtablewidth}|p{0.156\DUtablewidth}|p{0.156\DUtablewidth}|}
\hline
\textbf{%
time
} & \textbf{%
velocity
} & \textbf{%
acceleration
} \\
\hline
\endfirsthead
\hline
\textbf{%
time
} & \textbf{%
velocity
} & \textbf{%
acceleration
} \\
\hline
\endhead
\multicolumn{3}{c}{\hfill ... continued on next page} \\
\endfoot
\endlastfoot

0.0
 & 
1.4186
 & 
-5.01
 \\
\hline

2.0
 & 
1.376512
 & 
11.919
 \\
\hline

4.0
 & 
1.1E+1
 & 
14.717624
 \\
\hline
\end{longtable*}


%___________________________________________________________________________

\subsection*{\phantomsection%
  Mathematics and Computer Code%
  \addcontentsline{toc}{subsection}{Mathematics and Computer Code}%
  \label{mathematics-and-computer-code}%
}

Inline mathematics, such as v = sin(x),
allows the formula to be specified both as LaTeX and as plain text.
This results in a professional LaTeX typesetting, but in other formats
the text version normally looks better than raw LaTeX mathematics with
backslashes. An inline formula like v = sin(x) is
typeset as:
%
\begin{quote}{\ttfamily \raggedright \noindent
\$\textbackslash{}nu~=~\textbackslash{}sin(x)\$|\$v~=~sin(x)\$
}
\end{quote}

The pipe symbol acts as a delimiter between LaTeX code and the plain text
version of the formula.

Blocks of mathematics are better typeset with raw LaTeX, inside
\texttt{!bt} and \texttt{!et} (begin tex / end tex) instructions.
The result looks like this:
%
\begin{quote}{\ttfamily \raggedright \noindent
\textbackslash{}begin\{eqnarray\}\\
\{\textbackslash{}partial~u\textbackslash{}over\textbackslash{}partial~t\}~\&=\&~\textbackslash{}nabla\textasciicircum{}2~u~+~f,~label\{myeq1\}\textbackslash{}\textbackslash{}\\
\{\textbackslash{}partial~v\textbackslash{}over\textbackslash{}partial~t\}~\&=\&~\textbackslash{}nabla\textbackslash{}cdot(q(u)\textbackslash{}nabla~v)~+~g\\
\textbackslash{}end\{eqnarray\}
}
\end{quote}

Of course, such blocks only looks nice in LaTeX. The raw
LaTeX syntax appears in all other formats (but can still be useful
for those who can read LaTeX syntax).

You can have blocks of computer code, starting and ending with
\texttt{!bc} and \texttt{!ec} instructions, respectively. Such blocks look like:
%
\begin{quote}{\ttfamily \raggedright \noindent
from~math~import~sin,~pi\\
def~myfunc(x):\\
~~~~return~sin(pi*x)\\
~\\
import~integrate\\
I~=~integrate.trapezoidal(myfunc,~0,~pi,~100)
}
\end{quote}

It is possible to add a specification of a (ptex2tex-style)
environment for typesetting the verbatim code block, e.g., \texttt{!bc xxx}
where \texttt{xxx} is an identifier like \texttt{pycod} for code snippet in Python,
\texttt{sys} for terminal session, etc. When Doconce is filtered to LaTeX,
these identifiers are used as in ptex2tex and defined in a
configuration file \texttt{.ptext2tex.cfg}, while when filtering
to Sphinx, one can have a comment line in the Doconce file for
mapping the identifiers to legal language names for Sphinx (which equals
the legal language names for Pygments):
%
\begin{quote}{\ttfamily \raggedright \noindent
\#~sphinx~code-blocks:~pycod=python~cod=py~cppcod=c++~sys=console
}
\end{quote}

By default, \texttt{pro} and \texttt{cod} are \texttt{python}, \texttt{sys} is \texttt{console},
while \texttt{xpro} and \texttt{xcod} are computer language specific for \texttt{x}
in \texttt{f} (Fortran), \texttt{c} (C), \texttt{cpp} (C++), and \texttt{py} (Python).
.. \texttt{rb} (Ruby), \texttt{pl} (Perl), and \texttt{sh} (Unix shell).

% (Any sphinx code-block comment, whether inside verbatim code

% blocks or outside, yields a mapping between bc arguments

% and computer languages. In case of muliple definitions, the

% first one is used.)

One can also copy computer code directly from files, either the
complete file or specified parts.  Computer code is then never
duplicated in the documentation (important for the principle of
avoiding copying information!). A complete file is typeset
with \texttt{!bc pro}, while a part of a file is copied into a \texttt{!bc cod}
environment. What \texttt{pro} and \texttt{cod} mean is then defined through
a \texttt{.ptex2tex.cfg} file for LaTeX and a \texttt{sphinx code-blocks}
comment for Sphinx.

Another document can be included by writing \texttt{\#include "mynote.do.txt"}
on a line starting with (another) hash sign.  Doconce documents have
extension \texttt{do.txt}. The \texttt{do} part stands for doconce, while the
trailing \texttt{.txt} denotes a text document so that editors gives you the
right writing enviroment for plain text.


%___________________________________________________________________________

\subsection*{\phantomsection%
  Macros (Newcommands), Cross-References, Index, and Bibliography%
  \addcontentsline{toc}{subsection}{Macros (Newcommands), Cross-References, Index, and Bibliography}%
  \label{macros-newcommands-cross-references-index-and-bibliography}%
  \label{newcommands}%
}

Doconce supports a type of macros via a LaTeX-style \emph{newcommand}
construction.  The newcommands defined in a file with name
\texttt{newcommand\_replace.tex} are expanded when Doconce is filtered to
other formats, except for LaTeX (since LaTeX performs the expansion
itself).  Newcommands in files with names \texttt{newcommands.tex} and
\texttt{newcommands\_keep.tex} are kept unaltered when Doconce text is
filtered to other formats, except for the Sphinx format. Since Sphinx
understands LaTeX math, but not newcommands if the Sphinx output is
HTML, it makes most sense to expand all newcommands.  Normally, a user
will put all newcommands that appear in math blocks surrounded by
\texttt{!bt} and \texttt{!et} in \texttt{newcommands\_keep.tex} to keep them unchanged, at
least if they contribute to make the raw LaTeX math text easier to
read in the formats that cannot render LaTeX.  Newcommands used
elsewhere throughout the text will usually be placed in
\texttt{newcommands\_replace.tex} and expanded by Doconce.  The definitions of
newcommands in the \texttt{newcommands*.tex} files \emph{must} appear on a single
line (multi-line newcommands are too hard to parse with regular
expressions).

Recent versions of Doconce also offer cross referencing, typically one
can define labels below (sub)sections, in figure captions, or in
equations, and then refer to these later. Entries in an index can be
defined and result in an index at the end for the LaTeX and Sphinx
formats. Citations to literature, with an accompanying bibliography in
a file, are also supported. The syntax of labels, references,
citations, and the bibliography closely resembles that of LaTeX,
making it easy for Doconce documents to be integrated in LaTeX
projects (manuals, books). For further details on functionality and
syntax we refer to the \texttt{doc/manual/manual.do.txt} file (see the
\href{https://doconce.googlecode.com/hg/doc/demos/manual/index.html}{demo page}
for various formats of this document).

% Example on including another Doconce file (using preprocess):


%___________________________________________________________________________

\section*{\phantomsection%
  From Doconce to Other Formats%
  \addcontentsline{toc}{section}{From Doconce to Other Formats}%
  \label{from-doconce-to-other-formats}%
  \label{doconce2formats}%
}

Transformation of a Doconce document \texttt{mydoc.do.txt} to various other
formats applies the script \texttt{doconce format}:
%
\begin{quote}{\ttfamily \raggedright \noindent
Terminal>~doconce~format~format~mydoc.do.txt
}
\end{quote}

or just:
%
\begin{quote}{\ttfamily \raggedright \noindent
Terminal>~doconce~format~format~mydoc
}
\end{quote}

The \texttt{mako} or \texttt{preprocess} programs are always used to preprocess the
file first, and options to \texttt{mako} or \texttt{preprocess} can be added after the
filename. For example:
%
\begin{quote}{\ttfamily \raggedright \noindent
Terminal>~doconce~format~LaTeX~mydoc~-Dextra\_sections~-DVAR1=5~~~~~\#~preprocess\\
Terminal>~doconce~format~LaTeX~yourdoc~extra\_sections=True~VAR1=5~~\#~mako
}
\end{quote}

The variable \texttt{FORMAT} is always defined as the current format when
running \texttt{preprocess}. That is, in the last example, \texttt{FORMAT} is
defined as \texttt{LaTeX}. Inside the Doconce document one can then perform
format specific actions through tests like \texttt{\#if FORMAT == "LaTeX"}.

Inline comments in the text are removed from the output by:
%
\begin{quote}{\ttfamily \raggedright \noindent
Terminal>~doconce~format~LaTeX~mydoc~remove\_inline\_comments
}
\end{quote}

One can also remove such comments from the original Doconce file
by running
source code:
%
\begin{quote}{\ttfamily \raggedright \noindent
Terminal>~doconce~remove\_inline\_comments~mydoc
}
\end{quote}

This action is convenient when a Doconce document reaches its final form
and comments by different authors should be removed.


%___________________________________________________________________________

\subsection*{\phantomsection%
  HTML%
  \addcontentsline{toc}{subsection}{HTML}%
  \label{id1}%
}

Making an HTML version of a Doconce file \texttt{mydoc.do.txt}
is performed by:
%
\begin{quote}{\ttfamily \raggedright \noindent
Terminal>~doconce~format~HTML~mydoc
}
\end{quote}

The resulting file \texttt{mydoc.html} can be loaded into any web browser for viewing.


%___________________________________________________________________________

\subsection*{\phantomsection%
  LaTeX%
  \addcontentsline{toc}{subsection}{LaTeX}%
  \label{id2}%
}

Making a LaTeX file \texttt{mydoc.tex} from \texttt{mydoc.do.txt} is done in two steps:
.. Note: putting code blocks inside a list is not successful in many

% formats - the text may be messed up. A better choice is a paragraph

% environment, as used here.
%
\begin{description}
\item[{\emph{Step 1.} Filter the doconce text to a pre-LaTeX form \texttt{mydoc.p.tex} for}] \leavevmode 
\texttt{ptex2tex}:
%
\begin{quote}{\ttfamily \raggedright \noindent
Terminal>~doconce~format~LaTeX~mydoc
}
\end{quote}

\end{description}

LaTeX-specific commands (``newcommands'') in math formulas and similar
can be placed in files \texttt{newcommands.tex}, \texttt{newcommands\_keep.tex}, or
\texttt{newcommands\_replace.tex} (see the section \hyperref[macros-newcommands-cross-references-index-and-bibliography]{Macros (Newcommands), Cross-References, Index, and Bibliography}).
If these files are present, they are included in the LaTeX document
so that your commands are defined.

\emph{Step 2.} Run \texttt{ptex2tex} (if you have it) to make a standard LaTeX file:
%
\begin{quote}{\ttfamily \raggedright \noindent
Terminal>~ptex2tex~mydoc
}
\end{quote}

or just perform a plain copy:
%
\begin{quote}{\ttfamily \raggedright \noindent
Terminal>~cp~mydoc.p.tex~mydoc.tex
}
\end{quote}

Doconce generates a \texttt{.p.tex} file with some preprocessor macros
that can be used to steer certain properties of the LaTeX document.
For example, to turn on the Helvetica font instead of the standard
Computer Modern font, run:
%
\begin{quote}{\ttfamily \raggedright \noindent
Terminal>~ptex2tex~-DHELVETICA~mydoc
}
\end{quote}

The title, authors, and date are by default typeset in a non-standard
way to enable a nicer treatment of multiple authors having
institutions in common. However, the standard LaTeX ``maketitle'' heading
is also available through:
%
\begin{quote}{\ttfamily \raggedright \noindent
Terminal>~ptex2tex~-DLATEX\_HEADING=traditional~mydoc
}
\end{quote}

A separate titlepage can be generate by:
%
\begin{quote}{\ttfamily \raggedright \noindent
Terminal>~ptex2tex~-DLATEX\_HEADING=titlepage~mydoc
}
\end{quote}

The \texttt{ptex2tex} tool makes it possible to easily switch between many
different fancy formattings of computer or verbatim code in LaTeX
documents. After any \texttt{!bc} command in the Doconce source you can
insert verbatim block styles as defined in your \texttt{.ptex2tex.cfg}
file, e.g., \texttt{!bc cod} for a code snippet, where \texttt{cod} is set to
a certain environment in \texttt{.ptex2tex.cfg} (e.g., \texttt{CodeIntended}).
There are over 30 styles to choose from.

\emph{Step 3.} Compile \texttt{mydoc.tex}
and create the PDF file:
%
\begin{quote}{\ttfamily \raggedright \noindent
Terminal>~latex~mydoc\\
Terminal>~latex~mydoc\\
Terminal>~makeindex~mydoc~~~\#~if~index\\
Terminal>~bibitem~mydoc~~~~~\#~if~bibliography\\
Terminal>~latex~mydoc\\
Terminal>~dvipdf~mydoc
}
\end{quote}

If one wishes to use the \texttt{Minted\_Python}, \texttt{Minted\_Cpp}, etc., environments
in \texttt{ptex2tex} for typesetting code, the \texttt{minted} LaTeX package is needed.
This package is included by running \texttt{doconce format} with the
\texttt{-DMINTED} option:
%
\begin{quote}{\ttfamily \raggedright \noindent
Terminal>~ptex2tex~-DMINTED~mydoc
}
\end{quote}

In this case, \texttt{latex} must be run with the
\texttt{-shell-escape} option:
%
\begin{quote}{\ttfamily \raggedright \noindent
Terminal>~latex~-shell-escape~mydoc\\
Terminal>~latex~-shell-escape~mydoc\\
Terminal>~makeindex~mydoc~~~\#~if~index\\
Terminal>~bibitem~mydoc~~~~~\#~if~bibliography\\
Terminal>~latex~-shell-escape~mydoc\\
Terminal>~dvipdf~mydoc
}
\end{quote}

The \texttt{-shell-escape} option is required because the \texttt{minted.sty} style
file runs the \texttt{pygments} program to format code, and this program
cannot be run from \texttt{latex} without the \texttt{-shell-escape} option.


%___________________________________________________________________________

\subsection*{\phantomsection%
  Plain ASCII Text%
  \addcontentsline{toc}{subsection}{Plain ASCII Text}%
  \label{plain-ascii-text}%
}

We can go from Doconce ``back to'' plain untagged text suitable for viewing
in terminal windows, inclusion in email text, or for insertion in
computer source code:
%
\begin{quote}{\ttfamily \raggedright \noindent
Terminal>~doconce~format~plain~mydoc.do.txt~~\#~results~in~mydoc.txt
}
\end{quote}


%___________________________________________________________________________

\subsection*{\phantomsection%
  reStructuredText%
  \addcontentsline{toc}{subsection}{reStructuredText}%
  \label{id3}%
}

Going from Doconce to reStructuredText gives a lot of possibilities to
go to other formats. First we filter the Doconce text to a
reStructuredText file \texttt{mydoc.rst}:
%
\begin{quote}{\ttfamily \raggedright \noindent
Terminal>~doconce~format~rst~mydoc.do.txt
}
\end{quote}

We may now produce various other formats:
%
\begin{quote}{\ttfamily \raggedright \noindent
Terminal>~rst2html.py~~mydoc.rst~>~mydoc.html~\#~HTML\\
Terminal>~rst2latex.py~mydoc.rst~>~mydoc.tex~~\#~LaTeX\\
Terminal>~rst2xml.py~~~mydoc.rst~>~mydoc.xml~~\#~XML\\
Terminal>~rst2odt.py~~~mydoc.rst~>~mydoc.odt~~\#~OpenOffice
}
\end{quote}

The OpenOffice file \texttt{mydoc.odt} can be loaded into OpenOffice and
saved in, among other things, the RTF format or the Microsoft Word format.
That is, one can easily go from Doconce to Microsoft Word.


%___________________________________________________________________________

\subsection*{\phantomsection%
  Sphinx%
  \addcontentsline{toc}{subsection}{Sphinx}%
  \label{id4}%
}

Sphinx documents demand quite some steps in their creation. We have automated
most of the steps through the \texttt{doconce sphinx\_dir} command:
%
\begin{quote}{\ttfamily \raggedright \noindent
Terminal>~doconce~sphinx\_dir~author="authors'~names"~\textbackslash{}\\
~~~~~~~~~~title="some~title"~version=1.0~dirname=sphinxdir~\textbackslash{}\\
~~~~~~~~~~theme=mytheme~file1~file2~file3~...
}
\end{quote}

The keywords \texttt{author}, \texttt{title}, and \texttt{version} are used in the headings
of the Sphinx document. By default, \texttt{version} is 1.0 and the script
will try to deduce authors and title from the doconce files \texttt{file1},
\texttt{file2}, etc. that together represent the whole document. Note that
none of the individual Doconce files \texttt{file1}, \texttt{file2}, etc. should
include the rest as their union makes up the whole document.
The default value of \texttt{dirname} is \texttt{sphinx-rootdir}. The \texttt{theme}
keyword is used to set the theme for design of HTML output from
Sphinx (the default theme is \texttt{'default'}).

With a single-file document in \texttt{mydoc.do.txt} one often just runs:
%
\begin{quote}{\ttfamily \raggedright \noindent
Terminal>~doconce~sphinx\_dir~mydoc
}
\end{quote}

and then an appropriate Sphinx directory \texttt{sphinx-rootdir} is made with
relevant files.

The \texttt{doconce sphinx\_dir} command generates a script
\texttt{automake-sphinx.sh} for compiling the Sphinx document into an HTML
document.  This script copies directories named \texttt{figs} or \texttt{figures}
over to the Sphinx directory so that figures are accessible in the
Sphinx compilation.  If figures or movies are located in other
directories, \texttt{automake-sphinx.sh} must be edited accordingly. One
can either run \texttt{automake-sphinx.sh} or perform the steps in the
script manually.

Doconce comes with a collection of HTML themes for Sphinx documents.
These are packed out in the Sphinx directory, the \texttt{conf.py}
configuration file for Sphinx is edited accordingly, and a script
\texttt{make-themes.sh} can make HTML documents with one or more themes.
For example,
to realize the themes \texttt{fenics} and \texttt{pyramid}, one writes:
%
\begin{quote}{\ttfamily \raggedright \noindent
Terminal>~./make-themes.sh~fenics~pyramid
}
\end{quote}

The resulting directories with HTML documents are \texttt{\_build/html\_fenics}
and \texttt{\_build/html\_pyramid}, respectively. Without arguments,
\texttt{make-themes.sh} makes all available themes (!).

If it is not desirable to use the autogenerated scripts explained
above, here are the complete manual procedure of generating a
Sphinx document from a file \texttt{mydoc.do.txt}.

\emph{Step 1.} Translate Doconce into the Sphinx dialect of
the reStructuredText format:
%
\begin{quote}{\ttfamily \raggedright \noindent
Terminal>~doconce~format~sphinx~mydoc
}
\end{quote}

\emph{Step 2.} Create a Sphinx root directory with a \texttt{conf.py} file,
either manually or by using the interactive \texttt{sphinx-quickstart}
program. Here is a scripted version of the steps with the latter:
%
\begin{quote}{\ttfamily \raggedright \noindent
mkdir~sphinx-rootdir\\
sphinx-quickstart~<{}<EOF\\
sphinx-rootdir\\
n\\
\_\\
Name~of~My~Sphinx~Document\\
Author\\
version\\
version\\
.rst\\
index\\
n\\
y\\
n\\
n\\
n\\
n\\
y\\
n\\
n\\
y\\
y\\
y\\
EOF
}
\end{quote}

\emph{Step 3.} Copy the \texttt{tutorial.rst} file to the Sphinx root directory:
%
\begin{quote}{\ttfamily \raggedright \noindent
Terminal>~cp~mydoc.rst~sphinx-rootdir
}
\end{quote}

If you have figures in your document, the relative paths to those will
be invalid when you work with \texttt{mydoc.rst} in the \texttt{sphinx-rootdir}
directory. Either edit \texttt{mydoc.rst} so that figure file paths are correct,
or simply copy your figure directories to \texttt{sphinx-rootdir}.

\emph{Step 4.} Edit the generated \texttt{index.rst} file so that \texttt{mydoc.rst}
is included, i.e., add \texttt{mydoc} to the \texttt{toctree} section so that it becomes:
%
\begin{quote}{\ttfamily \raggedright \noindent
..~toctree::\\
~~~:maxdepth:~2\\
~\\
~~~mydoc
}
\end{quote}

(The spaces before \texttt{mydoc} are important!)

\emph{Step 5.} Generate, for instance, an HTML version of the Sphinx source:
%
\begin{quote}{\ttfamily \raggedright \noindent
make~clean~~~\#~remove~old~versions\\
make~html
}
\end{quote}

\emph{Step 6.} View the result:
%
\begin{quote}{\ttfamily \raggedright \noindent
Terminal>~firefox~\_build/html/index.html
}
\end{quote}

Note that verbatim code blocks can be typeset in a variety of ways
depending the argument that follows \texttt{!bc}: \texttt{cod} gives Python
(\texttt{code-block:: python} in Sphinx syntax) and \texttt{cppcod} gives C++, but
all such arguments can be customized both for Sphinx and LaTeX output.


%___________________________________________________________________________

\subsection*{\phantomsection%
  Google Code Wiki%
  \addcontentsline{toc}{subsection}{Google Code Wiki}%
  \label{google-code-wiki}%
}

There are several different wiki dialects, but Doconce only support the
one used by \href{http://code.google.com/p/support/wiki/WikiSyntax}{Google Code}.
The transformation to this format, called \texttt{gwiki} to explicitly mark
it as the Google Code dialect, is done by:
%
\begin{quote}{\ttfamily \raggedright \noindent
Terminal>~doconce~format~gwiki~mydoc.do.txt
}
\end{quote}

You can then open a new wiki page for your Google Code project, copy
the \texttt{mydoc.gwiki} output file from \texttt{doconce format} and paste the
file contents into the wiki page. Press \textbf{Preview} or \textbf{Save Page} to
see the formatted result.

When the Doconce file contains figures, each figure filename must be
replaced by a URL where the figure is available. There are instructions
in the file for doing this. Usually, one performs this substitution
automatically (see next section).


%___________________________________________________________________________

\subsection*{\phantomsection%
  Tweaking the Doconce Output%
  \addcontentsline{toc}{subsection}{Tweaking the Doconce Output}%
  \label{tweaking-the-doconce-output}%
}

Occasionally, one would like to tweak the output in a certain format
from Doconce. One example is figure filenames when transforming
Doconce to reStructuredText. Since Doconce does not know if the
\texttt{.rst} file is going to be filtered to LaTeX or HTML, it cannot know
if \texttt{.eps} or \texttt{.png} is the most appropriate image filename.
The solution is to use a text substitution command or code with, e.g., sed,
perl, python, or scitools subst, to automatically edit the output file
from Doconce. It is then wise to run Doconce and the editing commands
from a script to automate all steps in going from Doconce to the final
format(s). The \texttt{make.sh} files in \texttt{docs/manual} and \texttt{docs/tutorial}
constitute comprehensive examples on how such scripts can be made.


%___________________________________________________________________________

\subsection*{\phantomsection%
  Demos%
  \addcontentsline{toc}{subsection}{Demos}%
  \label{demos}%
}

The current text is generated from a Doconce format stored in the file:
%
\begin{quote}{\ttfamily \raggedright \noindent
docs/tutorial/tutorial.do.txt
}
\end{quote}

The file \texttt{make.sh} in the \texttt{tutorial} directory of the
Doconce source code contains a demo of how to produce a variety of
formats.  The source of this tutorial, \texttt{tutorial.do.txt} is the
starting point.  Running \texttt{make.sh} and studying the various generated
files and comparing them with the original \texttt{tutorial.do.txt} file,
gives a quick introduction to how Doconce is used in a real case.
\href{https://doconce.googlecode.com/hg/doc/demos/tutorial/index.html}{Here}
is a sample of how this tutorial looks in different formats.

There is another demo in the \texttt{docs/manual} directory which
translates the more comprehensive documentation, \texttt{manual.do.txt}, to
various formats. The \texttt{make.sh} script runs a set of translations.


%___________________________________________________________________________

\subsection*{\phantomsection%
  Dependencies%
  \addcontentsline{toc}{subsection}{Dependencies}%
  \label{dependencies}%
}

\DUadmonition[system-message]{
\DUtitle[system-message]{system-message}


{\color{red}WARNING/2} in \texttt{tutorial.rst}, line~385

\hyperlink{id5}{
Duplicate explicit target name: ``sphinx''.
}}

If you make use of preprocessor directives in the Doconce source,
either \href{http://code.google.com/p/preprocess}{Preprocess} or \href{http://www.makotemplates.org}{Mako} must be installed.  To make LaTeX
documents (without going through the reStructuredText format) you also
need \href{http://code.google.com/p/ptex2tex}{ptex2tex} and some style
files that \texttt{ptex2tex} potentially makes use of.  Going from
reStructuredText to formats such as XML, OpenOffice, HTML, and LaTeX
requires \href{http://docutils.sourceforge.net}{docutils}.  Making Sphinx
documents requires of course \href{http://sphinx.pocoo.org}{Sphinx}.
All of the mentioned potential dependencies are pure Python packages
which are easily installed.
If translation to \href{http://johnmacfarlane.net/pandoc/}{Pandoc} is desired,
the Pandoc Haskell program must of course be installed.

\end{document}
